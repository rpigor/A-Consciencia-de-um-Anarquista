% Recursos

\chapter[Recursos]{Recursos\\\vspace{-1.75ex}\hrule\vspace{0.5ex}\normalfont\LARGE Materiais para se conferir no caminho\vspace{-0.75ex}\\para o futuro}
\label{chap:rec}

\lettrine[lines=2]{\textcolor{LettrineColor}{\textbf{E}}}{xiste uma} enorme literatura sobre anarquismo. Como enfatizei, tenho extraído muito dela livremente. Quero listar aqui as fontes contemporâneas que contribuíram significativamente para o desenvolvimento deste livro, uma vez que nem todas são admitidas nas notas. Também quero chamar a atenção para alguns textos (literários e cinematográficos) que qualquer pessoa interessada no anarquismo talvez queira se familiarizar. O fato de eu ter incluído algo aqui não significa necessariamente que compartilhe sua perspectiva, é claro, mas apenas que acho interessante e que vale a pena explorar.

\bigskip

% que Deus me perdoe pela gambiarra
\setlength{\parskip}{6pt}
\begin{hangparas}{0.25in}{1}
\emph{Anarchism in America}. Dir. Steven Fischler e Joel Sucher. Perf. Murray Bookchin, Paul Avrich, Jello Biafra, Mollie Steimer, Mildred Loomis, Karl Hess, et al. Pacific Street 1983. DVD. AK 2005.\smallskip\\
Um documentário expressivo que fornece uma visão geral dos pensadores e ativistas anarquistas americanos que representam diversas escolas e contextos desde o século XIX até o presente.

\emph{An Anarchist FAQ}. By Iain McKay et al. InfoShop.org, Jan. 21, 2010 <\url{http://www.infoshop.org/page/AnAnarchistFAQ}> July 2, 2010.\smallskip\\
Uma influente exposição do anarquismo, apresentando contribuições de anarquistas com uma variedade de pontos de vista diferentes, discussões de argumentos a favor e contra o anarquismo e análises de diversas escolas de pensamento anarquista.

\emph{AntiWar.com}. Randolph Bourne Institute n.d. <\url{http://antiwar.com}> Jan. 27, 2011.\smallskip\\
Uma das principais fontes de notícias e comentários designados a desafiar o militarismo, o imperialismo e o estado de segurança nacional, operado por uma organização sem fins lucrativos com o nome do crítico social que disse a famosa frase: ``A guerra é a vitalidade do estado.'' É de orientação ideológica mista, mas uma boa quantidade de pessoas associadas a ela são anarquistas -- o que não é de surpreender, já que a oposição à violência de guerra estatista é um compromisso anarquista fundamental.

Avrich, Paul. \emph{Anarchist Voices: An Oral History of Anarchism in America}. Oakland, CA: AK 2005.\smallskip\\
Preciosos materiais relacionados à tradição anarquista nos Estados Unidos.

Bakunin, Mikhail Aleksandrovich. \emph{God and the State}. Mineola, NY: Dover 1970.

------. \emph{Statism and Anarchy}. Ed. Marshall Shatz. Cambridge: CUP 1990.\smallskip\\
Escritos anarquistas de um russo intenso, competidor de Karl Marx, que via a religião e o estatismo como igualmente ilusórios e acreditava que as ideias de Marx poderiam ser usadas para justificar a ditadura.

Barnett, Randy E. \emph{The Structure of Liberty: Justice and the Rule of Law}. New York: OUP 2000.\smallskip\\
O ilustre livro de Barnett fornece uma distintiva fundamentação da lei natural para a ordem jurídica de uma sociedade sem estado e explica como tal ordem pode funcionar.

Berkman, Alexander. \emph{What Is Anarchism?} Oakland, CA: AK 2003 [1937].\smallskip\\
Uma exposição simples e clara do que o autor rotulou como ``anarquismo comunista'', por um amigo de longa data e em certo ponto amante de Emma Goldman.

Bookchin, Murray. \emph{Post-Scarcity Anarchism}. 3d ed. Stirling: AK 2004.\smallskip\\
Uma clássica série de artigos do principal pensador social anarquista da América pós-Segunda Guerra.

Caplan, Bryan. \emph{Anarchist Theory FAQ: Or, Instead of a FAQ , by a Man Too Busy to Write One}. Version 5.2. N.p. n.d. <\url{http://econfaculty.gmu.edu/bcaplan/anarfaq.htm}>. July 2, 2010.\smallskip\\
Uma visão geral clara e abrangente de questões relacionadas ao anarquismo por um economista acadêmico que também escreve sobre filosofia e política.

Carson, Kevin A. ``The Distorting Effects of Transportation Subsidies.'' \emph{The Freeman: Ideas on Liberty} 60.9 (Nov. 2010): 17-20. <\url{http://www.thefreemanonline.org/featured/the-distorting-effects-of-transportation-subsidies}>

------. ``Health Care and Radical Monopoly.'' \emph{The Freeman: Ideas on Liberty} 60.2 (March 2010): 8-11. <\url{http://www.thefreemanonline.org/featured/healthcare-and-radical-monopoly}>

------. \emph{The Homebrew Industrial Revolution: A Low-Overhead Manifesto}. Charleston, SC: BookSurge 2010. N.p. 2010. <\url{http://homebrewindustrialrevolution.wordpress.com}>. July 2, 2010.

------. \emph{Organization Theory: A Libertarian Perspective}. Charleston, SC: BookSurge 2009.

------. \emph{Studies in Mutualist Political Economy}. Charleston, SC: BookSurge 2007. Mutualist.org, 2007. <\url{http://www.mutualist.org/id47.html}>. July 2, 2010.\smallskip\\
Carson é um brilhante e criativo sintetista e reinterprete da tradição anarquista, recorrendo tanto a clássicos do século XIX, como os escritos de Proudhon e Tucker, quanto a escritos mais recentes sobre história, economia e teoria política. Veja os materiais mutualistas em seu site -- <\url{http://www.mutualist.org}> -- e interaja com ele online em seu blog -- <\url{http://www.mutualist.blogspot.com}>.

\emph{Center for a Stateless Society}. Ed. Brad Spangler. Molinari Institute n.d. <\url{http://c4ss.org/}>. July 2, 2010.\smallskip\\
Artigos, comentários e outros materiais criticando o estado e seus aliados corporativos e imaginando alternativas sem estado. (Faço parte do conselho consultivo do \emph{Center}.)

Chomsky, Noam. \emph{Chomsky on Anarchism}. Oakland: AK 2006.\smallskip\\
Não apenas um linguista teórico de primeira linha e um articulado crítico de longa data da política externa do governo dos EUA, Chomsky também está entre os principais pensadores anarquistas sociais que escrevem nos tempos de hoje.

Clark, Stephen R. L. \emph{Civil Peace and Sacred Order}. Oxford: Clarendon-OUP 1989.

------. ``Slaves and Citizens'' e ``Anarchists against the Revolution.'' \emph{The Political Animal: Biology, Ethics, and Politics}. London: Routledge 1999.\smallskip\\
Clark apresenta uma versão atrativa do anarquismo para considerarmos e destaca semelhanças entre escolas anarquistas aparentemente divergentes. Embora, como ficará claro, eu me encontre politicamente à esquerda, seu ``anarco-conservadorismo'' tem se mostrado um valor permanente no meu próprio pensamento.

DeLeon, David. \emph{The American as Anarchist: Reflections on Indigenous Radicalism}. Baltimore, MD: Johns Hopkins UP 1978.\smallskip\\
Uma análise histórica do anarquismo como reflexo de uma persistente tendência anti-autoritária no pensamento americano.

De Cleyre, Voltairine. \emph{The Voltairine De Cleyre Reader}. Ed. A. J. Brigati. Oakland, CA: AK 2004.\smallskip\\
Uma importante anarquista americana no final do século XIX e início do século XX, De Cleyre cunhou a expressão ``anarquismo sem adjetivos''.

Ellul, Jacques. \emph{Anarchy and Christianity}. Trans. Geoffrey W. Bromiley. Grand Rapids: Eerdmans 1988.\smallskip\\
O importante e prolífico teórico social argumenta que o anarquismo não violento é uma apropriada expressão da fé cristã.

Friedman, David D. \emph{The Machinery of Freedom: Guide to a Radical Capitalism}. Chicago: IL: Open Court 1989. N.p. 2010. <\url{http://www.daviddfriedman.com/The_Machinery_of_Freedom_.pdf}>. July 1, 2010.\smallskip\\
Uma exposição clara, bem humorada e criativa do argumento a favor de uma variante de anarquismo orientada ao mercado através de uma perspectiva econômica.

Goldman, Emma. \emph{Anarchism and Other Essays}. New York: Mother Earth 1910.

------. \emph{Living My Life}. New York: Knopf 1931.\smallskip\\
Anarquista e feminista que uniu os movimentos anarquistas nos Estados Unidos e na Europa, Goldman disse a famosa frase: ``Eu quero liberdade, o direito à auto-expressão, o direito de todos a coisas belas e radiantes.''

Goodway, David, ed. \emph{For Anarchism: History, Theory, and Practice}. London: Routledge 1989.\smallskip\\
Uma coletânea de artigos examinando os estágios iniciais do anarquismo do século XX e oferecendo perspectivas diversas sobre a teoria anarquista.

Graeber, David. \emph{Fragments of an Anarchist Anthropology}. Chicago: Prickly Paradigm 2004. <\url{http://www.prickly-paradigm.com/paradigm14.pdf}>. July 3, 2010.\smallskip\\
Um compacto programa de desenvolvimento de uma teoria social anarquista, estabelecendo os fundamentos para as discussões sobre o estado, associações voluntárias e resistência, por um estudioso descrito por um ilustre colega como ``o melhor teórico antropológico de sua geração em qualquer lugar do mundo''.

Graham, Robert, ed. \emph{Anarchism: A Documentary History of Libertarian Ideas 1: From Anarchy to Anarchism (300CE to 1939)}. Montreal: Black Rose 2004.

------. \emph{Anarchism: A Documentary History of Libertarian Ideas 2: The Emergence of the New Anarchism (1939-1977)}. Montreal: Black Rose 2009.

------. \emph{Anarchism: A Documentary History of Libertarian Ideas 3: The New Anarchism (1974 to 2008)}. Montreal: Black Rose 2010.\smallskip\\
Uma vasta coletânea de materiais anarquista desde antes da Idade Média até o presente. As seleções são extraídas de diversas tradições intelectuais e culturais, muitas delas não ocidentais.

Guérin, Daniel. \emph{Anarchism: From Theory to Practice}. Trans. Mary Klopper. New York: Monthly Review 1970. Anarchist Library 2009. <\url{http://theanarchistlibrary.org/HTML/Daniel_Guerin__Anarchism__From_Theory_to_Practice.html}>. July 3, 2010.\smallskip\\
Uma influente visão geral da história e teoria anarquista que também apresenta informações históricas sobre experimentos anarquistas do século XX.

Hess, Karl. \emph{Dear America}. New York: Morrow 1975.\smallskip\\
Um redator-de-discursos-do-Goldwater-convertido-a-\emph{New-Leftist} explica sua convicção de que o anarquismo expressa os ideais americanos da melhor forma.

Higgs, Robert. \emph{Crisis and Leviathan: Critical Episodes in the Growth of American Government}. Oxford: OUP 1982.\smallskip\\
Um historiador econômico anarquista analisa cuidadosamente a ligação entre as crises econômicas, políticas e militares e o desenvolvimento canceroso do estado americano.

Holterman, Thom, e van Maarseveen, Henk, eds. \emph{Law and Anarchism}. Montreal: Black Rose 1984.\smallskip\\
Artigos sobre os problemas jurídicos de uma sociedade sem estado sob diversas perspectivas.

Jasay, Anthony de. \emph{Social Contract, Free Ride: A Study of the Public Goods Problem}. Oxford: Clarendon-OUP 1991.\smallskip\\
Um economista e filósofo argumenta, com base na teoria da escolha racional, que a ordem social é possível sem o estado, mas que o surgimento do estado é um perigo persistente.

Johnson, Charles W. \emph{Rad Geek's People's Daily: Official State Media for a Secessionist Republic of One}. N.p. 2010. <\url{http://radgeek.com}>. July 2, 2010.

------. ``Scratching By: How Government Creates Poverty as We Know It.'' \emph{The Freeman:Ideas on Liberty} 57.10 (Sep. 2007): 12-13.<\url{http://www.thefreemanonline.org/Featured/Scratching-By-How-Government-Creates-Poverty-as-We-KnowIt}>. July 2, 2010.\smallskip\\
Analítico, ardente, implacável -- Johnson oferece uma mistura de argumento filosófico, comentário político e cuspir fatos na sua cara. Especialmente forte nas diversas conexões entre a oposição ao estado e os desafios a vários tipos de subordinação não estatal culturalmente incorporadas, tanto violentas quanto não violentas.

\emph{Karl Hess: Toward Liberty}. Dir. Roland Hallé e Peter W. Ladue. Direct Cinema 1980.\smallskip\\
Um retrato vencedor do Oscar do gentil e razoável pensador e ativista anarquista e defensor do empoderamento local.

Kauffman, Bill. \emph{Bye Bye Miss American Empire: Neighborhood Patriots, Backcountry Rebels, and Their Underdog Crusades to Redraw America's Political Map}. White River Junction, VT: Chelsea Green 2010.\smallskip\\
Um bem informado proponente do contra do ``\emph{anarquismo front-porch}'' escreve um poema de amor aos movimentos separatistas razoáveis, tanto do passado quanto do presente.

Kinsella, N. Stephan. \emph{Against Intellectual Property}.Auburn, AL: Mises 2009. <\url{http://www.mises.org/books/against.pdf}>. Feb. 17, 2011.\smallskip\\
Um advogado anarquista e teórico jurídico oferece argumento sólido de que patentes, direitos autorais e outras formas de PI são criações injustas do estado.

Kolko, Gabriel. \emph{The Triumph of Conservatism: A Reinterpretation of American History, 1900- 1916}. New York: Free 1963. <\url{http://www.scribd.com/doc/17413331/Gabriel-Kolko-The-Triumph-of-Conservatism}>. Feb. 17, 2011.\smallskip\\
Apesar dele mesmo não ser anarquista, Kolko fornece muita munição para as críticas anarquistas ao estado nesse estudo de como as regulamentações da era progressista foram moldadas para servir aos interesses das grandes empresas.

Kropotkin, Peter. \emph{The Conquest of Bread}. New York: Vanguard 1995 [1894]. Project Gutenberg n.d. <\url{http://www.gutenberg.org/etext/23428}>. July 2, 2010.

------. \emph{Mutual Aid. A Factor of Evolution}. London: Freedom 1998 [1914]. Project Gutenberg n.d. <\url{http://www.Gutenberg.Org/etext/4341}>. July 2, 2010.\smallskip\\
Biólogo, geógrafo e teórico social, Kropotkin articulou uma visão da anarquia enfatizando a cooperação e o enraizamento no mundo natural.

Leeson, Peter T. \emph{The Invisible Hook: The Hidden Economics of Pirates}. Princeton: Princeton UP 2009.\smallskip\\
Um divertido exame de algumas questões complexas relacionadas à interação social através das lentes fornecidas pelos piratas do século XVIII, que oferecem evidências talvez surpreendentes da possibilidade de cooperação sem o estado.

Long, Roderick T. \emph{Austro-Athenian Empire}. N.p. 2010. <\url{http://aaeblog.com}>. July 2, 2010.\smallskip\\
Aqui e em seu site -- <\url{http://praxeology.net}> -- Long oferece uma prazerosa mistura de filosofia, política e extravagantes divagações culturais \emph{pop}. O trabalho filosófico completamente agradável de se ler inclui discussões sobre anarquismo, análise de classe, método econômico, feminismo e ética.

Long, Roderick T., e Machan, Tibor. \emph{Anarchism/Minarchism: Is a Government Part of a Free Country?} Farnham: Ashgate 2008.\smallskip\\
Argumentos atualizados de pessoas que acreditam que não deveria existir um estado e pessoas que acreditam que deveria haver estados muito limitados.

Martin, James J. \emph{Men against the State: The Expositors of Individualist Anarchism in America, 1827-1908}. Colorado Springs: Myles 1970. <\url{http://www.mises.org/books/Men_Against_the_State_Martin.pdf}>\smallskip\\
Um atraente retrato de muitos dos principais anarquistas do século XIX nos EUA, visto em relação a movimentos radicais contemporâneos.

Meltzer, Albert. \emph{Anarchism: Arguments For and Against}. Oakland, CA: AK 2001.\smallskip\\
Uma breve cartilha sobre ideias anarquistas projetadas para responder críticas marxistas. Desconsidera pensadores anarquistas incluindo Tolstói, Tucker e Proudhon.

Murphy, Robert P. \emph{Chaos Theory: Two Essays on Market Anarchy}. New York: RJ 2002. Mises Institute n.d. <\url{http://www.mises.org/books/chaostheory.pdf}>. July 2, 2010.\smallskip\\
Uma discussão útil sobre a gestão daqueles potencialmente violentos em uma sociedade sem estado. Murphy escreve em <\url{http://consultingbyrpm.com/blog}>.

Nozick, Robert. \emph{Anarchy, State, and Utopia}. New York: Basic 1974.\smallskip\\
Apresenta um argumento pouco convincente mas influente de que um estado poderia emergir legitimamente em uma sociedade sem estado.

Proudhon, Pierre J. \emph{General Idea of the Revolution in the Nineteenth Century}. Trans. John Beverly Robinson. Mineola, NY: Dover 2004 [1923]

------. \emph{System of Economical Contradictions}; or, \emph{The Philosophy of Misery}. Trans. Benjamin R. Tucker. New York: Arno 1973 [1888].

------. \emph{What is Property?} Ed. e trans. David R. Kelley e Bonnie G. Smith. Cambridge: CUP 1994.\smallskip\\
Provavelmente a primeira pessoa a usar a palavra ``anarquista'' para si mesmo, Proudhon competiu com Marx e desenvolveu uma abordagem distinta do anarquismo que ele rotulou de ``mutualismo''.

Richman, Sheldon. ``Libertarian Left: Free-Market Anti-Capitalism, the Unknown Ideal.'' \emph{The American Conservative} 10.3 (March 2011): 28-32. <\url{http://www.amconmag.com/blog/libertarian-left}>. Feb. 17, 2011.\smallskip\\
Fornece uma visão geral de um movimento intelectual e político em desenvolvimento na esquerda política, habitado em grande parte por anarquistas, que curiosamente une algumas cisões ideológicas e estratégicas tradicionais.

Rocker, Rudolph. \emph{Anarcho-Syndicalism: Theory and Practice}. Oakland, CA: AK 2004.

------. \emph{Pioneers of American Freedom: Origins of Liberal and Radical Thought in America}. Trans. Arthur E. Briggs. Los Angeles: Rocker 1949.\smallskip\\
\emph{Anarcho-Syndicalism} é um clássico da estratégia, teoria e história anarquista que enfatiza o compromisso com a liberdade compartilhada por proponentes de diferentes tendências anarquistas. \emph{Pioneers} é uma apreciação das tradições radicais americanas indígenas da perspectiva de um titã do anarquismo europeu.

Rothbard, Murray N. ``Confiscation and the Homestead Principle.'' \emph{Libertarian Forum} 1.6 (June 15, 1969): 3-4. Mises Institute n.d. <\url{http://mises.org/journals/lf/1969/1969_06_15.aspx}>. Feb. 7, 2010.

------. \emph{The Ethics of Liberty}. 2d ed. New York: New York UP 2003. Mises Institute n.d. <\url{http://mises.org/rothbard/ethics/ethics.asp}>. July 2, 2010.\smallskip\\
``Confiscation and the Homestead Principle'' elabora um modelo provocativo de redistribuição de riqueza sem o estado. \emph{The Ethics of Liberty} é uma exposição detalhada da base normativa da variante preferida do anarquismo de Rothbard.

Ruwart, Mary. \emph{Healing Our World in an Age of Aggression}. 3d ed. Kalamazoo, MI: Sunstar 2003.\smallskip\\
Uma discussão vívida e perspicaz dos métodos institucionais e pessoais de solução de problemas para além da ação do estado.

Sartwell, Crispin. \emph{Against the State: An Introduction to Anarchist Political Theory}. Buffalo, NY: SUNY 2008.\smallskip\\
Demonstra a incoerência dos argumentos tradicionais a favor da autoridade do estado.

Scott, James C. \emph{The Art of Not Being Governed: An Anarchist History of Upland Southeast Asia}. New Haven, CT: Yale UP 2009.\smallskip\\
Um estudo da anarquia em ação em uma região que inclui territórios reivindicados por sete países diferentes no Sudeste Asiático -- um perspicaz estudo de caso histórico e contemporâneo.

Shaffer, Butler. \emph{In Restraint of Trade: The Business Campaign against Competition}, 1918- 1938. Lewisburg, PA: Bucknell UP 1997.\smallskip\\
Um teórico legal anarquista explica como os grandes negócios usaram o estado regulatório para sua vantagem em um período crucial na história americana.

Skoble, Aeon J. \emph{Deleting the State: An Argument about Government}. Chicago: Open Court 2008.\smallskip\\
Habilmente disseca argumentos a favor da necessidade e legitimidade do estado.

Spooner, Lysander. \emph{No Treason: The Constitution of No Authority}; and, \emph{A Letter to Thomas F. Bayard. Larkspur}, CO: Pine Tree 1966.\smallskip\\
Spooner se junta a Benjamin Tucker como um dos dois anarquistas individualistas americanos do século XIX que exerceram a mais contínua influência no pensamento anarquista tardio. Spooner era um ardente adversário da escravidão, bem como do estado monopolista, e um fornecedor mal-sucedido dos mesmos serviços que o Serviço Postal dos Estados Unidos que foi efetivamente colocado para fora de jogo por ação do Congresso. Muitos dos escritos de Spooner estão disponíveis em formato eletrônico em <\url{http://www.lysanderspooner.org}>.

Stringham, Edward P., ed. \emph{Anarchy and the Law: The Political Economy of Choice}. Edison, NJ: Transaction 2007.

------. \emph{Anarchy, State, and Public Choice}. Cheltenham: Elgar 2006.\smallskip\\
\emph{Anarchy and the Law} é uma enorme coletânea de artigos históricos e contemporâneos interessados nos problemas morais e práticos levantados pelo anarquismo. \emph{Anarchy, State, and Public Choice} foca em argumentos econômicos a favor e contra a viabilidade do anarquismo.

Tannehill, Morris e Tannehill, Linda. \emph{The Market for Liberty}. 3d ed. San Francisco, CA: Fox 1993.\smallskip\\
Uma explicação detalhada de como os autores acreditam que a cooperação social poderia ser administrada sem o estado. Eles às vezes dizem algumas coisas bem bobas. Mas também exibem uma sensibilidade humana, \emph{hippie} que eu confesso achar encantadora.

Taylor, Michael. \emph{Community, Anarchy, and Liberty}. Cambridge: CUP 1982.

------. \emph{The Possibility of Cooperation}. Cambridge: CUP 1987.\smallskip\\
Modelagem de teoria dos jogos é cuidadosamente aplicada a problemas relacionados à justificação e à operação de uma sociedade não coercitiva.

Tolstoy, Leo. \emph{Government is Violence: Essays on Anarchism and Pacifism}. New Haven, CT: Phoenix 1990.\smallskip\\
Reflexões sobre a violência e o estado, por um ardente adversário de ambos.

Tucker, Benjamin. \emph{Instead of a Book, by a Man Too Busy to Write One}. New York: Tucker 1897. Fair-Use.org n.d. <\url{http://fair-use.org/benjamintucker/instead-of-a-book/}>. July 2, 2010.\smallskip\\
Uma coletânea de vívidos artigos polêmicos pelo decano dos anarquistas individualistas americanos do século XIX. Inclui a afirmação programática ainda amplamente discutida, ``State Socialism and Anarchism''.

Ward, Colin. \emph{Anarchism: A Very Short Introduction}. New York: OUP 2004.

------. \emph{Anarchy in Action}. London: Freedom 1982.\smallskip\\
História, teoria e prática anarquista da perspectiva do decano do anarquismo inglês pós-Segunda Guerra -- bem-humorado, humano e sempre completamente prático.

Ward, Colin, e Goodway, David. \emph{Talking Anarchy}. Nottingham: Five Leaves 2004.\smallskip\\
Uma extensa conversa entre o ícone anarquista Ward e o historiador Goodway que se centra em uma ampla gama de ativistas, ideias e prospectos anarquistas.

Wilbur, Shawn P. \emph{The Libertarian Labyrinth: Mutualist Anarchism and Its Context}. N.p. n.d. <\url{http://libertarian-labyrinth.org/}>

------. \emph{Two-Gun Mutualism and the Golden Rule}. N.p. 2010. <\url{http://libertarianlabyrinth.blogspot.com}>. July 2, 2010.\smallskip\\
Wilbur oferece uma teoria social e um comentário anarquista bem informado e perspicaz, além de uma ampla gama de textos históricos -- os obscuros assim como os relativamente bem conhecidos -- por anarquistas americanos e europeus (estes últimos às vezes disponíveis traduzidos pela primeira vez).
\end{hangparas}