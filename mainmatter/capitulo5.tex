% Quinto capítulo

\chapter{O Estado e a Liberdade Individual}
\label{chap:5}

\lettrine[lines=2]{\textcolor{LettrineColor}{\textbf{E}}}{u sou anarquista porque} o estado reprime a liberdade individual e facilita que outros façam o mesmo. Uma maneira especialmente importante pela qual o estado faz isso é através do funcionamento do direito penal [\emph{criminal law}]. Os abusos associados à existência do direito penal já são ruins o suficiente -- mas os policiais frequentemente ultrapassam as restrições já liberais ao seu comportamento para se envolver em terríveis atos de violência. O estado ataca a liberdade por meio do direito penal de todas as formas; levo em consideração alguns exemplos aqui, incluindo o abuso policial e o próprio direito penal, a destrutiva Guerra às Drogas, o abuso estatal de crianças e os ataques a relações sexuais consensuais.

\section{O delito do direito penal}

O direito penal é a mais importante instituição por meio da qual o estado exerce poder arbitrário sobre as pessoas, por conta da desconexão entre os danos factuais a pessoas reais, por um lado, e, por outro, as definições de delitos do direito penal e as sanções que ele impõe. Toda a noção de \emph{crime} é estatista. Um crime é algo que superficialmente parece ser um delito contra outra pessoa, mas ``na verdade'' é contra o rei -- exceto quando não há um rei; então, o estado entra no lugar do rei.

\subsection*{Justiça interpessoal sem o direito penal}

Em muitas sociedades mais antigas não havia realmente tal coisa como direito penal, entendido como o que determina os delitos contra o rei (ou equivalente). Delitos eram delitos contra outras pessoas, não contra o rei ou alguma abstração como o estado. Se você ferisse outra pessoa, você compensava ela ou sua família. Esse tipo de sistema era prático de diversas maneiras: ele não apresentava um sistema penitenciário ou judiciário caro (os custos do sistema jurídico teriam sido arcados pelos próprios litigantes e, presumivelmente, especialmente por aqueles que \emph{instauravam} processos contra terceiros); e focava em um objetivo razoável e claramente identificável -- compensar as pessoas pelos danos que elas verdadeiramente sofreram.

\subsection*{Descaso com as vítimas}

O crescimento do estado levou ao surgimento moderno e expansão drástica do direito penal. Essa expansão do poder do estado é profundamente preocupante.

Essencialmente, o direito penal do estado não trata de vítimas. A vítima proporciona, por assim dizer, a ocasião para o estado agir. Mas é o estado que está agindo e são os interesses do estado que o sistema estatal da justiça criminal é designado a garantir.

\subsection*{Falta de preocupação com a ocorrência de dano real}

Como o sistema está focado no estado em vez da vítima, ele corta a ligação direta entre (\emph{a}) estar sujeito a uma pena legal e (\emph{b}) realmente fazer algo prejudicial. Em um sistema baseado na vítima, no qual as pessoas reivindicam \emph{tort} umas contra as outras, uma pessoa real que alega ter sofrido um prejuízo real (ou alguém representando tal pessoa) deve demonstrar que você realmente a prejudicou para ter direito a indenização. Em um sistema baseado no estado que conta com processos iniciados pelo estado por \emph{crimes}, o estado não precisa demonstrar que você prejudicou alguém de uma maneira independentemente especificável para sujeitá-lo a penas potencialmente severas. Na realidade, o estado \emph{define} uma ação como prejudicando a si mesmo e processa as pessoas com base nisso. O estado \emph{determina antecipadamente} que determinado tipo de conduta será considerada um delito; e uma ação do tipo em questão \emph{ainda} será considerada um delito mesmo que ninguém demonstre, ou mesmo que \emph{pudesse} demonstrar, que sofreu um real prejuízo devido ao ato. Esse não é somente um ponto teórico: pense na ampla gama de transações que os estados criminalizam sem demonstrar em nenhum caso específico que elas causam algum dano notável aos participantes.

\subsection*{Falta de necessidade de vítimas reais}

Em um sistema baseado em \emph{torts}, as pessoas podem determinar o quão sério são os prejuízos e decidir se desejam ir atrás de indenização por eles. E a participação deliberada, livre e informada de alguém em uma atividade potencialmente prejudicial limitará a sua capacidade de obter indenização por prejuízos sofridos ligados a essa atividade. Esses fatores certamente aparecem em alguns processos criminais estatais: quando há uma vítima identificável, a vítima pode se recusar a prestar queixa e alguém acusado de um crime pode apresentar evidências sobre o comportamento da suposta vítima para justificar o seu comportamento ou limitar a sua culpabilidade. Mas, em outros casos, o estado não identifica as vítimas, e ninguém realmente \emph{envolvido} em algum suposto delito tem a opção de retirar as acusações -- casos envolvendo drogas são exemplos óbvios.

\subsection*{Falta de preocupação com a extensão do dano real}

Não apenas um sistema baseado em crimes não limita processos judiciais a casos em que pessoas reais podem demonstrar que elas ou seus entes queridos foram realmente prejudicados -- ele não permite nenhum balanceamento entre as consequências imposta pelo sistema jurídico para uma ação e a real quantidade de dano causado pela ação. Sem dúvidas, normalmente supõe-se que uma punição imposta pelo direito penal moderno reflita o nível de culpabilidade moral da pessoa condenada e a suposta gravidade do crime pelo qual ela foi condenada. Mas o sistema de justiça criminal \emph{assume} que o suposto delito é realmente prejudicial. Já que ninguém em particular precisa mostrar que realmente foi prejudicado pela conduta do agressor, muitas vezes não há momento no processo em que alguma suposta vítima tenha que mostrar -- ou esteja em posição de mostrar -- o quanto ela realmente foi prejudicada.

Punir uma conduta porque ela viola a lei, e não porque ela é comprovadamente prejudicial em qualquer caso em particular, faz com que seja fácil para o estado impor penas para comportamentos que alguma outra pessoa -- alguém que não é diretamente afetado -- não gosta. O direito penal fornece outro contexto onde o estado pode subsidiar. Comunidades decentes em sociedades sem estado sem dúvidas não teriam muito tempo a perder com moralistas. Mas, em uma comunidade na qual pessoas de fato queiram assediar outras pessoas cujos estilos de vida não aprovam, os pretensos assediadores teriam que \emph{eles mesmos} arcar com o custo do assédio. Em contrapartida, ser um moralista é fácil se o estado estiver do seu lado. Você pode saciar a sua vontade de ver outras pessoas serem assediadas em virtude das suas práticas religiosas, seus hábitos sexuais, as substâncias que elas consomem ou qualquer outra coisa que você ache desagradável, simplesmente persuadindo o estado a assediar por você. Você pode votar ou fazer \emph{lobby} em apoio a medidas que o estado paga ao cobrar de todos que pagam impostos. Você não precisa se preocupar com o custo de assediar os outros se esses custos forem involuntariamente compartilhados por todas as pessoas que o estado pode cobrar tributos.

\section{O injustificado sistema de justiça criminal estatal}

O estado não justifica a sua autoridade do direito penal afirmando que ele representa as vítimas. E, é claro, ele também não argumenta que precisa dessa autoridade para que possa exercer o total controle, se necessário, sobre seus sujeitos. As justificativas frequentemente oferecidas a favor do direito penal parecem muitas vezes racionalizações \emph{ex post facto} de práticas que o estado pretende se envolver independentemente delas serem justificadas ou não. Práticas herdadas da época em que o direito penal inequivocamente se tratava de delitos contra o rei continuaram a existir muito tempo depois do fim da monarquia absolutista e do descrédito da noção de direito divino dos reis. Mas o estado difunde justificativas a favor dessas práticas.

\subsection*{A retribuição se baseia em um erro}

As duas principais justificativas são que o sistema de justiça criminal efetua a \emph{retribuição} e que ele promove a \emph{dissuasão}. A noção de retribuição é conveniente porque ela \emph{pressupõe} o erro  da conduta proibida pelo direito penal; não há, como eu disse, nenhuma necessidade do estado realmente \emph{demonstrar} que a conduta prejudicou alguém. A retribuição supostamente justifica o sistema de justiça criminal do estado porque as pessoas que fizeram coisas ruins supostamente \emph{merecem} ser punidas. Mas não existe maneira não circular de dar sentido a essa ideia. A verdade é que a punição retributiva não beneficia as vítimas; causar dano a uma pessoa por si só não \emph{constitui}, de forma alguma, um benefício genuíno para outra pessoa. Não importa o quanto você tenha me feito mal, não estou objetivamente em melhor condição porque você foi prejudicado, seja por mim ou pelo estado. A ideia de dissuasão se baseia em um erro. Ela também é preocupante porque parece tirar proveito de uma hostilidade primitiva e um desejo de vingança.

\subsection*{A dissuasão transforma as pessoas em objetos}

A dissuasão também não fornece uma justificativa muito boa para o sistema de justiça criminal. Punir alguém por fazer algo é justificável com base na dissuasão se o ato de puni-la tornar os outros menos propensos a fazer a mesma coisa. A ideia por trás da dissuasão é, de fato, fazer a pessoa punida de exemplo.

Sem dúvidas, é bom que as pessoas sejam desencorajadas a se envolver em condutas que de fato causam danos aos outros. Mas a dissuasão trata as pessoas como coisas, como objetos a serem manipulados, como meios para fins de terceiros em vez de fins em si mesmos. Isso viola o que eu considero como, no mínimo, um princípio moral fundamental, de que nunca é certo prejudicar alguém proposital ou instrumentalmente. (Isso não exclui causar dano a alguém durante o ato de defender a si mesmo ou a outra pessoa; mas nesse caso causar dano ao agressor não é o intuito -- é apenas um subproduto da sua ação defensiva.) Pois o principal intuito da dissuasão é causar dano ao suposto infrator \emph{com o propósito} de desencorajar outras pessoas de replicar o seu comportamento.

A dissuasão também parece moralmente problemática por outras razões. Por exemplo, se dissuadir sérios danos fosse realmente uma justificativa independente para o uso da força, poderia ser aceitável impor penalidades horríveis para danos pequenos se fazê-lo parecesse impedir a sua repetição. De maneira semelhante, pode ser aceitável incriminar e até executar pessoas que sabe-se serem inocentes para evitar danos futuros. Se acreditarmos que fazer esse tipo de coisas não é razoável, temos boas razões para rejeitar a dissuasão como uma fonte independente de justificação para o uso da força, uma vez que, se fosse essa uma fonte de justificação, esses tipos de escolhas seriam aceitáveis.

\subsection*{A reabilitação dá ao estado um poder assustador}

O exercício de poder do estado sobre as pessoas por meio do sistema de justiça criminal também é muitas vezes justificado como um meio de reabilitá-las. Novamente, essa é uma justificativa \emph{ex post facto}: o encarceramento já existia muito tempo antes de agentes estatais levarem a sério a ideia de reabilitação. Mas se as pessoas serão presas de qualquer maneira, sem dúvidas é uma boa ideia (da perspectiva do estado) ter uma razão para encarcerá-las que soe moderna.

A ideia da reabilitação é, em princípio, certamente boa: seria bom que pessoas que não se importam com os interesses dos outros mudassem suas atitudes e comportamentos de maneiras positivas. Mas atribuir ao estado a responsabilidade pela reabilitação das pessoas é muito problemático. Primeiro, dá ao estado a autoridade para decidir quais tipos de traços de caráter precisam ser eliminadas ou encorajadas. Assim, coloca o estado em uma área de debate altamente contenciosa para qualquer sociedade e espera que ele exerça um nível de competência que ele não é de fato capaz. Segundo, dá ao estado um enorme poder sobre os indivíduos -- não apenas poder para regular a sua conduta, o que já é ruim o suficiente, mas poder para regular o seu caráter e personalidade. Terceiro, dá ao estado esse poder sem nenhum limite claramente definido. O que é um nível satisfatório de reabilitação? Quanto tempo leva para que se efetue o tipo certo de reabilitação? Quem pode dizer se alguém está devidamente reabilitado? Não existem padrões objetivos que todos possam concordar, e a pessoa a ser reabilitada pode ser indefinidamente confiada ao misericordioso cuidado do estado.

A profunda inadequação do sistema de justiça criminal existente levou algumas pessoas a argumentar que o necessário é um sistema de justiça que é \emph{restaurativo}. Sistemas de justiça restaurativa focam não apenas na restituição, mas também na reconciliação entre o infrator e a vítima e na reintegração dos infratores em suas comunidades. Os sistemas de justiça restaurativa são extremamente promissores como alternativas às abordagens baseadas na retribuição e dissuasão. Suspeito que mecanismos restaurativos estariam disponíveis junto de sistemas que garantiriam restituição em muitas comunidades de uma sociedade sem estado. Mas há algo indiscutivelmente problemático em usar a autoridade coercitiva do estado para promover a restauração: a reconciliação obrigada pelo estado provavelmente seria desconfortável e talvez criaria significativos riscos físicos e emocionais para as vítimas e infratores. A justiça restaurativa faz sentido, mas não como uma justificativa para o sistema de justiça criminal do estado.

\subsection*{Algumas pessoas precisam ser contidas}

Obviamente, algumas condutas realmente causam danos a outras pessoas. Algumas vezes, aqueles que se envolvem em tais condutas representam ameaças reais e contínuas a outras pessoas. E o sistema jurídico precisa nos proteger dessas ameaças. Ele precisa conter pessoas regularmente violentas.

Isso não significa, é claro, que vale tudo. Nunca é razoável causar mais danos do que o necessário. E, se esse é o caso, então não há justificativa, por exemplo, para usar força letal contra uma pessoa que representa ameaça genuína para os outros mas que pode ser contida segura e economicamente. Mas alguns tipos de restrições parecem fazer sentido para alguns tipos de pessoas. Isso não fornece, no entanto, justificativa alguma para o sistema de justiça criminal do estado, ou, de fato, para qualquer sistema de justiça significativamente diferente do sistema de justiça civil fornecido pelo direito contratual e de \emph{tort}. Uma série de medidas cautelares e outros recursos podem conter pessoas persistentemente violentas. É possível evitar que essas pessoas causem danos a outras sem o sistema de justiça criminal e, de fato, sem toda a ideia de \emph{crime} -- entendido como algo diferente de dano real causado a uma pessoa real que efetivamente vê aquilo \emph{como} danoso.

\section{\emph{Status} e privilégios aumentam o risco da violência policial}

Uma crescente variedade de condutas agora está sob o domínio do direito penal do estado. Isso significa que existem mais leis criminais a serem aplicadas e mais pessoas com a chance de violá-las. E isso significa, por sua vez, que o estado precisa de mais aplicadores da lei. Então não é de surpreender que as forças policiais tenham crescido durante o século passado, que seus orçamentos e mandatos tenham se expandido, ou que tenham adquirido ferramentas e brinquedos mais rápidos, mais poderosos e mais assustadores.

Já que ele pode impor penas a quase qualquer conduta, o direito penal é uma fonte de poder estatal quase ilimitado. Então forças policiais podem ser perigosas simplesmente porque são encarregadas de aplicar o direito penal. Mas também são perigosas porque, como aplicadoras da lei, elas são conferidas com a autoridade operacional para fazer uso da violência monopolizada pelo estado com o intuito de manter qualquer tipo de ordem que os mestres do estado considerem desejável. Obviamente, as forças policiais contêm muitas pessoas conscientes e bem-intencionadas. Mas o papel que os policiais são exigidos a cumprir faz com que seja difícil que eles sejam algo além de defensores do \emph{status quo}, e as ferramentas, oportunidades e privilégios que eles recebem torna mais fácil que alguns, infelizmente, façam uso agressivo da força.

As culturas internas de muitos dos departamentos de polícia recompensam policiais por ignorarem as restrições a seus comportamentos e por demonstrarem sua masculinidade por meio da violência. Por exemplo: foi recentemente alegado que policiais de Londres torturaram prisioneiros por afogamento -- para obter informações em um caso de maconha (note: a tortura é frequentemente justificada com referência a terríveis casos de atentado com bomba-relógio, mas, uma vez justificada, aparentemente se torna rotineira).\footnoteA{Veja Charles Johnson, ``The Police Beat,'' \emph{Rad Geek People's Daily: Official State Media for a Secessionist Republic of One} (n.p., June 11, 2009) <\url{http://radgeek.com/gt/2009/06/11/the-police-beat-2}> (July 1, 2010).} Policiais na Virgínia prenderam uma mulher por escrever sobre as atividades deles em um blog, citando um estatuto que tornava crime (!) identificar publicamente um policial para fins de intimidação.\footnoteA{``\,`Uh-Oh They're Here': A Persistent Blogger Annoys Police -- and Winds up in Jail,'' \emph{Washington Post} (\emph{The Washington Post} Co., Aug. 10, 2009) <\url{http://www.washingtonpost.com/wp-dyn/content/article/2009/08/09/AR2009080902126.html}> (July 1, 2010).}

Policiais no Texas invadiram uma casa, armados e sem anunciar, com base em uma denúncia infundada e suspeita de que a casa continha maconha. Quando o proprietário tentou se defender de \emph{hooligans} que ele não sabia que eram policiais, os policiais atiraram nele. E a busca deles não encontrou nenhuma droga. Um policial de Nova York espancou e algemou uma mulher bêbada de meia-idade que tropeçou nele.

Em outro lugar em Nova York, policiais espancaram duas lésbicas que não eram suspeitas de crime algum -- simplesmente porque elas apareceram na cena de um incidente enquanto os policiais estavam dispersando os espectadores --, tudo isso enquanto as insultavam com apelidos. Policiais da Pensilvânia espancaram e jogaram spray de pimenta em um menino de 13 anos porque ele se distanciou deles enquanto tentava explicar ao pai no telefone que sua mãe havia se envolvido em um acidente de carro. Um policial do Arizona prendeu um homem por atravessar fora da faixa e urinar em público antes de algemá-lo, batendo sua cabeça contra o capô de um carro e o jogando contra uma cerca de arame farpado. Cerca de 24 policiais perseguiram e espancaram um homem que era supostamente culpado da chocante ofensa de ``andar de \emph{minibike} sem capacete e sem licença''. Até o momento em que este chocante incidente chegou ao fim, 14 pessoas tinham sido presas por crimes como tentar descobrir o que estava acontecendo e se recusar a permitir que os policiais entrassem em uma casa sem um mandado. Quando as pessoas começaram a filmar o abuso policial usando celulares, elas foram espancadas e seus celulares foram roubados pela polícia.

Essas recentes histórias de violência fora de controle não são histórias sobre ``maçãs podres''. É dessa maneira que os defensores do estado e suas forças policiais gostam de conceber as coisas. Mas os problemas fundamentais são sistêmicos. Eles resultam de dar aos policiais um poder relativamente sem restrições para usar a força e da cultura de violência que permeia muitos departamentos de polícia.

Suponha que você esteja dirigindo por aí e observe uma viatura policial pelo seu retrovisor. Suponha que ela fique atrás de você enquanto dá várias voltas. Você tenderia a se sentir aliviado por se beneficiar da proteção especial dos heróis responsáveis por nos manter seguros? Ou você sente um aperto no estômago enquanto ansiosamente pensa -- ao mesmo tempo que tenta não chamar atenção para si mesmo -- em uma maneira de fugir? Quando você lê uma notícia sobre um policial que reage com raiva ou até mesmo violentamente ao ser filmado, você se surpreende? Quando o comentarista na TV, defendendo a necessidade de maiores poderes policiais, garante que \emph{você} não terá de se preocupar quando câmeras forem instaladas em todos os cantos, você se pergunta com o que os policiais pensam que \emph{eles} devem se preocupar?

Talvez eles saibam em algum nível que a desunião entre os policiais e as pessoas a quem eles deveriam servir continua a crescer. Como destaca a recente e polêmica prisão do professor de Harvard Henry Louis ``Skip'' Gates, muitos policiais, mesmo os benevolentes e responsáveis, parecem ter um senso de permissão, uma percepção de que é aceitável para eles usar a força quando não seria aceitável para qualquer outra pessoa fazer o mesmo. Muitos policiais parecem acreditar que eles têm todo o direito de assediar, prender ou espancar pessoas que não consideram suficientemente respeitosas. E provavelmente não é muito surpreendente que eles estejam inclinados a reagir dessa forma: eles são funcionários do estado, e a existência e autoridade do estado têm como premissa a noção de que algumas pessoas são mais iguais do que outras, que algumas pessoas têm direito a privilégios que outras não têm -- e que não há problemas que elas façam uso da força para manter esses privilégios.

Em todo o Estados Unidos, o governo federal está cada vez mais fazendo uso do financiamento e de outras relações que ele construiu nas últimas décadas para militarizar as funções comuns da patrulha policial -- especialmente ao fornecer armas militares de ponta aos policiais e, implicitamente, ao encorajá-los a usar essas armas. Parece que os policiais cujo comportamento violento já deveria estar nos alarmando estão agora sendo preparados para desempenhar papéis cada vez mais importantes na manutenção do poder do estado frente à inquietação civil que os funcionários estatais estão claramente antecipando.

Em uma sociedade sem estado, as pessoas obviamente poderiam precisar se defender contra a violência. E voluntários do bairro ou profissionais de segurança poderiam ajudar. Mas nenhuma pessoa e nenhuma associação de proteção seria imune à responsabilidade de causar dano indevidamente a alguém, mesmo que no curso de lidar com a violência injusta. Ninguém teria um passe livre de agente do estado.

Muitos dos incidentes que acabei de descrever ocorreram quando policiais foram chamados para ajudar as pessoas ou resolver disputas. Em um mundo dominado pelo estado, os serviços de segurança são quase monopolistas. As pessoas muitas vezes se metem em apuros ao se defender, e isso significa que muitas vezes elas não são livres para pedir que outras pessoas que não sejam policiais do governo as defendam. É improvável que outras pessoas além de policiais queiram se envolver em situações potencialmente violentas, não apenas porque elas podem não se sentir confortáveis lidando elas mesmas com a violência, mas também porque elas mesmas podem ser alvos da violência \emph{policial} se os policiais chegarem e as considerarem como parte do problema, em vez da solução. Quando você liga para policiais contratados pelo estado para resolver um problema, muitas vezes você na prática não tem escolha a não ser depender de um grupo fixo de policiais de uma única agência policial. 

Em uma sociedade sem estado, é claro, nenhum grupo voluntário ou associação profissional seria privilegiado dessa maneira. E isso significa que os grupos que respondessem a situações com força excessiva, que se comportassem de maneiras fora de controle, provavelmente não seriam chamados uma segunda vez. As pessoas teriam \emph{escolhas}; e as pessoas provavelmente não gostariam de pagar repetidamente pelos serviços de profissionais de segurança que se envolvessem em violência e abuso.

Políticas de remuneração que afetam os policiais também podem contribuir para a má conduta deles. Policiais podem receber a mesma quantia, quer eles estejam investigando assassinatos violentos, quer estejam atacando pessoas que escolhem não usar capacete enquanto andam de \emph{minibike}. É fácil imaginar que, em um sistema baseado em \emph{torts}, a remuneração de um profissional de segurança poderia vir da indenização por danos paga por ou em nome de alguém responsável por um prejuízo cuja indenização foi assegurada pelo profissional. Mas quando, como no caso do homem na \emph{minibike}, ninguém sofreu prejuízo, não haveria indenização por danos e, então, não haveria montante de dinheiro para remunerar os profissionais de segurança que o prenderam. Não haveria, assim, incentivo para que profissionais de segurança fossem atrás dele -- muito menos para espancar ele e os vários espectadores.

Sem a imunização estatal dos policiais contra a responsabilidade por seus abusos, sem o monopólio estatal dos serviços de segurança e sem a criação de incentivos pelo direito penal para que policiais percam tempo lidando com ações que não causam danos indenizatórios, pessoas inclinadas a agir de forma abusiva e violenta seriam rapidamente eliminadas e grupos de pessoas responsáveis pela segurança dos outros seriam muito mais cuidadosos ao escolher a quem conceder o uso da força e como a força deveria ser usada. Em contrapartida, para o estado é fácil proteger aqueles que mantêm o seu poder até mesmo da responsabilidade por atos terríveis de violência e impor sanções reduzidas a muitos daqueles que são responsabilizados por abusos. Eu suspeito que isso ocorra tanto porque as autoridades estatais julgam necessário que aqueles que usam da violência em seu nome tenham uma relativa liberdade para atuar de maneira eficaz, quanto porque algumas autoridades estatais provavelmente aprovam o medo que a tolerância a esse tipo de comportamento gera nas populações sujeitas. Acabar com a violência policial significa acabar com os privilégios especiais que os policiais desfrutam como agentes do estado.

E aqui voltamos novamente ao estado como a raiz do problema. Porque as agências policiais, as agências que treinam, equipam e dirigem os policiais, são agências do estado. E isso significa que elas estão protegidas pela doutrina jurídica da ``imunidade soberana''. A persistência dessa doutrina fornece mais evidências de que o estado moderno e democrático ocupa, de alguma forma, o lugar do rei. É uma doutrina que começou com a suposição de que a lei era do rei e que o rei portanto não poderia fazer algo de errado. Se o rei não podia errar, é claro, então com que fundamento alguém poderia alegar que o rei era responsável por algum tipo de dano indenizatório? Não há mais tantos reis por aí, e na maioria das sociedades em que há reis, os reis são figuras representativas, e não governantes absolutos fora do alcance da lei. Mas agora os estados e suas subdivisões são tratados exatamente como o rei.

Os governos às vezes graciosamente permitem que as pessoas os processem. Mas há limites, especialmente limites para indenizações em dinheiro. E a lei muitas vezes faz com que seja difícil mostrar que um policial ou outro funcionário do estado agiu de maneira inaceitável ou que a agência que supervisiona o funcionário deveria ter sido mais cuidadosa. Assim, os governos e suas agências estão protegidos das reais responsabilidades jurídicas na maior parte do tempo.

Para piorar as coisas, quando eles têm de pagar indenização, eles podem cobrar do público, já que reivindicações de danos são normalmente cobertas por políticas de seguro que são pagas com fundos extraídos de pessoas comuns sob a mira de armas. Não parece muito justo pedir que aqueles que já pagam impostos ao estado paguem a conta dos funcionários estatais que se envolvem em condutas abusivas. Em uma sociedade sem estado, indivíduos e grupos que causaram danos a terceiros enquanto prestavam serviços de proteção não poderiam desviar a responsabilidade para outras pessoas. \emph{Eles} teriam que arcar com os custos dos abusos em que se envolveram. Não é difícil imaginar que eles se comportariam muito melhor como resultado.

\section{A volta da proibição}

Um exemplo particularmente claro do tipo de abuso que a existência do direito penal torna possível é a Guerra às Drogas.

Se é que eles se lembram, a maioria dos americanos se recorda da Lei Seca como uma espécie de piada, com álcool amplamente disponível para compra à base de ``piscadelas e cutucadas''.\footnoteNT{N.T.: \emph{``wink-wink nudge-nudge''}. Expressão que se refere às piscadelas e cutucadas que alguém dá em outra pessoa quando quer insinuar alguma coisa sem dizê-la explicitamente. Nesse caso, insinuar o desejo de comprar álcool em uma época de proibição.} Certamente, americanos respeitáveis não a levaram muito a sério: até Earl Warren, uma homem certinho (como procurador-geral do estado, ele liderou tentativas para acabar com os jogos de azar na costa da Califórnia, por exemplo), chegava em casa do seu trabalho de promotor público do condado de Alameda na época da Lei Seca para desfrutar de um copo de uísque. Mas, apesar disso, ela deu ao estado a oportunidade de gastar uma enorme quantia de dinheiro e ameaçar e prender pessoas não violentas. Ela também estimulou a violência do crime organizado e aumentou a taxa nacional de homicídios. Não é surpreendente que a maioria dos americanos ficaram felizes com o fim da Lei Seca.

\subsection*{O custo da Guerra às Drogas}

Mas essa mesma mentalidade está muito em alta nos dias de hoje, com políticos gastando quantias inacreditáveis de dinheiro atacando o consumo de outras substâncias que alguns de seus constituintes desaprovam e mandando um grande número de pessoas para a prisão, frequentemente por longos períodos. O governo dos EUA gastou \$15 bilhões de dólares com a Guerra às Drogas em 2010. Na época em que escrevi este parágrafo, em janeiro de 2011, agências nacionais, estaduais e locais nos Estados Unidos \emph{já} haviam gasto quase \$6 bilhões de dólares na fracassada guerra às drogas no ano novo, e mais de 125.000 pessoas já haviam sido presas por delitos relacionados a drogas. Dessas pessoas, 64.519 foram presas por violar leis relacionadas a maconha -- a grande maioria simplesmente por possuir produtos de cannabis. Nos Estados Unidos, uma pessoa é presa por violar leis de drogas a cada dezenove segundos.\footnoteA{``Drug War Clock,'' \emph{DrugSense} (n.p., July 1, 2010) <\url{http://www.drugsense.org/wodclock.htm}> (Jan. 27, 2011).}

Em 2005, mais de um quinto das pessoas nas prisões estaduais e mais da metade das pessoas nas prisões federais foram encarceradas por delitos relacionados a drogas.\footnoteA{Algumas delas podem ter se envolvido em violência associada a transações de drogas, em vez de meramente participar pacificamente dessas transações. Por diversas razões, como sugiro abaixo, a frequência da violência relacionada às drogas aumenta drasticamente com a criminalização.} Mais de 25\% dos presos negros e latinos estão cumprindo pena por atividades relacionadas a drogas.\footnoteA{``Drug War Prisoner Count Over Half a Million, US Prison Population at AllTime High,'' \emph{Drug War Chronicle} (Drug Reform Coordination Network, Oct. 28, 2005) <\url{http://stopthedrugwar.org/chronicle-old/409/toohigh.shtml}> (July 1, 2010).}

\subsection*{A arbitrariedade da Guerra às Drogas}

Certamente há espaço para divergências sobre até que ponto diversas substâncias são prejudiciais. Está claro, no entanto, que a proibição de substâncias prejudiciais é qualquer coisa menos uniforme e consistente -- consumir produtos com alto teor de gordura parece matar muitos americanos, mas há pouca pressão a favor de uma violenta Guerra aos Alimentos Gordurosos levando à prisão das pessoas que compram e vendem \emph{milkshakes} e batatas fritas. Está igualmente claro que, quaisquer que sejam as soluções adequadas para qualquer um dos danos associados à venda e consumo de drogas atualmente ilegais, as penas de prisão, que destroem a vida das pessoas e mancham suas fichas, são respostas abusivamente injustas ao comportamento não violento.

\subsection*{A Guerra às Drogas como uma criação do estado}

Também está claro que nada disso aconteceria se não houvesse estado. Sem o estado, na grande maioria dos casos não haveria ninguém interessado em entrar com um processo por danos causados como resultado de uma transação voluntária de drogas. O estado, em contrapartida, não precisa se preocupar em demostrar os danos a fim de usar seu poder ilegítimo para prender e encarcerar pessoas, e pode financiar a sua crescente guerra às drogas extraindo os fundos necessários de contribuintes indispostos a fazer isso.

A posse e venda de drogas não seriam criminalizadas sem a ação do estado (nem seriam, é claro, quaisquer outras coisas). Provavelmente também haveria muito menos atos de violência associados às transações de drogas se tais transações não fossem criminalizadas. Como suas disputas são relacionadas a transações ilegais, as pessoas não têm acesso ao sistema jurídico para resolver essas disputas. E o fato de suas transações já serem ilegais tende a torná-las mais dispostas a recorrer à violência -- para quem já está molhado um pingo é besteira.\footnoteNT{N.T.: \emph{one might as well be hanged for a sheep as for a lamb}. Expressão idiomática que nesse caso faz referência ao fato de que, sabendo que já poderia ser punido por vender drogas, recorrer à violência não faria muita diferença.}

Além disso, o caráter ilegal das transações de drogas significa que essas transações são conduzidas grande parte em segredo. Por elas serem assim, aqueles que participam nelas são mais propensos a se envolver em fraude, roubo e violência do que seriam caso as vendas de drogas ocorressem abertamente, já que o escrutínio público tende a reduzir o comportamento agressivo, até mesmo além do medo de sanções legais. Além disso, como o estado torna a venda de drogas ilegal, as transações de drogas são muito mais caras do que seriam de outra forma. Um resultado do alto custo do comércio de drogas é que há menos vendedores de drogas do que haveria sem o envolvimento do estado. Os vendedores podem, assim, cobrar preços muito altos e ganhar muito mais pelos produtos que eles vendem do que se houvesse mais vendedores envolvidos. Como grandes quantias de dinheiro estão, portanto, envolvidas em muitas transações de drogas, os vendedores podem estar mais propensos a usar da violência para proteger seus bens ou roubar de terceiros do que estariam caso os ganhos potenciais fossem menores. Por fim, a natureza custosa dos preços das drogas faz com que seja mais difícil que aqueles que desejam comprar drogas o façam e, portanto, faz com que seja mais provável que eles não vejam outra maneira de conseguir tais drogas além da fraude ou do roubo.

Há pelo menos mais uma maneira perturbadora pela qual a guerra às drogas é uma criação do estado: mesmo enquanto ataca a venda privada e o consumo de substâncias como cocaína e heroína, o estado incentiva o negócio das drogas. Por exemplo, como o jornalista Gary Webb mostrou com detalhes exaustivos, a \emph{Central Intelligence Agency} [CIA] do governo dos EUA facilitou o transporte de drogas para Los Angeles e sua subsequente venda, a fim de facilitar que os \emph{contras} na Nicarágua obtivessem financiamento (já que o financiamento direto pelo governo dos EUA havia sido proibido pelo Congresso). De maneira semelhante, alguns observadores afirmaram que a CIA auxiliou chefes afegãos do tráfico e facilitou o transporte de ópio do Afeganistão em seu nome em troca da sua ajuda no combate contra a ocupação russa de seu país. Afirmações semelhantes foram feitas em relação ao apoio da CIA aos chefes do tráfico -- aliados supostamente valiosos -- no Afeganistão nos dias de hoje. O uso contínuo do poder estatal contra vendedores e consumidores de drogas é particularmente preocupante, visto que os agentes estatais podem estar deliberadamente envolvidos no patrocínio do negócio das drogas.

\subsection*{Vítimas inocentes}

Os custos humanos impostos pela guerra às drogas sobre as pessoas que compram e vendem drogas são enormes. Mesmo as pessoas que não estão envolvidas no tráfico de drogas podem facilmente se tornar vítimas. As autoridades oferecem incentivos para que as pessoas dedurem as outras por serem fornecedoras de drogas. Mesmo quando as evidências são escassas ou aparentemente inexistentes, alguém pode ser condenado e sentenciado à prisão com base no testemunho de um informante. Leis de confisco civil permitem que agências governamentais apreendam e vendam bens supostamente usados ou adquiridos com lucros de um esquema criminoso. O pior é que elas podem fazer isso sem condenações penais -- ou mesmo acusações criminais. Quando os bens de uma pessoa são apreendidos por uma lei de confisco, a lei coloca sobre \emph{ela} [a pessoa] o ônus de convencer o estado a devolvê-los ao provar que eles não foram usados para cometer crimes. Assim, uma agência de aplicação da lei tem um forte incentivo para alegar que alguém vendeu ou planejou vender drogas; e se isso acontecer, a lei de confisco pode autorizar que a agência tome seus bens para uso próprio ou para venda.

Além disso, a desconsideração pela liberdade e dignidade pessoais refletida no empreendimento da própria guerra às drogas, a noção de que policiais podem fazer tudo o que precisarem fazer no decorrer de suas tarefas atribuídas pelo estado e a cultura de violência que existe em muitas agências policiais, tudo isso faz com que não seja surpreendente que pessoas que nem mesmo estejam envolvidas na compra ou venda de drogas sejam feridas enquanto o estado tenta aplicar as suas leis de drogas.

Por exemplo: membros de uma equipe da SWAT aparentemente não perceberam que a casa que eles estavam invadindo não era a casa mencionada no mandado que supostamente autorizava as suas ações; isso não os impediu de atirar e matar John Adams, de 64 anos, que estava sentado em frente à sua televisão. O governo dos EUA forneceu informações imprecisas à força aérea do Peru que resultaram na queda de um avião que levava missionários e na morte de uma missionária batista de 35 anos e sua filha de sete meses. Um homem de 43 anos sem armas em suas mãos implorou pela sua vida ao ser baleado por homens que começaram a persegui-lo e de quem ele, compreensivelmente, tentou fugir; embora ele possa ter pensado que esses homens com roupas casuais eram bandidos comuns, eles eram -- você adivinhou corretamente -- policiais antidrogas tentando cumprir um mandado contra outra pessoa. Um homem foi a uma casa pagar uma dívida de \$20 dólares; ele nunca voltou -- porque foi baleado na cabeça por um policial durante uma operação da SWAT naquela casa que coincidiu de acontecer enquanto ele estava lá. Uma mulher de 84 anos foi acidentalmente baleada em sua cama por policiais durante uma operação antidrogas.\footnoteA{Pete Guither, ``Drug War Victims,'' \emph{Drug WarRant} (Drug WarRant, 2010) <\url{http://www.drugwarrant.com/articles/drug-war-victim/}> (July 1, 2010).}

Mesmo quando as fatalidades são menos terríveis, a Guerra às Drogas é destrutiva. Ela arruína a vida de pessoas. Ela ameaça a liberdade de todos -- nesse sentido, todos são suas vítimas. Ela não existiria sem o estado, e a sua contínua perversidade é outra razão para irmos em direção à vida sem o estado o mais rápido possível.

\section{O estado entre quatro paredes}

O fato de que a responsabilidade penal não depende que alguém demonstre a ocorrência ou extensão do dano que o réu causou faz com que seja mais fácil para o estado tentar regular o comportamento sexual das pessoas. Controlar a vida sexual de outras pessoas não custa muito aos supostos reguladores, porque eles podem forçar o público a pagar por isso por meio de impostos; o impacto das regras é sentido por um número suficientemente limitado de pessoas e o custo de aplicá-las é mínimo quando comparado com a carga tributária em geral, então é fácil que aqueles que desejam imponham tais regras.

As necessidades e desejos sexuais das pessoas variam drasticamente. Coisas que podem ser prejudiciais para algumas pessoas podem não ser para outras. E mesmo que uma pessoa de fato se prejudique ao fazer uma escolha tola, não segue que seja razoável que outra pessoa a sujeite a penas legais por isso: qualquer pessoa que deseje liberdade suficiente para ela mesma cometer erros deve estender a mesma liberdade aos outros.

Mas, por diversas razões, algumas pessoas têm ideias muito rígidas sobre como deveria ser a vida sexual das outras pessoas. Às vezes, essas ideias estão enraizadas em tradições religiosas. Às vezes, elas dependem de visões filosóficas. E, às vezes, elas refletem profundas respostas emocionais que podem carecer de uma justificativa independente. De qualquer forma, elas tendem a ser profundamente enraizadas. E aqueles que as defendem realmente parecem acreditar que o mundo irá colapsar caso as relações sexuais consensuais de outras pessoas continuem a não ser controladas.

Uma das maneiras mais estranhas pelas quais esse tipo de medo parece se manifestar é na crença de algumas pessoas de que os casamentos entre pessoas de sexo diferente serão ameaçados se os casamentos entre pessoas do mesmo sexo se tornarem amplamente aceitos. Eu nunca entendi que tipo de ameaça é essa. As pessoas têm medo que mais casais de sexo diferente se separem se o casamento entre pessoas do mesmo sexo for tratado como normal? Que os casamentos de pessoas de sexo diferente não serão tão especiais se um número ligeiramente maior de pessoas puder se casar? Que sua decisão de ter filhos no contexto do casamento não significará tanto se casais do mesmo sexo, que não podem ter filhos (assim como casais estéreis de sexo diferente, cujo direito de casar nunca é questionado), puderem se casar? (Obviamente, o estado nem sequer deveria se encarregar dos casamentos; mas se o fizer, ele não tem o mínimo direito de ser discriminatório.)

De qualquer forma, qualquer que seja o fundamento para essa aversão profunda às práticas de outras pessoas que algumas pessoas parecem ter, elas podem usar o estado para pôr em prática seus preconceitos. Virgínia, Virgínia Ocidental, Flórida, Michigan, Mississippi e Dakota do Norte todos tratam a coabitação entre pessoas solteiras como delito criminal. Alguns estados impõem responsabilidade penal pela posse de brinquedos eróticos.\footnoteA{``Sex and Politics,'' \emph{Woodhull Freedom Foundation} (Woodhull Freedom Foundation, 2009) <\url{http://www.woodhullfoundation.org/sexpolitics.htm}> (July 1, 2010).} Dois adolescentes héteros da Flórida fizeram sexo consensual e compartilharam fotos de seu encontro entre eles; como resultado, ambos foram classificados como criminosos sexuais.\footnoteA{``Court Issues Unbelievably Stupid Sex Crime Ruling,'' \emph{Alas, a Blog} (n.p., Feb. 21, 2007) <\url{http://www.amptoons.com/blog/archives/2007/02/21/court-issues-unbelievablystupid-sex-crime-ruling}> (July 1, 2010)} Há não muito tempo, dependendo de onde a transação ocorresse, você poderia ser preso por comprar anticoncepcionais nos EUA. Até muito pouco tempo atrás, vários estados dos EUA criminalizavam o sexo entre pessoas do mesmo sexo.

E o exército do governo dos EUA (o mesmo exército cujos soldados se envolveram em atos de tortura extremamente sexualizados no Iraque\footnoteA{Naomi Wolf, ``Sex Crimes in the White House,'' \emph{Huffington Post} (Huffington Post, July 7, 2008) <\url{http://www.huffingtonpost.com/naomi-wolf/sex-crimes-in-the-whiteh_b_111221.html}> (July 1, 2010).}) ainda criminalizam uma série de práticas sexuais consensuais. Membros do Senado e da Câmara dos Representantes que aprovam sucessivas versões do \emph{Uniform Code of Military Justice} [UCMJ] sem dúvida enfrentariam sérias dificuldades políticas se impusessem penas criminais a pessoas comuns pelas condutas sexuais que o UCMJ criminaliza. Mas ao impor essas penas aos militares, eles podem sinalizar o seu apoio a valores supostamente tradicionais e se deleitar em sua própria hipocrisia com mínimo custo político ou pessoal.

O estado invade a liberdade sexual das pessoas. E a existência do estado -- com a sua capacidade de criminalizar comportamentos mesmo que ninguém se considere vítima deles e mesmo que nenhuma suposta vítima esteja disposta ou seja capaz de demonstrar a ocorrência ou extensão do dano real -- torna a criminalização de relações sexuais consensuais possível. O estado força todos a arcar com o custo de satisfazer o desejo irracional de algumas pessoas de regular a vida sexual de outras. Se você valoriza a oportunidade de explorar e canalizar a sua própria sexualidade, você tem mais uma razão para se opor ao estado.

\section{O estado abusa das crianças}

O estado é sistematicamente hostil à liberdade das crianças.

Você nunca saberia disso se apenas ouvisse a retórica hipócrita dos políticos. Eles gostam de enjoativamente repetir o mantra de que: ``É tudo pelas crianças.'' Pense na frequência com que ``um futuro melhor para nossos filhos'' serve como justificativa para alguma nova idiotice política.

Mas dê uma olhada no que o estado de fato faz.

Uma das maneiras mais preocupantes com que ele deixa isso claro é ao emprestar o seu apoio forçoso a pais interessados em manter seus filhos sob seu controle. Crianças que não querem morar com os pais podem ser forçadas a fazê-lo: elas podem ser arrastadas para casa pela polícia, mantidas em casas das quais querem fugir. E a lei está perfeitamente disposta a apoiar os pais que usam da sua autoridade concedida pelo estado para insistir que seus filhos participem de sádicos programas de modificação de comportamento destinados a torná-los mais obedientes -- às custas da sua individualidade e liberdade. Ela está até disposta a apoiar os pais que sequestram seus filhos e os sujeitam a tentativas de ``desprogramação'' porque eles entraram em comunidades religiosas que seus pais não gostam. Muitas vezes, o estado trata as crianças como propriedade de seus pais.

Claro, não é apenas o seu apoio à autoridade parental irracional que muitas vezes coloca o estado como um oponente da liberdade das crianças. Pense nas regras -- como aquela que mencionei anteriormente -- que tratam os adolescentes como incapazes de controlar a sua própria sexualidade. Ou em decretos locais de toque de recolher que negam às pessoas o acesso aos próprios espaços do estado em determinados períodos de tempo -- apenas por conta das suas idades.

Se o estado se recusasse a se manifestar contra elas, as crianças teriam pelo menos uma maior chance de se desenvolver de maneira livre e criativa. Se você valoriza a liberdade delas, se você acredita que elas merecem ser respeitadas como indivíduos únicos que são, diga \emph{não} ao estado.

\section{O estado faminto}

Os ataques do estado à liberdade individual nunca terminam. Ele é insaciável.

A inércia do desenvolvimento do estado faz com que seja difícil reverter o crescimento de seu poder. Uma vez que ele tenha reivindicado a autoridade para fazer algo, é improvável que ele renuncie essa autoridade. É mais difícil derrubar uma lei ou regulamentação existente do que promulgar uma nova: para manter uma lei ou regulamentação em vigor (na ausência de uma cláusula de prazo de validade [\emph{sunset clause}]), os legisladores ou reguladores precisam apenas escolher não fazer nada. Então, o conjunto de poderes existentes do estado é quase sempre o ponto de partida para qualquer discussão sobre quais políticas devem ser adotadas. O estado pode se expandir; mas ele dificilmente diminuirá.

Então, o estado insaciável reivindica mais e mais poder.

Por exemplo: enquanto escrevo, o Congresso dos EUA continua a considerar a implementação de um sistema nacional de carteira de identidade. O primeiro mandato para tal sistema, incorporado no chamado \emph{REAL ID Act}, foi contestado por uma ampla gama de grupos sensatamente anti-estado. Em seu lugar, o governo está propondo um sistema semelhante, agora (grosseiramente) rotulado como \emph{PASS ID} (desagradavelmente lembrando os sistemas de passaportes internos de regimes autoritários) e apresentando muitas, apesar de não todas, das características duvidosas do \emph{REAL ID}.

Por anos, o governo incentivou a utilização dos números de Previdência Social [\emph{Social Security numbers}] para propósitos de identificação, permitindo que funcionários estatais rastreiem os movimentos e transações financeiras das pessoas. Durante a era Bush, paranoicos aficionados por segurança e controladores autoritários uniram forças para aumentar a capacidade do estado de monitorar e supervisionar.

Uma das conquistas mais memoráveis deles foi transformar o aeroporto americano de algo que era um pedaço do purgatório em (no mínimo) o oitavo círculo do inferno. Medidas reforçadas de segurança pioram o trânsito, submetem pessoas a revistas humilhantes, tratam passageiros de companhias aéreas como suspeitos de terrorismo e negam a amigos oportunidades de compartilhar o tédio do processo de pré-embarque com os passageiros.

Os ataques de 11 de setembro forneceram a desculpa para que as autoridades nacionais implementassem planos de repressão de segurança que obviamente estavam em planejamento por anos. Mesmo quando o governo dos EUA nos anos Clinton atacava grupos dissidentes e antigovernamentais como o Ramo Davidiano em Waco e a família de Randy Weaver em \emph{Ruby Ridge}, ele também estava preparando o \emph{Anti-Terrorism and Effective Death Penalty Act}, que antecipava as medidas de segurança do governo Bush.

Enquanto houver estado, as autoridades estatais buscarão mais poder.

\section{Se você se importa com a liberdade, rejeite o estado}

Quer se trate de liberdade de expressão, privacidade, sexo, guerra às drogas ou violência policial, o estado é inimigo da liberdade individual. O estado continua buscando mais poder. Ele continua encontrando novas maneiras de limitar as oportunidades dos outros de tomar decisões sobre as suas próprias vidas. Ele proporciona aos puritanos maneiras baratas de regular o comportamento dos outros e dá aos seus próprios agentes autorizações irrestritas para fazer uso da violência contra pessoas comuns. Enquanto houver estado, a liberdade individual estará em sério perigo.