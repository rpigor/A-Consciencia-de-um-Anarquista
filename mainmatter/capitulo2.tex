% Segundo capítulo

\chapter{Peixes, Bicicletas e o Estado}
\label{chap:2}

\lettrine[lines=2]{\textcolor{LettrineColor}{\textbf{E}}}{u sou anarquista porque} acredito que o estado não é nem necessário nem inevitável. Não precisamos do estado para prevenir a violência e preservar a ordem. O estado não é capaz de administrar a economia. E, apesar da pressão estatista, alternativas ao estado floresceram -- o que faz com que seja difícil ver o estado como inevitável.

\section{O estado como mantenedor da paz?}

Mesmo que ele não seja legítimo, dirão alguns estatistas, o estado é \emph{útil}. Apesar dele ser ilegítimo, realmente devemos apoiar a manutenção da autoridade do estado porque \emph{precisamos} dela. Precisamos dela, segundo o argumento, porque a ameaça de violência \emph{estatal} é necessária para nos proteger da violência de \emph{uns dos outros}. Se as pessoas souberem que o estado intervirá nos conflitos privados para manter a paz, teremos menor probabilidade de sermos roubados, assaltados e assassinados.

Para quem propõe esse tipo de argumento, a questão da legitimidade é irrelevante. Não importa se consentimos com a autoridade do estado ou não. Se não tivermos consentido, e daí? Consideração prudencial por nós mesmos e preocupação benevolente pelos outros determinam que mantenhamos o poder do estado. Caso contrário, nos encontraremos submersos em conflitos constantes, muitas vezes violentos.

É importante entender o que esse argumento \emph{não} estabelece. Ele não fornece nenhuma razão direta para que se preste atenção a qualquer comando decretado pelo estado. Ele \emph{apenas} fornece um argumento para apoiar a operação contínua do estado \emph{como um mecanismo para prevenir a violência} contra as pessoas ou seus bens. Se o estado opta por criminalizar alguma prática sexual que a maioria acontece de não gostar, por exemplo, ele está usando sua autoridade para reprimir a dissidência e impor a conformidade, e não, \emph{per se}, para inibir conflitos violentos. A menos que possa ser demonstrado que qualquer desacordo social corre um sério risco de se tornar uma disputa violenta e, portanto, requer ação preventiva por parte do estado, esse argumento sugere que o único tipo de estado que merece apoio é aquele que protege as pessoas contra a violência real -- um estado mínimo, de fato (e bastante diferente de qualquer estado que consigo imaginar no mundo atual ou em qualquer ponto da história do mundo).

\section{O quanto precisamos do estado?}

Mas por que deveríamos assumir, de qualquer forma, que precisamos do estado -- uma organização com o monopólio do uso da força em um determinado território -- para nos proteger contra a violência? 

Afinal, as pessoas podem se proteger contra a violência. Vizinhos podem vigiar as casas e locais de trabalho uns dos outros; eles podem trabalhar juntos para repelir quem é violento. E, mesmo sem o estado, o trabalho de algumas pessoas poderia ser o fornecimento de proteção contra a violência: o emprego de alguém poderia ser o de defender os outros de ataques violentos (e, talvez, de realizar tarefas relacionadas, como recuperar bens perdidos e obter reparações dos agressores). Não há nada logicamente contraditório ou praticamente impossível na entrega desses tipos de serviços por voluntários ou trabalhadores sem o envolvimento do estado. Por que a necessidade de serviços de proteção implicaria qualquer necessidade do estado?

Uma resposta comum é que, sem o estado, os mantenedores da paz, sejam eles voluntários ou profissionais, poderiam acabar envolvidos em conflito entre si. Assim, dizem os estatistas, uma estrutura abrangente é essencial para evitar conflitos violentos entre facções armadas.

À primeira vista, essa afirmação não parece totalmente plausível. Afinal, não existe um estado mundial supervisionando o comportamento de cada país. Mas a maioria deles não está em guerra na maior parte do tempo. Em vista dos custos da violência, e porque as pessoas são mais propensas a aderir a normas que exigem paz, uma autoridade abrangente com o monopólio da violência não parece obviamente necessária para impedir que atos agressivos ocorram.

Grupos individuais de vizinhos e trabalhadores terão razões semelhantes para evitar se envolver em atos de violência. E, em uma pequena escala, no nível do bairro ou cidade, os custos da agressão serão ainda maiores: será mais fácil que comunidades mantenham normas anti-agressivas e que vizinhos que desaprovam o comportamento agressivo dos outros os penalizem por suas ações pouco razoáveis. E, é claro, os custos e problemas de coordenação envolvidos quando um bairro tenta se defender contra bandidos de outro bairro serão muito mais administráveis do que aqueles envolvidos quando um estado, com fundos originados de impostos à sua disposição, entra em guerra com outro estado.

Um estado, por definição, exerce o poder de monopólio. E monopolistas são notoriamente ineficientes. Quando uma empresa pode impedir legalmente que qualquer outra faça o mesmo trabalho que ela realiza, ela cobrará preços exorbitantes e fornecerá um serviço ruim. Nossa experiência com outros monopolistas certamente não nos dá nenhuma razão para pensar que o estado, um monopólio, provavelmente fornecerá segurança, justiça e outros serviços de alta qualidade a baixo custo. E, claro, o estado está sob uma pressão ainda menor para fornecer serviços de alta qualidade e baixo custo do que um monopolista comum: um monopolista comum pode excluir outros de fornecer os bens e serviços que ele oferece, mas as pessoas geralmente são livres para evitar a compra desses bens e serviços; em contrapartida, o monopólio que o estado é pode e de fato força as pessoas a comprarem o que ele vende, nas condições que ele mesmo estabelece.

Ainda mais problemático é o fato de que o estado é uma entidade extremamente \emph{perigosa}. Ele é frequentemente violento -- em larga escala. Embora os estados de fato controlem o mau comportamento de gangues menores de criminosos, eles frequentemente oprimem seu próprio povo e atacam e espoliam o povo de outros estados. Não existe em última análise uma maneira significativa de agregar e comparar diferentes atos de violência. Mas parece claro que as mesmas razões gerais que podemos ter para temer atos violentos cometidos por outras pessoas são razões para temer o mau comportamento do estado.

Os custos diretos e indiretos da violência são consideráveis e não quero subestimá-los. Esses custos estão certamente entre as importantes razões pelas quais uma sociedade sem estado não teria de ser destruída pela violência entre facções armadas. Mas não há razão alguma para pensar que a maioria das pessoas, na maioria das sociedades hoje, são pacíficas e cooperativas principalmente porque temem que o estado retaliará com violência se elas se comportarem de maneira agressiva. A maioria das pessoas, suspeito eu, respeitam as normas sociais que exigem interações pacíficas e voluntárias com outras pessoas por outras razões. Elas conseguem compreender a razoabilidade dessas normas, tanto do ponto de vista moral quanto prático (precisamos uns dos outros, afinal, e paz e cooperação geralmente são mais agradáveis do que violência). E essas normas foram incutidas nessas pessoas por ensinos e modelos, tanto deliberados quanto inconscientes. E poderia-se razoavelmente esperar que o mesmo tipo de ensinos e modelos desempenhariam o papel que já desempenham hoje em uma sociedade sem estado.

\section{Paz interna, guerra externa}

Consigo imaginar que um estatista possa argumentar, em resposta ao que eu disse, que há consideravelmente menos violência dentro de um determinado país do que entre países (não tenho certeza se concordo com isso, mas aceitarei para fins de argumentação) porque dentro de um país existe um sistema de legislação e resolução de disputas que foi acordado.

O estatista pode dizer algo assim: não importa se existe uma única agência policial dentro de um estado: muitas vezes, na verdade, existem muitas dessas tais agências, muitas vezes administradas e financiadas de forma independente. O que importa, ao invés disso, é que haja um acordo geral sobre os princípios jurídicos que essas agências devam seguir e sobre os tribunais cujas decisões elas devam aplicar. É esse acordo, o estatista pode argumentar, que garante que diversas agências de aplicação da lei possam cooperar para manter a paz dentro de um estado.

Observe que, a esse ponto, o estatista fez uma enorme concessão ao anarquista. O estatista reconheceu que uma única agência absolutamente poderosa não é necessária para manter a paz. Considere os Estados Unidos: não há base plausível para sustentar que todas as agências nacionais, estaduais e locais de aplicação da lei formem um único empreendimento cooperativo, uma organização gigante e coordenada. Essas agências certamente influenciam umas às outras. Claramente existem pessoas que gostariam de centralizar o controle das agências de aplicação da lei, e temos toda razão para temer o tipo de poder que poderia ser exercido sobre as pessoas comuns se elas \emph{fossem} centralizadas. Mas hoje em dia elas são obviamente independentes e o estatista não parece inclinado a disputar isso: ele está reconhecendo que muitas diferentes agências de aplicação da lei podem coexistir pacificamente. No entanto, defende ele, sua coexistência pacífica depende do reconhecimento mútuo da autoridade do sistema jurídico.

O estatista, no entanto, não deveria dar muita importância a esse fato. Afinal, existem muitos sistemas jurídicos diferentes. Os policiais em Louisiana não aplicam e obedecem às mesmas leis estaduais e decretos locais que suas contrapartes em Massachusetts, nem respondem aos mesmos tribunais. Uma considerável diversidade jurídica é claramente compatível com a paz social. E está claro que as pessoas podem resolver disputas pacificamente apesar dos conflitos entre sistemas jurídicos: tribunais podem aplicar regras de conflito de leis\footnoteNT{N.T.: \emph{conflict-of-law rules}. No contexto de regras do direito internacional, que mediam disputas envolvendo jurisdições diferentes.} para garantir que um processo razoável seja seguido e um resultado razoável alcançado quando alguém de Wyoming processa alguém de Missouri por uma disputa que diz respeito a um evento na Califórnia, mas que está, por acordo, sujeito às leis de Delaware. (Na verdade, regras de conflito de leis tornam possível a resolução metódica de disputas envolvendo sujeitos e sistemas jurídicos de diferentes \emph{estados}.)

O estatista talvez concorde que é possível haver disputas metódicas entre pessoas que se identificam com comunidades cujos sistemas jurídicos diferem. Mas ele pode optar por manter uma posição recuada: no mundo de hoje, estados e governos locais reivindicam autoridade absoluta sobre as pessoas que vivem em seus respectivos territórios; as pessoas podem confiar que aqueles supostamente sujeitos a outros sistemas jurídicos manterão seus acordos porque seus governos os obrigarão a fazê-lo. Mas ninguém obriga agentes governamentais a exigir que as pessoas cumpram seus acordos. Eles cooperam uns com os outros, suspeito eu, como resultado de uma combinação de fatores: normas que ditam justiça e cooperação, o desejo de reputações que levem à contínua confiança e cooperação e os custos dos conflitos que poderiam surgir caso eles encorajassem as pessoas a ignorar suas obrigações. Os mesmos tipos de fatores incentivariam as pessoas em uma sociedade sem estado a cooperar umas com as outras; eles também alinhariam as pessoas que tomam decisões, influenciando instituições das várias comunidades em tal sociedade, de forma a favorecer a cooperação em vez da negligência para com as obrigações.

O exemplo local que o estatista deseja invocar aqui de fato parece depender principalmente de um consenso sobre regras de escolha da lei aplicável\footnoteNT{N.T.: \emph{choice-of-law rules}. No mesmo contexto da nota de tradução anterior.}, uma vez que, obviamente, não depende da existência de um único corpo de legislação relevante ou de uma única agência de aplicação da lei. E regras de escolha da lei aplicável podem obviamente ser utilizadas para resolver disputas em uma sociedade sem estado da mesma forma que em um estado moderno com diversos sistemas jurídicos. 

Talvez o estatista queira dizer que, embora existam diversos tipos de agências de aplicação da lei e sistemas jurídicos nos Estados Unidos, por exemplo, a polícia nacional e agências militares estão sempre à disposição para resolver conflitos entre eles. Segundo essa visão, a capacidade de coexistência dos diversos sistemas jurídicos depende da disponibilidade da violência do estado em segundo plano como meio de regular disputas. Sem a ameaça da força pelas agências nacionais, conflitos entre as agências locais de aplicação da lei seriam tão frequentes quanto conflitos entre exércitos nacionais.

No entanto, não tenho certeza se o monopólio da violência do estado é realmente o único fator relevante aqui. Primeiro, em alguns estados a violência é comum. A vida de muitas pessoas em muitos estados é violenta e, pois bem, bastante ruim. Estados nem sempre mantêm a violência sob controle. E estados frequentemente se envolvem em contínuos atos de violência contra pessoas que vivem dentro de ``suas'' fronteiras. Então não está claro se uma comparação entre violência intraestadual e violência interestadual sempre resulta a favor do estado. Em segundo lugar, o nível de violência dentro de um determinado estado que não resulta dos crimes do próprio estado não é apenas consequência do grau em que o estado ameaça usar a força contra indivíduos violentos. Pobreza, laços econômicos, normas culturais e homogeneidade cultural (ou a existência ou a não existência de normas culturais que promovem respostas bem-sucedidas à heterogeneidade cultural) também importam. Não é muito surpreendente que o nível de violência na maioria das sociedades ocidentais seja relativamente baixo. Mas o baixo nível de violência provavelmente é consequência do fato de que essas sociedades são economicamente confortáveis, que as pessoas são economicamente interdependentes e que os valores que apoiam a cooperação e a paz social são amplamente difundidos. Todos esses fatores parecem estar presentes nessas sociedades, quer exista estado ou não. Então a relativa estabilidade dessas sociedades não fornece evidências particularmente fortes para a importância da autoridade do estado.

Normas que favorecem a cooperação e a justiça e se opõem à força agressiva provavelmente tenderiam a manter as coisas relativamente calmas mesmo sem a ameaça de violência do estado. Além disso, os custos de atacar os outros seriam consideráveis -- e, sem o estado, seriam arcados (dependendo de como uma determinada comunidade organizou os serviços de segurança) por voluntários, membros de cooperativas de defesa mútua, instituições de caridade ou trabalhadores de tempo integral no ramo da segurança e defesa e aqueles que os pagam. É provável que nenhuma dessas pessoas ficariam muito felizes em arcar com os custos financeiros associados a conflitos violentos, custos que incluem tempo desperdiçado, recursos perdidos, reputação manchada, danos físicos e morte. E isso significa que haverá considerável pressão para evitar esse tipo de conflito, de fazer acordos com outras pessoas que tendem a reduzí-los e de evitar pessoas que tendem a provocá-los.

Eu falei aqui sobre comunidades e, para simplificar, tratei uma comunidade como localizada geograficamente. Mas ela certamente não precisa ser assim. As pessoas podem pertencer a diversas comunidades sobrepostas. E diferentes comunidades -- diferentes redes e organizações socais, congregações religiosas, clubes, grupos de pessoas envolvidas no mesmo tipo de trabalho -- podem perfeitamente manter diferentes sistemas jurídicos. Diferentes tipos de grupos podem desenvolver corpos jurídicos apropriados para diferentes tipos de circunstância e diferentes tipos de disputa. E onde as preocupações de grupos diferentes se sobrepõem, os mesmos tipos de regra de conflito de leis que regem as disputas entre pessoas de comunidades territorialmente distintas podem ser aplicadas.

\section{Um tipo diferente de governança}

Lembre-se das palavras da Declaração. Seus signatários insistiram ``que, sempre que qualquer forma de governo se torne destrutiva de tais fins, cabe ao povo o direito de alterá-la ou aboli-la e instituir novo governo''. Mas e se sequer nenhum estado receber (ou merecer) o consentimento do povo? Os diversos e imprevisíveis mecanismos de resolução de disputas que se desenvolverão em uma sociedade sem estado podem ser fontes confiáveis de ordem e segurança. Substituir o estado pelo tipo de anarquia que estou abordando aqui não é uma questão de eliminar o governo se ``governo'' realmente significa apenas ``governança'' -- gestão, administração, criação e manutenção da ordem. Criar uma sociedade sem estado significa criar um tipo particular de gestão, administração, manutenção da ordem -- um tipo que não depende da violência, da posse de ninguém do monopólio da força.

A paz é produtiva; a violência é custosa. Laços comunitários unem pessoas umas às outras e reforçam as normas de cooperação, justiça e compaixão. E à medida que comunidades humanas exploram e experimentam, elas podem conceber uma enorme variedade de estratégias criativas para a resolução de conflitos sem o monopólio da força do estado. Ao contrário do argumento estatista de que não teríamos paz sem o estado, as pessoas que querem paz precisam menos do estado do que um peixe precisa de uma bicicleta (a bicicleta, afinal, dificilmente fará mal ao peixe, enquanto o estado é positivamente perigoso).

\section{O estado e a economia}

O argumento mais comum a favor do estado nos dias de hoje é provavelmente a afirmação de que ele é necessário para prevenir a violência. Creio que esteja claro que esse argumento não funciona, que diversos meios de preservar e restaurar a paz estariam disponíveis em uma sociedade sem estado. Outro argumento de que o estado é indispensável pode ser aquele em que as pessoas precisem do estado porque uma autoridade central é necessária para guiar a produção e distribuição de bens e serviços. Mas também não acredito que o estado seja necessário para gerir a produção e distribuição de maneira eficiente -- na verdade, quase certamente ele \emph{não pode} fazê-lo.

Existem várias razões interrelacionadas para isso. Acredito que a mais essencial é que o estado não sabe realmente o que as pessoas querem ou quais recursos estão disponíveis. Ele sem dúvidas poderia reunir um vasto banco de dados de todos os recursos físicos, trabalhadores e suas habilidades disponíveis. Mas nada como isso existe hoje em dia; seria extremamente caro e demorado; exigiria a implementação de um poder computacional quase inimaginável; e apenas ter as informações necessárias sem dúvida daria aos agentes estatais uma capacidade tremenda de manipular a vida das pessoas.

Mas suponhamos que todos esses obstáculos possam ser superados. Ainda seria o caso que o estado não teria conhecimento das preferências reais das pessoas por bens e serviços. Não, não há dúvidas de que ele poderia simplesmente gerir a produção e distribuição sem considerar as preferências das pessoas, apenas decidindo o que elas precisam e entregando os bens e serviços que ele decidiu produzir como resultado. Mas alguém realmente acredita que isso seria sensato? As pessoas sabem muito mais sobre suas próprias circunstâncias e, portanto sobre o que seria útil ou não para elas, do que burocratas do estado. E, claro, apenas desfrutar da liberdade de escolher qual dos diversos bens se deseja é importante por si só.

Talvez o estado pudesse fazer uma pesquisa com as pessoas para determinar as suas preferências. Ele poderia até mesmo manter registros de pesquisas individuais para que soubesse exatamente o que cada indivíduo deseja. Mas, novamente, isso daria ao estado um poder enorme. Isso exigiria um imenso investimento em gerenciamento de dados. E exigiria que as pessoas dedicassem um bom tempo respondendo às perguntas da pesquisa.

Imagine que esse conjunto de problemas pudesse ser resolvido. Ainda assim não resolveria a dificuldade associada a recursos finitos: nem todas as coisas que desejo são coisas que posso ter; tenho que fazer escolhas; tenho que racionar recursos escassos. Para determinar como realizar essa tarefa, o estado teria que pedir às pessoas não apenas para fornecer preferências, mas para pesá-las em relação umas às outras, para haver uma ideia de quanto tempo e energia devem ser gastos na entrega de quais bens e serviços em específico.

Para complicar ainda mais as coisas para o estado, nem sempre é óbvio quais informações são relevantes para o processo de planejamento da produção e distribuição de determinados bens. O estado pode não saber se deve ou não adquirir as informações de determinadas pessoas e essas próprias pessoas podem não perceber que certas informações que possuíam eram relevantes.

Além disso, é extremamente improvável que o estado saiba com antecedência a maneira mais eficiente de resolver um problema específico de produção ou distribuição. O estado poderia direcionar investimentos a uma gama específica de abordagens -- mas ainda assim correria um risco significativo de ignorar alternativas que poderiam ser identificadas pelos outros de maneira criativa. Se, no entanto, ele apoiasse diversos empreendimentos de diferentes pessoas para identificar mecanismos criativos de produção ou distribuição, ele não saberia de antemão quais merecem apoio e, os que merecem, o quanto. É difícil imaginar que um sistema dirigido pelo estado não sufocaria a criatividade e ignoraria possibilidades inovadoras.

Esse ponto é especialmente evidente se você pensar sobre a questão, não, ``\emph{Quantos widgets devemos fazer?}'', mas sim, ``\emph{Devemos fazer widgets ou zidgets}?'' ou, de maneira ainda mais essencial, ``\emph{Devemos fazer uma fábrica que possa ser usada para construir peças para máquinas de fazer widgets (e muitas outras coisas), ou um tipo completamente diferente de fábrica}?'' Sob qual possível base o estado deveria tomar essas decisões? Que informação sobre as possíveis preferências do consumidor pode-se pensar que ele tem? Como recursos seriam racionados entre esses tipos de usos básicos? Suponha que o estado administre a economia determinando o nível de investimento em fábricas, terras comerciais e industriais, infraestrutura e assim por diante. Ele não pode perguntar sobre as \emph{preferências} do consumidor nesses casos e procurar condizer os níveis de investimento a essas preferências. Isso porque o próprio estado \emph{é} o consumidor -- ele será o comprador das terras e das fábricas -- e o que ele está tentando fazer é descobrir quais exatamente suas preferências deveriam \emph{ser}.

Existem razões gerais para pensar que a gestão eficiente da produção e distribuição de bens e serviços pelo estado (ou, de fato, por qualquer autoridade centralizada) não funcionaria. E essas razões se aplicam a quaisquer bens e serviços que eles realmente \emph{sejam}. Isso inclui a produção e distribuição de bens e serviços que os defensores do estado tipicamente querem sugerir que somente ele pode fornecer, aqueles que nossa necessidade supostamente ajuda a justificar o estado, mesmo que ele seja ilegítimo. O argumento, lembre-se, é de que precisamos do estado para evitar que matemos uns aos outros e para resolver conflitos. Mas não há nenhuma razão especial para pensar que o estado seria melhor em gerir centralizadamente a produção e distribuição econômica de serviços de prevenção da violência e resolução de conflitos do que em gerir centralizadamente a produção e distribuição econômica de outros bens e serviços. A inadequação do estado como gestor econômico ajuda a enfraquecer um argumento crucial frequentemente oferecido a favor de sua continuada existência.

Há também outro problema muito real relacionado à gestão estatal da produção e distribuição. O poder exigido para implementar tal sistema seria enorme. A tentação oferecida ao tirano em potencial pela disponibilidade de tal poder seria grande. E possuir esse tipo de poder possibilitaria que até mesmo burocratas bem-intencionados que cometessem erros causassem um enorme dano em pouco tempo.

O fato de que o estado não consegue gerir com sucesso a produção e distribuição econômica de bens e serviços não \emph{prova} que não deveria haver estado. Talvez, por exemplo, existam bens e serviços para os quais há boas razões que as pessoas desejem, mas que não podem ser produzidos economicamente. É por isso que meu argumento contra o estado é dividido em várias partes. Eu acredito, de fato, que é possível produzir todos os bens e serviços que poderíamos desejar sem o estado. Mas mesmo que se pudesse ser demonstrado que esse não é o caso, os outros tipos de razões que enfatizei, incluindo a ilegitimidade e a periculosidade essencial do estado, pesariam muito contra o estado. Acredito que faria sentido abandonar o estado mesmo que isso também significasse abandonar algumas outras coisas que poderíamos genuinamente desejar.

Os custos diretos de operação de um sistema estatal de gestão econômica seriam enormes. E existiriam diversos custos indiretos. Isso incluiria a produtividade reduzida resultante de qualquer tentativa de administrar o processo criativo de maneira centralizada. Também incluiriam diversos outros tipos de problemas -- evidentes nos dias de hoje mesmo em economias que não são totalmente administradas pelo estado. Estes incluem ineficiências criadas por subsídios estatais para elites privilegiadas, encargos impostos pelo estado sobre os pobres e incentivos fornecidos pelo estado que sustentam grandes organizações empresariais centralizadas -- incentivos que estimulam atividades econômicas ineficientes e sustentam organizações nas quais frequentemente é debilitante, alienante e opressivo trabalhar.

\section{A praticidade da anarquia}

Obviamente, resistir ao desenvolvimento de sociedades sem estado é muito do interesse das pessoas que supervisionam e lucram com as operações dos estados. O estado é tão claramente custoso e opressor que muitas pessoas provavelmente explorariam alternativas a ele com bastante entusiasmo se imaginassem que ele pudesse ser substituído. O argumento mais prático a favor do apoio ao estado é de que realmente não há alternativa à sua continuada existência. Apontei algumas razões gerais do porquê pode fazer sentido ser cético sobre essa afirmação. Mas o anarquista não precisa depender somente de argumentos abstratos sobre o que poderia funcionar ou o que provavelmente funcionaria. Em uma variedade de contextos, as pessoas parecem prosperar sem muita ou nenhuma ajuda de uma autoridade com o monopólio da força.

\subsection*{Relações entre estados}

Comece com o óbvio: não existe um estado mundial. Alguns estados são muito mais poderosos que outros, mas nenhum estado exerce ou reivindica algo parecido com o monopólio da força em todo o mundo (mesmo que alguns sem dúvidas gostariam de fazê-lo). É claro, existem conflitos. Mas na maior parte do tempo os estados interagem pacificamente. Eles resolvem disputas. Eles reconhecem a legitimidade dos resultados dos processos de resolução de disputas. E não há nenhum Big Brother pronto para obrigá-los a fazer isso. É razoável apostar que eles fazem isso por diversos motivos: a violência é custosa; reputações são importantes de se manter; e sem dúvidas, pelo menos às vezes, os funcionários do estado realmente querem cooperar com os outros e realmente internalizaram normas que exigem um comportamento justo, respeitoso e pacífico. Independentemente da explicação em qualquer caso em particular, no entanto, os estados interagem sem o auxílio de um Leviatã mundial.

\subsection*{Comércio internacional}

E quanto às pessoas e organizações envolvidas na troca de bens e serviços além das fronteiras estatais? Sem um Leviatã global, não há ninguém para criar ou implementar qualquer tipo de lei comercial global. Certamente, existem tratados (embora, novamente, não haja nenhum estado mundial para fazer os estados aderirem a eles). Mas os tratados dificilmente cobrem todas as questões que podem surgir nas transações comerciais. Às vezes, árbitros resolvem disputas comerciais internacionais. Às vezes, tribunais nacionais os resolvem -- mesmo que fazê-lo signifique ter que interpretar e aplicar leis estrangeiras desconhecidas. Pessoas que são membros de diferentes comunidades políticas com diferentes sistemas jurídicos conseguem resolver disputas jurídicas sem depender de uma única autoridade abrangente com um monopólio da força e, às vezes, de fato, com a ajuda de mecanismos voluntários de arbitragem. O fato de que as disputas comerciais internacionais podem ser resolvidas sem a ajuda do Leviatã sugere que o estado é menos essencial do que as pessoas costumam supor.\footnoteA{Peter T. Leeson, ``One More Time with Feeling: The Law Merchant, Arbitration, and International Trade,'' \emph{Indian Journal of Economics and Business} spec. iss. (Sep. 2007): 29-34.}

\subsection*{A resolução autônoma de disputas dos comerciantes}

As pessoas também podem optar por sair do sistema jurídico do estado quando elas lidam umas com as outras \emph{dentro} das fronteiras do estado. Muitas pessoas fizeram isso por séculos durante a Idade Média e o Renascimento. A \emph{lex mercatoria}, a lei dos próprios comerciantes, surgiu como uma resposta previsível à ausência de padrões uniformes para transações comerciais envolvendo comerciantes de diferentes regiões. Comerciantes em viagem poderiam tirar proveito dos tribunais mercantis estabelecidos em feiras comerciais em toda a Europa. Os tribunais impunham padrões aceitáveis à comunidade mercantil -- bem razoáveis, uma vez que os padrões se aplicavam a disputas entre os próprios comerciantes. Os padrões foram aqueles que tinham evoluído ao longo do tempo à medida que comerciantes descobriam o que funcionava e o que não funcionava, o que era justo e o que não era, dados os tipos de circunstâncias em que eles caracteristicamente se encontravam. E mecanismos eficazes de aplicação muitas vezes incluíam boicotes dirigidos àqueles que se recusavam a pagar ou então a aceitar os julgamentos dos tribunais mercantis.

Sem dúvidas, os tribunais eclesiais e estatais também se envolveram em disputas comerciais, e boicotes dificilmente eram os únicos mecanismos de aplicação de leis. E não é como se houvesse um único código mercantil uniforme, escrito em tipo frio com seções enumeradas, aceito por todos comerciantes: havia, sem dúvidas, variações locais e a lei local certamente tinha papel importante nessa ocasião. Com todas essas qualificações, no entanto,\footnoteA{Veja, por exemplo, Paul Milgrom, Douglass North e Barry Weingast, ``The Role of Institutions in the Revival of Trade: The Law Merchant, Private Judges, and the Champaigne Fairs,'' \emph{Economics and Politics} 2.1 (March 1990):1-23. Mas cp. Stephen Edward Sachs, ``From St. Ives to Cyberspace: The Modern Distortion of the Medieval `Law Merchant,'\,'' \emph{American University International Law Review} 21.5 (2006): 685-812. Focando na feira de St. Ives (registros de tribunal relacionados os quais ele examina detalhadamente), Sachs critica as afirmações de que a \emph{lex mercatoria} era uniforme, que as disputas dos comerciantes eram resolvidas principalmente pelas próprias instituições dos comerciantes e que autoridades locais não estavam envolvidas na resolução de tais disputas.} parece que os comerciantes medievais frequentemente conseguiam desenvolver e aplicar normas legais de maneiras que os ajudavam a resolver disputas envolvendo pessoas de diferentes jurisdições por conta própria, sem a participação do estado. Gerações posteriores de comerciantes continuaram a fazê-lo.\footnoteA{Veja, por exemplo, Lisa Bernstein, ``Opting out of the Legal System: Extralegal Contractual Relations in the Diamond Industry,'' \emph{Journal of Legal Studies} 21.1 (Jan. 1992): 145-53; Edward P. Stringham, ``The Extralegal Development of Securities Trading in Seventeenth Century Amsterdam,'' \emph{Quarterly Review of Economics and Finance} 43.2 (Sum. 2003): 321-44.} A experiência deles também levanta questões óbvias sobre a suposta indispensabilidade do estado como uma fonte de regras legais e de aplicação da lei.

\subsection*{Islândia medieval}

A Islândia medieval não era um paraíso anarquista. Mas funcionou razoavelmente bem sem instituições estatais para a administração da justiça e manutenção da paz civil.\footnoteA{David D. Friedman, ``Private Creation and Enforcement of Law: A Historical Case,'' \emph{Journal of Legal Studies} 8.3 (Mar. 1979): 399-415; cp. William Ian Miller, \emph{Bloodtaking and Peacemaking: Feud, Law and Society in Saga Iceland} (Chicago: U of Chicago P 1997).} Nenhuma entidade detinha o monopólio do uso da força. Além dos fundos necessários para sustentar um único trabalhador de meio período, o sistema legal não dependia de qualquer tipo de receita tributária. Havia uma aceitação geral de um conjunto de normas que governavam a maneira que as instituições limitadas -- júris, assim como outros grupos, organizados mas basicamente voluntários, que arcavam com os custos de acesso dos membros aos tribunais -- atendiam às necessidades das pessoas. As instituições eram suficientemente estáveis para durar vários séculos, durante os quais a Islândia não teve um rei, uma burocracia ou um sistema tributário.

\subsection*{Irlanda medieval}

As coisas eram parecidas na Irlanda medieval. Embora existissem reis -- regionais ou da ilha inteira -- em diversos momentos, eles tinham pouco poder, e tentativas de estabelecer e manter reinos nem sempre foram malsucedidas. As pessoas pertenciam a associações voluntárias que atestavam por elas e garantiam que elas compensassem se ferissem outros. Conflitos eram resolvidos por juízes não profissionais. Como na Islândia, não havia nenhuma noção jurídica de crime como um delito contra o estado; em vez disso, todos os conflitos legais envolviam supostos prejuízos a pessoas particulares. Embora a Irlanda nesse período tivesse algumas características superficiais dos estados posteriores, ela apresentava um sistema de justiça que claramente desmentiu a noção de que uma autoridade centralizada com um monopólio da força é necessária para resolver conflitos potencialmente violentos.\footnoteA{Cp. Joseph R. Peden, ``Property Rights in Celtic Irish Law,'' \emph{Anarchy and the Law: The Political Economy of Choice}, ed. Edward P. Stringham (Oakland, CA: Independent 2007) 565-85.}

\subsection*{Condado de Shasta, Califórnia}

Fazendeiros no condado de Shasta, na Califórnia, poderiam sem dúvidas valer-se do sistema de tribunal local. Mas, via de regra, eles não o fazem. Eles evoluíram um conjunto de normas que governam os tipos de conflitos que provavelmente encontrarão e as maneiras pelas quais essas normas devem ser aplicadas. Aderindo a essas normas, eles administram uma gama de disputas sem muita dependência do estado.\footnoteA{Robert Ellickson, \emph{Order without Law: How Neighbors Settle Disputes} (Cambridge, MA: Harvard UP 1994).}

\subsection*{Somália contemporânea}

A Somália não teve um governo central por uma década e meia. Aqui está o que é fascinante: os somalis estão melhores do que estavam antes. As instituições locais ajudaram as pessoas a resolver disputas de maneira satisfatória e facilitaram trocas econômicas com pessoas em ou de outros países -- tudo sem o suposto benefício de alguma entidade com um monopólio da força.

A Somália permanece desesperadamente pobre e assolada com a violência, e sem dúvidas alguém poderia imaginar um governo perfeito e sem defeitos com a capacidade de melhorar as coisas. Mas a realidade continua sendo que, apesar da tremenda pressão externa e da bandidagem interna, os somalis continuaram a manter uma sociedade autogerida e sem estado que representa uma melhora distinta em relação à ditadura para a qual eles haviam sofrido anteriormente.\footnoteA{Peter T. Leeson, ``Better Off Stateless: Somalia before and after Government Collapse,'' \emph{Journal of Comparative Economics} 35.4 (2007): 689-710 <\url{http://www.peterleeson.com/Better_Off_Stateless.pdf}> (July 1, 2010); Benjamin Powell, Ryan Ford e Alex Nowrasteh, ``Somalia after State Collapse: Chaos or Improvement?,'' \emph{Journal of Economic Behavior and Organization} 67.3-4 (Sep. 2008): 657-70 <\mbox{\url{http://www.independent.org/pdf/working_papers/64_somalia.pdf}}> (July 1, 2010). Veja também Michael Van Notten, \emph{The Law of the Somalis: A Stable Foundation for Economic Development in the Horn of Africa}, ed. Spencer Heath McCallum (Trenton, NJ: Red Sea 2005). Van Notten passou mais de uma década estudando a lei consuetudinária da Somália a fim de aprofundar a compreensão de seu potencial para servir como um modelo para uma ordem jurídica sem estado.}

\subsection*{A Internet}

O estado gostaria de comandar a Internet. Mas ele não o faz. Online, as pessoas conseguem evitar os decretos do estado -- proclamando dissidência, planejando manifestações, se envolvendo em transações proibidas de todos os tipos. Obviamente, eles não podem contar com a ajuda do estado se uma das transações proibidas der errada. Mas mesmo quando um acordo perfeitamente legal está em questão, com que frequência alguém invoca a autoridade do estado? Websites fornecem mecanismos tanto para a avaliação da reputação de pessoas oferecendo comprar ou vender bens e serviços, quanto para resoluções de disputas também. Conveniência, pressão social, normas internalizadas, depósitos em dinheiro e a necessidade de manter o nível de confiança necessário para que os outros estejam interessados em trocar bens e serviços com eles sem dúvidas aumentam a probabilidade das pessoas manterem acordos e, se não o fizerem, aumentam a probabilidade delas cumprirem as decisões dos mecanismos de resolução de disputas acordados.

\subsection*{O Oeste ``Selvagem'' \footnoteNT{N.T.: \emph{The ``Wild'' West}. Optei por essa tradução em alternativa ao mais usual ``Velho Oeste'' devido ao jogo linguístico feito pelo autor com a palavra ``selvagem''.}}

O Oeste não era tão selvagem quanto os filmes repetidamente o fazem parecer. Na realidade, era um lugar relativamente pacífico, com relativamente poucos casos de violência. E -- nenhuma surpresa até agora -- o estado não estava especialmente envolvido. É claro, parte do Oeste estava teoricamente sob controle do governo dos Estados Unidos e de vários governos estaduais e territoriais. Mas a realidade é que a capacidade das autoridades de impedir ou resolver conflitos foi limitada pelos custos de transporte e comunicação e o número relativamente pequeno de funcionários do governo disponíveis.

Então como as pessoas viviam? Sem o envolvimento prático do estado, ou de qualquer coisa parecida com o estado, as pessoas resolviam disputas, recuperavam bens roubados e faziam justiça quando as coisas davam erradas. Às vezes, as sanções que eles impunham eram mais duras do que eu estaria inclinado a apoiar. Mas vale a pena notar que as execuções, presumivelmente os tipos mais sérios de sanções, eram mais raros do que você pode imaginar e que o devido processo parece ter sido respeitado de maneira consistente. Na ausência do estado, as pessoas lidavam com seus próprios negócios com sucesso talvez surpreendente.\footnoteA{Terry L. Anderson e Peter J. Hill, \emph{The Not So Wild, Wild West: Property Rights on the Frontier} (Stanford, CA: Stanford Economics and Finance 2003).}

\subsection*{Piratas}

Certo, aqui está a parte divertida: os piratas fizeram um trabalho surpreendentemente bom em regular as suas relações. Eles o fizeram sem o estado -- mas também como membros de organizações envolvidas em atos de violência agressiva. Isso pode sugerir, provavelmente deve sugerir, que os piratas não eram especialmente compassivos e sensatos. Apesar disso, eles se provaram capazes de ordenar seus relacionamentos de forma confiável. Eles organizavam estruturas de autoridade, decidiam com antecedência sobre a divisão de saque após um ataque, procedimentos disciplinares aprovados e assim por diante. Como? Não porque o Leviatã estava os supervisionando, mas porque, por uma questão de praticidade, eles precisavam fazê-lo se desejavam prosperar -- e, de fato, sobreviver.\footnoteA{Peter Leeson, \emph{The Invisible Hook: The Hidden Economics of Pirates} (Princeton, NJ: Princeton UP 2009).}

\section{O estado: quem precisa dele?}

Não precisamos do estado para preservar a paz ou manter a justiça e ordem social. Pessoas cooperando voluntariamente podem se manter seguras e resolver disputas de forma mais eficiente e justa do que o estado. O estado é perigoso e desperdiçador e não pode -- não poderia -- determinar razoavelmente os níveis de produção e padrões de distribuição apropriados para bens e serviços, incluindo aqueles associados à defesa e justiça. As experiências do passado e do presente de uma gama de grupos sociais -- incluindo a experiência dos estados interagindo uns com os outros -- sugerem que a pesada mão da violência estatal não é necessária para gerar a cooperação ordenada. E, na verdade, como vou destacar nos próximos três capítulos, o estado em si constitui uma enorme ameaça à liberdade e bem-estar de todos, ameaça esta que dá a todos nós boas razões para desejar e procurar por alternativas.