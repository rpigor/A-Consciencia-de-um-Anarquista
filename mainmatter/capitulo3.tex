% Terceiro capítulo

\chapter{Estado, Grandes Empresas e Privilégio Econômico}
\label{chap:3}

\lettrine[lines=2]{\textcolor{LettrineColor}{\textbf{E}}}{u sou anarquista porque} acredito que o estado tende a consolidar o poder dos ricos e ajudá-los a explorar os outros. Ele fomenta a pobreza ao garantir privilégios para os ricos e aqueles bem relacionados. Ele promove modelos hierárquicos de organização empresarial e a centralização de poder no local de trabalho. Ele cria e incentiva a persistência de monopólios e outros cartéis que aumentam o poder das elites privilegiadas às custas de todas as outras pessoas. E ele sanciona e perpetua a violência que foi e continua a ser usada para desapropriar os pobres, a classe trabalhadora e a classe média em favor de grandes proprietários de terras e ricos líderes empresariais.

\section{O estado cria elites}

O estado está ativamente envolvido em todos os aspectos da vida econômica. E, seja esse efeito deliberado ou não, o resultado prático de seu envolvimento -- ao contrário da impressão que você possa obter da mídia tradicional -- é que \emph{a balança está constantemente inclinada em favor das elites privilegiadas}. O estado cria e reforça o privilégio: regras especiais para pessoas especiais, mantidas pela ameaça ou uso da força.

Uma maneira de se olhar para esse tipo de privilégio é pensá-lo em função das maneiras pelas quais as pessoas obtêm recursos. Em termos gerais, podemos distinguir os meios \emph{econômicos}, \emph{sociais} ou \emph{civis} de aquisição de recursos, por um lado, e, por outro, os meios \emph{políticos} e \emph{militares}.\footnoteA{Ainda de maneira mais simples, podemos distinguir os meios voluntários dos involuntários. Os anarquistas tardios devem esse conjunto de distinções ao sociólogo alemão Franz Oppenheimer, que articulou e defendeu a ideia de que o estado se originou da conquista violenta. Veja Franz Oppenheimer, \emph{The State} (San Francisco: Fox 1997).}

Os meios civis de aquisição de recursos são pacíficos e não manipulativos: talvez você transforme seus recursos existentes, criando algo valioso para você; talvez alguém decida livremente dar-lhe algo; ou talvez você troque bens ou serviços livremente com os outros. Usar os meios políticos envolve violência ou manipulação: talvez alguém use violência real para tirar riquezas de outra pessoa (como o estado faz com muita frequência); talvez alguém use a ameaça de violência para se tornar rico (na maioria das vezes, quando o estado adquire recursos, ele está seguindo essa abordagem); ou talvez um agente estatal tire proveito da falsa crença de alguém de que o estado é legítimo (uma crença que ele mesmo talvez compartilhe) a fim de adquirir recursos para o estado. Em qualquer caso, a elite, aqueles que se beneficiam e controlam o estado, são aqueles que empregam os meios políticos de adquirir recursos para enriquecer. Isso os distingue nitidamente do resto de nós, que (presumindo que não somos bandidos) usamos os meios civis para obter recursos.

A divisão entre elites, que usam os meios políticos para adquirir riquezas, e pessoas que usam os meios civis é muito antiga. Não podemos ter certeza de exatamente quando os estados surgiram. Mas há boas razões para pensar que muitos ou a maioria dos estados se originaram da \emph{conquista} -- da apropriação forçada de um grupo de pessoas por outro. Pessoas que com sucesso usaram da violência para se dar privilégios -- controle sobre o trabalho, bens e terras de outras pessoas -- passaram a contar histórias reconfortantes e que legitimavam a si mesmos. ``Estamos no comando porque os deuses nos colocaram no comando'' ou ``Estamos no comando porque somos inerentemente superiores'' -- não importa realmente qual seja a história; o que é importante é que as pessoas que ganharam privilégios pela força conseguiram convencer a si mesmas (e aos outros, incluindo suas vítimas) de que elas de alguma forma \emph{mereciam} seus privilégios.

Talvez esse tipo de história das origens do estado não esteja realmente correta -- é difícil dizer, já que os primeiros estados surgiram milhares de anos atrás. O que está claro, porém, é que o mesmo tipo de dinâmica continuou ao longo da história. Na Inglaterra, por exemplo, terras antes comunais eram cercadas e apropriadas por grandes proprietários. Muitas das pessoas que ocupavam as ``Satânicas Usinas'' da Revolução Industrial haviam sido despojadas das terras nas quais trabalhavam e impedidas de procurar trabalho livremente por leis equivalentes às de passaporte interno. A violência estatal manteve os pobres sob controle e dependentes da boa vontade dos proprietários de fábricas e aristocratas.

O estado detém as rédeas da força. E, por esse motivo, ele está envolvido no fortalecimento do poder da elite. 

Uma vez que um estado está estabelecido, pessoas com riquezas e poder podem influenciá-lo e alcançar seus objetivos de maneira muito mais eficiente do que se tivessem que alcançá-los convencendo ou manipulando pessoas individuais ou pequenos grupos para agir de acordo com elas. Seduzir ou formar parcerias com um único político ou burocrata pode render enormes recompensas para uma pessoa ou grupo rico. A existência do estado, e sua inevitável suscetibilidade à manipulação, aumenta drasticamente o poder das pessoas com riquezas.

O estado contribui para a criação e manutenção de divisões de classe não apenas porque a existência da máquina estatal que pode ser facilmente tomada por meio de suborno e corrupção torna mais fácil para os ricos moldarem os resultados do processo político, mas também devido ao caráter inerentemente explorador do funcionamento do estado. Afinal, o que o estado faz rotineiramente é distorcer e restringir as interações livres e criativas das pessoas umas com as outras. O estado oferece às elites enormes oportunidades de evitar que elas paguem os custos reais de muitas das atividades nas quais elas se envolvem. As políticas do estado as ajudam a manter a sua riqueza e poder justamente ao desviar para outras pessoas os custos de fazer o que preserva e expande suas posições econômicas. Assim, o estado simultaneamente fortalece as elites, incentiva o comportamento ineficiente e explora pessoas que não fazem parte da elite.

Através de cartéis, monopólios, subsídios, criação inflacionária da moeda e a operação do sistema tributário, o estado extrai riquezas de quem as cria e as transfere para elites politicamente conectadas. Assim, ele aumenta a riqueza e o poder delas. Na ausência do estado, sem sua proteção, elas rapidamente perderiam muito de seu poder e influência; com o estado do seu lado, suas vantagens sociais e econômicas são repetidamente ampliadas.

\section{O estado brinca com monopólios}

O estado cria e sustenta a classe exploradora por meio de monopólios\footnoteNT{N.T.: \emph{The State Plays Monopoly}. Não ficou claro se era a intenção do autor fazer uma alusão bem humorada ao jogo de tabuleiro \emph{Monopoly} no título desta seção, então escolhi uma tradução mais abrangente.} e subsídios -- transferindo riqueza para elites improdutivas e as protegendo da pressão exercida por outros que desejam fornecer, por menor preço, os bens e serviços que elas oferecem para as pessoas.

No curto prazo, uma empresa específica pode se tornar dominante em um ambiente específico e talvez até ter sucesso nas tentativas de expulsar completamente outras empresas. No longo prazo, entretanto, essa empresa tenderá a se tornar grande e preguiçosa e a extrair lucros de monopólio. A esse ponto, membros do público tenderão a tentar obter os bens e serviços que ela oferece de outras que os atendem de forma mais eficiente. Como resultado, sua posição de monopólio não irá durar muito.

Pelo menos é isso que aconteceria se outras empresas realmente \emph{pudessem} fornecer os mesmos tipos de bens e serviços para os mesmos clientes do monopolista. Mas o quadro é muito diferente se o estado entra em jogo, ameaçando usar força contra aqueles que oferecem os mesmos tipos de bens ou serviços que o monopolista oferece às pessoas. Se o estado pode proibir outros de fazer o mesmo tipo de trabalho que o monopolista, o monopolista pode permanecer grande e preguiçoso e pode continuar a explorar as pessoas comuns, enriquecendo às suas custas.

O próprio estado é um monopolista, como já sugeri. Ele procura manter um monopólio do uso da força. Isso pode soar como um objetivo nobre -- afinal, quem deseja que todos sejam violentos? Mas o estado usa sua posição de monopólio para apoiar outros monopólios. Ele opera alguns, como o Serviço Postal dos Estados Unidos, diretamente. (O anarquista americano do século XIX, Lysander Spooner, foi notoriamente reprimido pelo governo dos EUA quando tentou entregar correspondências; o Congresso deliberada e implacavelmente o deixou desempregado.) Mais frequentemente, ele concede privilégios de monopólio às elites privilegiadas e seus negócios. Ocasionalmente, ele o faz porque alguns agentes estatais genuinamente acreditam que as pessoas comuns estarão em melhor condição se o estado criar ou manter um monopólio neste ou naquele setor da economia. Com maior frequência, no entanto, os agentes estatais \emph{dizem} que estão servindo ao bem comum enquanto na verdade beneficiam os seus comparsas e mantêm os preços altos ao suprimir alternativas.

O estado cria monopólios de todos os tipos. Mais fundamentalmente, é claro, além de seu próprio monopólio reivindicado do uso da força, ele reivindica o direito, na prática, de criar outros monopólios à vontade: ele diz, na prática, que possui uma espécie de poder residual de monopólio. Mas, de maneira mais específica, ele sistematicamente confere privilégios de monopólio de diversas variedades que são especialmente importantes.

\subsection*{Patentes e direitos autorais}

Por exemplo: sem o estado claramente não existiriam patentes. Uma patente impede \emph{você} de criar algo que \emph{eu} criei mesmo que você chegue ao seu entendimento do processo, composto químico ou qualquer outra coisa relevante de forma totalmente independente da minha própria descoberta paralela e mesmo que você use seus próprios bens físicos para criá-lo. Como eu convenço o estado de que o mereço, o estado me dá o privilégio, na prática, de forçar você a parar de trabalhar (convenientemente, o estado fará o trabalho sujo necessário para mim). Os direitos de patentes permitem que empresas farmacêuticas, por exemplo, extraiam enormes lucros das pessoas comuns.

Ao contrário da opinião popular, as patentes não são necessárias para estimular a produtividade. Acordos que exigem que as pessoas não revelem informações secretas podem obviamente fornecer algumas das proteções que as patentes oferecem, por exemplo (embora não poderiam nunca impedir que as pessoas desenvolvessem produtos ou processos de maneira independente). Mas, mesmo sem a proteção das patentes, uma pessoa inovadora muitas vezes teria bons motivos para querer ser a primeira a oferecer um determinado produto ou processo.

Os direitos autorais também são criações fiduciárias do estado. Os autores da Constituição dos EUA aparentemente pensavam que era óbvio que dar aos autores o monopólio sobre ``seus escritos'' iria ``promover o progresso da ciência e das artes úteis''. Mas os direitos autorais hoje em dia se estendem para muito além da morte de seus criadores, conferindo direitos muito mais extensivos do que qualquer um poderia razoavelmente esperar que fossem necessários para incentivar o trabalho criativo. Direitos autorais não são necessários para garantir que tal trabalho seja concluído: as pessoas publicam livros e escrevem músicas por razões totalmente alheias ao desejo de retornos financeiros. Acordos de ponto-de-venda poderiam obviamente exigir indenização por perdas devido à cópia de obras criativas para fins comerciais (embora estes precisariam envolver um consentimento mais deliberado por parte do comprador do que aquele envolvido nos contratos contemporâneos de embalagem descartável). E escritores e artistas poderiam oferecer aos compradores uma variedade de razões para comprar suas criações deles do que de copiadores. Direitos autorais não são essenciais para a vitalidade do processo criativo. Eles são criações do estado que permitem ações judiciais inúteis contra fãs adolescentes de música -- e a extração de lucros de monopólio por empresas de mídia.

\subsection*{Restrições imigratórias}

Restrições imigratórias discriminam trabalhadores com base na sua origem nacional. Às vezes elas refletem uma aversão profunda a pessoas de outras sociedades. E é óbvio que elas prejudicam as pessoas de outras comunidades que desejam trabalhar, as impedindo de trabalhar mesmo quando há uma demanda genuína pelo seu trabalho. Mas as restrições imigratórias de um estado também prejudicam, obviamente, aqueles que já estão sob a jurisdição do estado e demandam serviços prestados por trabalhadores de outras comunidades. Assim, essas restrições monopolísticas -- que privilegiam algumas pessoas enquanto excluem outras -- reduzem o bem-estar econômico dos trabalhadores de outras comunidades \emph{e} dos próprios sujeitos do estado.

\subsection*{Licenças}

O estado sistematicamente exige que as pessoas não realizem diversos tipos de trabalho sem a sua permissão. A aplicação de requisitos de licenciamento é muitas vezes um exercício de mesquinharia tirânica, como quando um policial em Tulare, na Califórnia, fechou a barraca de limonada de uma garotinha porque ela não tinha uma licença comercial. Mas mesmo que os requisitos de licenciamento não gerem o mesmo tipo de indignação pública, eles constantemente prejudicam a capacidade das pessoas de trabalhar. Eles aumentam as barreiras de entrada e impulsionam os lucros das pessoas com licenças. Os operadores de táxi de Nova York, por exemplo, desfrutam de lucros de oligopólio devido às restrições de entrada no trabalho de motorista de táxi impostos pelos requisitos de licenciamento da cidade. Igualmente para os médicos: pense em quantos serviços médicos rotineiros poderiam ser realizados por enfermeiras, profissionais de enfermagem ou técnicos de vários tipos se requisitos de licenciamento não os impedissem. Como os médicos são os únicos que podem realizar esses serviços, eles podem exigir preços elevados dos pacientes. (Nunca deixo de me surpreender quando debates sobre a redução dos custos de serviços médicos deixam de focar nas contribuições para esses custos resultantes dos requisitos de licenciamento.) Advogados também muitas vezes podem cobrar taxas extremamente altas, porque o licenciamento limita a prestação de serviços jurídicos básicos por não advogados. Os requisitos de licenciamento criados e mantidos pelo estado criam monopólios que desviam riquezas para licenciados privilegiados.

Os requisitos de licenciamento tornam o trabalho mais caro porque adquirir ou manter uma licença pode exigir que alguém compre e mantenha equipamentos caros. Por exemplo: regras de ``saúde e segurança'' ou de zoneamento podem impedir alguém de operar uma pequena padaria na sua própria casa. Requisitos legais podem reduzir as padarias ``legítimas'' àquelas que empregam fornos grandes -- fornos economicamente viáveis de se usar apenas em uma grande padaria. Essas regras impedem alguém sem o dinheiro (ou tempo) para adquirir e operar uma instalação relativamente grande e longe da sua casa de operar uma padaria. Assim, eles desfavorecem \emph{startups} e pessoas sem dinheiro e, então, tendem a proteger produtores estabelecidos e pessoas com dinheiro contra a pressão exercida por iniciantes mais pobres.

\subsection*{Regulamentações em benefício das empresas}

Oficiais do estado muitas vezes afirmam que várias leis e regulamentações foram criadas para proteger o público dos interesses empresariais. E, sem dúvidas, proteger o público é realmente o objetivo de algumas das pessoas que apoiam regulamentações dos mais variados tipos. Mas a realidade é que regulamentações muitas vezes servem principalmente para reduzir a pressão que as grandes empresas muitas vezes são submetidas por empresas iniciantes. E isso continuará sendo verdade enquanto for o estado quem cria as regulamentações, já que órgãos legislatórios e reguladores podem facilmente ser dominados pelos interesses privados os quais eles foram projetados para regular. Empresas com fundos a mais no bolso têm dinheiro para gastar em \emph{lobby} e suborno. E é provável que todas as empresas estejam muito mais interessadas em focar nas leis e regulamentações relacionadas às suas operações do que quase qualquer outra pessoa. Assim, elas estarão muito mais bem equipadas para influenciar o desenvolvimento de tais leis e regulamentações do que a maioria do público geral ou, de fato, a maioria dos legisladores e suas equipes.

Além disso, seja qual for o conteúdo ou motivação das regulamentações, empresas grandes e estabelecidas provavelmente acharão mais fácil gastar o que é necessário para cumprir as novas regulamentações -- enquanto empresas iniciantes menores não. Então não é surpreendente que, quando leis e regulamentações que restringem as empresas são adotadas, elas frequentemente funcionam para o benefício das grandes empresas e em detrimento das pequenas.

Assim, as regulamentações da era progressiva vendidas ao público como projetadas para proteger os consumidores das grandes empresas na verdade serviram para proteger empresas da concorrência -- às custas dos consumidores. De maneira semelhante, os cartéis industriais criados no início da década de 30 ofereciam precisamente o que as grandes empresas vinham exigindo nas duas décadas anteriores. Os líderes da indústria tentaram impor regras a outras empresas que reduziriam a pressão que os \emph{players} estabelecidos sofriam dos negócios independentes e iniciantes. Enquanto essas regras eram voluntárias, muitas vezes eram ignoradas -- e preços mais baixos e maior responsividade beneficiavam o público. Então programas regulatórios foram projetados para transformar essas regras em requisitos legais, assim permitindo que os grandes \emph{players} continuassem produzindo bens e serviços com preços desnecessariamente altos, enquanto sujeitavam empresas que atendiam ao público de maneira mais eficiente a sanções destinadas a fazê-las se alinhar. O que se passa por regulamentação estatal de interesse público frequentemente prejudica o bem-estar público.

Um outro exemplo: regulamentações que ninguém consegue entender ou implementar sem adquirir conhecimento especializado e gastar muito dinheiro limitam efetivamente o número de pessoas que podem sequer pensar em produzir (por exemplo) dispositivos médicos. Certamente é caro projetar e construir tais dispositivos; mas muitas vezes é ainda mais caro cumprir as regulamentações relevantes. Elas aumentam significativamente os custos de entrada, o que significa que poucas pessoas podem sequer pensar em desenvolver novos produtos. Seu efeito, portanto, é o de proteger os \emph{players} existentes e um número limitado de novos \emph{players} com muito dinheiro.

\subsection*{Moeda}

O estado caracteristicamente insiste em monopolizar a emissão da moeda. Os anarquistas discordam até que ponto as pessoas precisariam da moeda em uma sociedade sem estado, mas uma convicção anarquista essencial é que, se uma atividade \emph{é} apropriada, o estado não deveria ser capaz de monopolizá-la. Ao monopolizar a emissão da moeda, o estado pode manipular a oferta monetária. Muitas vezes ele o faz deliberadamente -- como, por exemplo, quando paga suas dívidas relacionadas à guerra com moeda recentemente criada. Se essas dívidas forem pagas assim que a moeda for criada, os credores do estado podem obter um bom valor; com o tempo, no entanto, os efeitos do aumento da oferta monetária são sentidos em toda a economia, resultando na diminuição do valor da moeda existente. Sem anunciar um novo imposto, o estado efetivamente cometeu roubo como uma forma de financiar suas atividades de guerra.

Se diversas moedas estivessem em circulação, elas poderiam ser trocadas da mesma forma que moedas nacionais atualmente são. No entanto, como as pessoas poderiam escolher moedas fortes frente a moedas fracas, os efeitos da expansão monetária arbitrária seriam minimizados: as pessoas poderiam responder às mudanças nos valores de moedas específicas escolhendo fazer transações com outras, e haveria uma pressão contínua em busca de estabilidade sobre as várias entidades que emitiram moeda. Deixar os bancos livres para emitir sua própria moeda contra seus próprios ativos tenderia a garantir uma moeda forte, já que, se moedas pudessem circular e ser trocadas livremente, as taxas de câmbio mudariam à luz das mudanças no valor real dos ativos que lastreiam a moeda. Embora ocultar as informações relevantes pudesse ter sido relativamente fácil no passado, mecanismos de câmbio através da Internet poderiam tornar os câmbios entre as diversas moedas rápidos e relativamente baratos, e a possibilidade de câmbio tenderia a permitir que as pessoas evitem moedas ruins e optem pelas boas. Mas a oportunidade de transacionar em diversas moedas não está disponível. Como o estado mantém o monopólio da moeda, ele é capaz de exercer um poder enorme, e muitas vezes destrutivo, sobre a vida econômica das pessoas -- limitando suas opções enquanto as força a usar a moeda que ele cria e manipula em benefício das elites políticas.\footnoteA{Em uma região na qual diversas moedas circularam ao mesmo tempo, a prática comum pode convergir para um padrão de \emph{commodity} específica -- talvez um metal precioso, alguma outra \emph{commodity} específica ou uma cesta de \emph{commodities}. Eu não quero, ao ignorar essa possibilidade no texto, implicar qualquer juízo específico sobre um padrão de \emph{commodity}; só acredito que a maneira correta de descobrir se esse padrão faz sentido é experimentá-lo. Anarquismo é sobre experimentação antes de mais nada.}

\subsection*{Bancos e crédito}

A atividade bancária e a emissão de crédito também são, na prática, monopólios. O estado determina quais condições uma empresa precisa atender para se qualificar como um banco credenciado. Regras legais normalmente exigem que uma empresa tenha reservas de capital consideráveis para se qualificar como um banco. E muitas vezes permitem que um banco seja credenciado em uma dada área apenas se puder mostrar que ele é \emph{necessário} -- e se ele é necessário é avaliado em parte à luz de seu provável impacto nos lucros de \emph{outros} bancos. (Os outros bancos são frequentemente muito capazes de influenciar as decisões de credenciamento.)

Essas regras servem para proteger não o público, mas sim os bancos existentes. Se a capitalização e outras exigências fossem eliminadas, de forma que pequenos bancos pudessem facilmente ser fundados, a proliferação de bancos, cada um deles interessado em atrair depositantes, tenderia a reduzir as taxas de juros. Bancos com reservas limitadas poderiam oferecer alguns riscos para os depositantes, mas, desde que não houvesse fraude, os depositantes seriam livres para escolher opções de risco que fossem apropriadas para eles.

Bancos pequenos podem incluir bancos \emph{mútuos}. Quando elas participam de atividades bancárias mútuas, as pessoas usam seus próprios recursos como garantia em um empreendimento cooperativo que pode emitir moeda contra os recursos garantidos (de qualquer tipo -- terra, \emph{commodities}, etc.). Esse tipo de banco cooperativo dificilmente cobraria de seus membros juros acima do mínimo necessário para financiar as suas próprias operações: a disponibilidade de alternativas cuidaria disso.

Em contrapartida, como o estado impõe limites sobre quem pode oferecer serviços bancários e de crédito, o atual setor bancário desfruta de um monopólio coletivo \emph{de facto}. A cartelização do setor bancário permite que os bancos extraiam lucros de monopólio de um público indisposto. E devido às garantias do FDIC,\footnoteNT{N.T.: Abreviação de \emph{Federal Deposit Insurance Corporation}. Agência de seguro e garantia das contas bancárias americanas.} que na prática são subsídios, os bancos muitas vezes veem pouca necessidade de manter mais reservas do que o exigido por lei ou de encontrar maneiras criativas de atender às necessidades dos clientes.

\section{O estado subsidia os ricos e bem relacionados}

O estado preserva o poder e riqueza da classe exploradora -- não apenas criando cartéis e monopólios, mas também subsidiando atividades ineficientes de seus comparsas.

\subsection*{Tarifas}

Lembro-me de discutir sobre tarifas com meu pai quando era um aluno do ensino médio. Eu não sabia economia básica na época. Mas eu sabia que havia algo de errado em tratar bens e serviços de maneira diferente porque eles vieram de outros países. Isso era chauvinista, nacionalista, discriminatório.

Hoje eu também percebo o quanto as tarifas prejudicam as pessoas comuns no território de um estado que as impõe -- enquanto beneficia as elites. Tarifas são, na prática, subsídios do estado às indústrias e empresas favorecidas. Tarifas estatais não podem realmente excluir bens ou serviços de fora de suas fronteiras. Mas tarifas podem tornar esses bens e serviços muito menos atrativos para os compradores dentro de suas fronteiras. Ao fazer isso, ele impulsiona empresas ricas e bem relacionadas que não querem que produtores estrangeiros vendam mais barato que elas. Os produtores estrangeiros se tornam vítimas do preconceito nacionalista -- mas o mesmo é o caso daqueles sujeitos ao estado, que não podem obter bens e serviços tão baratos quanto poderiam de outra forma e que são forçados a subsidiar empresas privilegiadas. Um exemplo particularmente claro: subsídios agrícolas, que sustentam o agronegócio ineficiente às custas dos produtores agrícolas estrangeiros e que, portanto, constituem contínuas fontes significativas de pobreza ao redor do mundo.

\subsection*{Transporte}

Subsídios estatais afetam significativamente o transporte. Em suma, os subsídios estatais tornam o transporte de longa distância mais viável economicamente quando comparado ao transporte de curta distância, haja vista o que seria de outra forma. Investimento na construção de estradas é um exemplo óbvio. O uso da desapropriação para adquirir as terras nas quais rodovias e ferrovias podem ser construídas reduz seus custos aos seus usuários principais, que na prática são subsidiados. Rodovias são principalmente utilizadas por grandes empresas que transportam mercadorias por longas distâncias. Então subsídios que lhes permitem pagar menos do que sua cota justa nos custos das rodovias lhes dão uma vantagem injusta sobre as empresas que \emph{não} dependem do transporte de longa distância. O financiamento estatal de rodovias subsidia grandes empresas em específico -- e as encoraja a permanecerem grandes e se tornarem maiores.

\subsection*{Construção, imóveis e expansão urbana}

A capacidade de deduzir pagamentos de juros de hipoteca de imóveis da renda tributável é frequentemente concebida como uma forma de garantir que todos possam viver ``o sonho americano''. Mas, em termos práticos, equivale a um enorme subsídio às indústrias imobiliária e de construção -- um subsídio que aumenta os preços das casas (se eu posso deduzir pagamentos de juros, talvez eu esteja mais disposto a pagar mais por uma casa do que estaria de outra forma). Ao encorajar a compra de novas casas e, portanto, a construção de novas casas, o subsídio de hipoteca torna a expansão urbana mais provável (terras perto dos centros das cidades já estão ocupadas e muitas vezes são muito caras, então faz sentido instalar casas recém-construídas perto das periferias urbanas). O mesmo se aplica, é claro, a subsídios de transporte: se for mais barato dirigir, as pessoas estarão mais dispostas do que estariam de outra forma a se deslocar longas distâncias de e para o trabalho, e cidades-dormitórios surgirão para atendê-las.

\subsection*{Pesquisa e desenvolvimento}

Concessões estatais e outros tipos de subsídios forçam o público a pagar o custo das atividades de pesquisa e desenvolvimento das grandes empresas. Investir em pesquisa e desenvolvimento não só é caro, mas também arriscado -- não há garantias de que renderá algo de valor. Mas se a pesquisa básica que beneficia uma empresa pode ser realizada por um cientista universitário financiado por uma bolsa pública antes de entrar no domínio público, a empresa pode colher os benefícios enquanto arca apenas com uma fração do custo real. Ao mesmo tempo, dada a vasta quantia gasta em pesquisa e desenvolvimento pelo estado, as pessoas interessadas em aplicar os seus resultados podem estar menos dispostas a gastar dinheiro em pesquisa básica; direta e indiretamente, o estado pode expulsar as pessoas de seu trabalho ao tornar o que elas estão fazendo antieconômico.

\subsection*{Proteção contra responsabilidade por danos ambientais}

Um bom exemplo do problema das regulamentações: leis oficialmente caracterizadas como destinadas à proteção do meio ambiente às vezes na verdade impedem que os poluidores sejam processados pelos danos que causam (o mesmo acontece com, por exemplo, dispositivos médicos). Ao anunciar que, digamos, o governo federal dos EUA é responsável por cuidar do problema da poluição, a lei também pode estar dizendo aos moradores locais que talvez queiram entrar com processos judiciais: \emph{Não se meta}! Regras de preempção federal podem ser justificadas por garantirem regularidade e previsibilidade. Mas a sua função real pode muitas vezes ser a de proteger empresas das tentativas de responsabilizá-las por danos que não são reconhecidos pelas leis relevantes e que poderiam ser evitados por empresas mais cuidadosas dispostas a dedicar mais esforços para evitar prejudicar consumidores e o mundo não humano e para proteger a sua reputação com os consumidores.

\subsection*{Leis trabalhistas em benefício das empresas}

Embora o estado certamente conceda alguns privilégios aos sindicatos (em parte para contrabalancear os inúmeros privilégios que ele confere às empresas), pode-se dizer que as grandes empresas são beneficiárias da legislação trabalhista. Tal legislação é muitas vezes retratada como tendo limitado o poder dessas empresas. E, sem dúvida, alguns líderes empresariais não estavam muito felizes com várias características do \emph{National Labor Relations Act}, e o NLRA de fato exigia que empresas negociassem com os sindicatos em algumas circunstâncias nas quais elas antes não precisavam. Mas visto dentro do contexto, fica claro que o NLRA levou -- e era provavelmente pretendido por ao menos alguns de seus apoiadores a levar -- a uma redução nos conflitos de trabalho. Os sindicatos pré-NLRA, e não os legisladores, venceram as primeiras grandes batalhas na luta pela jornada de oito horas, por exemplo. Sindicatos dispostos a entrar em confrontos repetidamente saíram vitoriosos em disputas com empregadores. A atual estrutura da legislação trabalhista teve o efeito prático de \emph{limitar} as opções e oportunidades dos trabalhadores ao, por exemplo:

\begin{itemize}
\item proibir acordos \emph{union shop} e \emph{closed shop};
\item criar oportunidades para que empregadores influenciem os limites das unidades de negociação de maneiras a fortalecer suas posições;
\item eliminar boicotes secundários ou greves gerais ou em múltiplos locais;
\item exigir prazos de reflexão [\emph{cooling-off periods}];
\item eliminar agências sindicais de contratação [\emph{union hiring halls}];
\item obrigar a arbitragem de disputas trabalhistas;
\item proibir vários tipos de participação não sindical na gestão;
\item permitir o término de greves por decreto presidencial; e
\item exigir que negociações coletivas foquem em uma gama restrita de questões trabalhistas.
\end{itemize}

Essa estrutura apoiou -- de fato, cooptou -- sindicatos interessados em trabalhar dentro do atual conjunto de relações de poder. Mas tornou mais difícil para sindicatos como o \emph{Industrial Workers of the World} (os ``Wobblies'') a desafiar o poder dos empregadores. Sem o estado para inclinar a balança, sindicatos como o IWW poderiam usar o seu real poder de barganha para aumentar a influência dos trabalhadores sobre suas vidas no trabalho, sua segurança no emprego e seus níveis de remuneração.

\subsection*{Bailouts}

Uma lista de subsídios estatais para plutocratas não estaria completa sem referência aos \emph{bailouts} inimaginavelmente enormes distribuídos pelos governos Bush II e Obama. A maior quantia de dinheiro de \emph{bailout} foi indiscutivelmente para o bolso de \emph{Wall Street} -- morada de bancos de investimento irresponsáveis e corretoras que agora foram ensinadas uma lição valiosa: se você aposta com o dinheiro das pessoas e promove a especulação irresponsável (especulação irresponsável que é em primeiro lugar alimentada, é claro, pelo mau comportamento do Fed), você não precisa se preocupar -- um paraquedas de ouro, pago com despesas públicas, estará esperando por você. Montadoras de carro aprenderam uma lição semelhante: faça carros caros, ineficientes e poluentes que as pessoas não querem e o público será cobrado para assegurar as suas operações falhas. Ao mesmo tempo, os recursos alocados para montadoras ineficientes \emph{não estarão} disponíveis para usos com outros fins -- incluindo a criação de tecnologias de transporte mais eficientes e ambientalmente amigáveis.

\subsection*{Tomando terras para empresas privilegiadas}

Suponha que o estado tome terras para construir uma estrada. O fato de que a terra precisa ser tomada significa que o estado não estava preparado para pagar o que o proprietário queria; um agente não estatal não poderia ter obtido a terra ao preço que o estado estava preparado a pagar, e o dinheiro que o estado ``economizou'' equivale a um subsídio ao novo proprietário. E o fato de que pessoas comuns serão obrigadas a pagar pela terra sem ser considerado o quanto elas utilizam da estrada que é construída significa que aqueles que as utilizam muito, como empresas que transportam mercadorias, estão sendo subsidiados. A autoridade de desapropriação -- a autoridade que o estado alega ter para tomar as terras das pessoas -- deveria ser perturbadora em muitos contextos. Mas ações de desapropriação são especialmente preocupantes quando a terra é apropriada para ser transferida para o controle de uma empresa -- um empreiteiro, talvez, ou um grande varejista. Empreiteiros privados são, não surpreendentemente, grandes \emph{players} na política local, e é preciso pouco \emph{insight} para adivinhar se eles terão mais influência do que proprietários de imóvel sobre o que governos locais fazem. Quando empresas se beneficiam de vendas forçadas, pessoas comuns -- pessoas cujas casas são confiscadas, pessoas comuns cujos impostos pagam processos judiciais -- estão sendo obrigadas a subsidiar suas atividades.

\subsection*{Criando privilégios utilizando o sistema tributário}

Se o estado não tem autoridade legítima, então ninguém o deve nenhuma obrigação de apoiar suas atividades. Assim como, de fato, uma gangue de bandidos violentos, o estado não tem direito legítimo sobre os bens de ninguém. Então reduções de impostos devem parecer algo bom se você é anarquista. Quando o estado reduz impostos, ele está reduzindo o seu roubo do trabalho (apropriado quando o estado confisca seus frutos) e bens das pessoas. Mas o estado, apesar disso, age injustamente e promove a ineficiência quando impõe encargos injustos maiores a alguns, porque isentou os seus favoritos de algumas ou todas obrigações fiscais.\footnoteA{Meu foco aqui é a concessão de privilégios especiais a indivíduos ou empresas, ou a indústrias específicas. Se \emph{toda} renda corporativa fosse alocada aos acionistas para fins de tributação, por exemplo, provavelmente não haveria razão para considerar isso um subsídio às corporações.}

Suponha que um município deseje atrair um grande varejista. Ele pode tentar fazer isso oferecendo ao varejista um belo de um acordo no que tange os impostos sobre suas vendas. No papel, o acordo é, de fato, uma redução do tributo exigido pelo estado. Na realidade, embora não seja um subsídio, ele funciona exatamente como um: o estado escolhe um destinatário específico para conceder tratamento especial e melhora a sua posição econômica em relação a de todos os outros -- uma vez que outros varejistas não se beneficiam do acordo, eles estão em desvantagem quando comparados com o grande varejista. Esse tipo de privilégio incentiva o comportamento ineficiente -- se não fosse ineficiente, funcionaria sem o privilégio -- e oferece vantagens especiais para os politicamente influentes (enquanto também aumenta o poder do estado, uma que ele é a origem dos presentes a serem distribuídos para seus favoritos).

\subsection*{Força militar}

A contratação \emph{cost-plus} e a aparente necessidade de realizar os sonhos de todo planejador militar significam que os gastos militares nos Estados Unidos são enormes e representam um vasto subsídio direto para uma variedade de contratantes de defesa. Mas gastos com atividades militares também oferecem subsídios indiretos a diversas indústrias. O uso da força militar sustentada por impostos na América Central para facilitar a vida dos produtores de banana dos Estados Unidos é um exemplo óbvio de subsídio estatal a uma indústria específica que beneficia elites enquanto impõe custos significativos a trabalhadores locais e suas comunidades. Usar força militar para abrir acesso ao petróleo para empresas da elite é claramente outro exemplo.

\section{O estado ajuda a criar e preservar hierarquias}

Monopólios e subsídios estatais se apoderam da criatividade e autonomia das pessoas, incentivam alocações ineficientes de recursos e forçam as pessoas comuns a pagar tributos às elites politicamente conectadas. Além disso, incentivam a criação e manutenção de estruturas organizacionais que enfraquecem trabalhadores e prejudicam comunidades.

\subsection*{Hierarquias são ineficientes}

Organizações grandes e hierárquicas são ineficientes. Aumentos de tamanho podem certamente oferecer alguma economia de escala. Mas existem também \emph{des}economias de escala. Gerenciar informações, monitorar trabalhadores, transportar mercadorias por longas distâncias -- tudo necessário para o funcionamento efetivo de tais organizações -- é custoso. É tão custoso que, sem o apoio estatal, é provável que muitas grandes burocracias organizacionais autoritárias se mostrassem ineficientes demais para sobreviver.

Isso é verdade em parte pelos mesmos motivos básicos que o estado não pode gerir a produção e distribuição. Quer você esteja falando de uma enorme organização autoritária centralizada tentando tomar decisões para toda uma economia, ou de uma organização menor, mas ainda centralizada e autoritária, tentando tomar decisões para um segmento correspondentemente menor da economia, alguns dos problemas básicos são os mesmos. Informações, informações vitais, são distribuídas por toda a organização. Nenhuma pessoa ou unidade tem acesso a isso tudo. E quanto maior a organização, mais difícil será assimilar todas essas informações e menos informações cada indivíduo provavelmente terá. Além disso, em uma grande empresa, muitas vezes não haverá uma maneira significativa de determinar os preços dos bens e serviços que diferentes unidades fornecem umas às outras, já que as diferentes unidades não são realmente proprietárias desses bens e serviços e terão uma capacidade limitada de precificá-los de maneira razoável. A presença de informações distribuídas e as dificuldades com as precificações internas limitam a eficiência das grandes empresas.

Mas essas empresas não precisam enfrentar os reais custos de ser grandes porque o estado as subsidia (algumas mais do que outras, é claro). De maneira muito intensa.

\subsection*{O formato corporativo é parcialmente um subsídio}

Comece pelo próprio formato corporativo. Pelo menos em parte, é algo que não existiria sem a ação do estado, que oferece diversos benefícios para aqueles que são capazes de tirar proveito de tal coisa -- notavelmente, da responsabilidade limitada. A responsabilidade limitada protege os proprietários de uma corporação contra ações judiciais. Isso significa, de maneira geral, que um processo judicial bem-sucedido contra uma corporação pode atingir apenas os ativos da empresa, e não os ativos dos investidores que muitas vezes são legal e socialmente identificados como proprietários da corporação.

No direito contemporâneo, a responsabilidade limitada se apresenta em duas formas básicas: responsabilidade limitada por contrato e responsabilidade limitada por \emph{tort}. Danos \emph{contratuais} são concedidos à luz dos acordos que as pessoas fizeram entre si -- como, por exemplo, quando alguém deixa de cumprir um compromisso de entregar bens prometidos e causa um prejuízo de negócios como resultado. Por outro lado, uma pessoa é responsável por \emph{tort} quando prejudica os interesses de alguém, seja qual for a sua relação com a pessoa ou entidade responsável.

É fácil ver como uma empresa poderia obter responsabilidade limitada por contrato sem qualquer ação do estado. Ao fazer um acordo com uma pessoa ou outra empresa, ela poderia simplesmente garantir que o acordo incluísse termos limitando a responsabilidade para os ativos da própria empresa no caso do acordo ser violado. Não seria necessário um \emph{status} corporativo para fazer isso; poderia ser uma empresa individual ou uma parceria. No entanto, o custo de garantir a proteção da responsabilidade limitada pode ser maior do que atualmente é, especialmente se a ausência do formato corporativo exigir que uma empresa negocie a responsabilidade limitada caso-a-caso.

Independentemente do que seja verdade para a responsabilidade limitada por contrato, sem a proteção conferida pelo \emph{status} corporativo criado pelo estado, não é de todo óbvio que uma empresa poderia desfrutar da proteção de responsabilidade limitada no que diz respeito a ações de \emph{tort}. Suponha que uma empresa seja responsável por um derramamento de substância química tóxica que resulta em uma epidemia de câncer que assola uma pequena cidade e que nenhum dos residentes dela tinha qualquer relação especial com a empresa. Se a empresa for uma empresa individual, uma ação judicial contra ela poderia atingir os ativos do proprietário. Se for uma parceria normal, uma ação judicial poderia atingir o patrimônio dos sócios. Nos dois casos, os proprietários legais seriam responsáveis pelas dívidas da empresa. Mas uma ação judicial contra uma empresa \emph{incorporada}\footnoteNT{N.T.: \emph{incorporated}. É comumente tido como abreviado por ``Inc.'', partícula que se nota no nome de grandes corporações como ``Apple Inc.'' e ``Facebook, Inc.''.} poderia atingir apenas os ativos \emph{da própria empresa}.

Sem a proteção da responsabilidade limitada que a incorporação inclui, os acionistas -- que atualmente são tratados como proprietários legais de uma corporação -- podem ser responsáveis por dívidas resultantes de \emph{torts} cometidos pela corporação. Se seriam ou não responsabilizados dependeria de quanto controle eles detinham junto da sua suposta propriedade legal, que tipo de influência eles exerciam sobre as pessoas diretamente responsáveis pelos \emph{torts} e que tipo de influência eles \emph{poderiam} ter exercido. Independentemente do que seja verdade para cada acionista em específico, obviamente há um argumento muito forte a favor da responsabilidade ilimitada para os diretores, que são responsáveis por supervisionar as atividades dos executivos. Mas certamente haveria momentos em que pelo menos alguns acionistas (especialmente investidores institucionais com influência significativa sobre os diretores) também seriam responsabilizados por \emph{torts} corporativos se a responsabilidade limitada fornecida pelo estado não os protegesse.

A exposição a esse tipo de responsabilidade poderia tornar as pessoas mais cautelosas quanto às decisões de investimento. Elas poderiam ser mais propensas a limitar o seu apoio a projetos que consideram de baixo risco de responsabilidade. Se não fizessem isso, elas precisariam tomar medidas para se proteger caso tribunais concedessem indenizações significativas de \emph{tort} contra empresas que elas haviam investido. (Presumivelmente, elas comprariam significativamente mais seguros do que atualmente; muitos provavelmente também fariam vários tipos de acordos de indenização.) Decisões corporativas irresponsáveis poderiam ser menos prováveis de ocorrer. Ao fornecer proteção de responsabilidade limitada a uma corporação, o estado está subsidiando comportamentos arriscados sem restrições -- o que os economistas costumam chamar de ``risco moral''. Ao mesmo tempo, isso é na prática roubar das pessoas que são incapazes, por conta da proteção de responsabilidade limitada, de obter reparação pelos danos que sofreram devido ao mau comportamento corporativo aprovado ou promovido por investidores. 

Além disso, o privilégio de incorporação concedido pelo estado transforma a corporação em uma entidade que é supostamente distinta dos acionistas e trabalhadores (administrativos e outros). Os investidores recebem dividendos -- às vezes -- das corporações cujas ações eles possuem, mas seu envolvimento real na governança corporativa é frequentemente limitado (de maneiras que certamente podem algumas vezes justificar a responsabilidade limitada). Isso não se dá apenas porque a responsabilidade limitada pode reduzir ou eliminar os incentivos para que um investidor permaneça ativamente envolvido nos assuntos de uma corporação na qual investe; é também porque a legislação corporativa moderna incentiva o investidor a ver o seu papel como limitado. Embora a fantasia de que administradores trabalham para os investidores seja repetida e, às vezes, enfatizada, as condições econômicas e as regras jurídicas na verdade dão aos administradores independência suficiente para que eles possam agir em nome da ``corporação'' em vez dos acionistas. Algumas vezes isso significa agir em nome da corporação como uma comunidade composta por diversos constituintes; mais frequentemente, significa identificar os interesses da corporação com os dos executivos seniores. Se eles escolherem fazê-lo, como é de costume, esses executivos podem simultaneamente apelar para a existência independente da corporação para justificar a sua independência relativa dos acionistas, enquanto apontam para a sua suposta responsabilidade para com os acionistas para justificar a negligência às preocupações legítimas dos trabalhadores, membros da comunidade local e outros. Dessa forma, também, o estado apoia a existência de grandes burocracias corporativas.

\subsection*{Requisitos de licenciamento protegem empresas privilegiadas}

Requisitos de licenciamento e muitos outros tipos de regulamentações limitam o tipo de trabalho que as pessoas podem realizar: grandes empresas conseguem cobrir os custos de manter a conformidade com as leis que \emph{startups} menores geralmente não conseguem. A legislação tributária recompensa arbitrariamente essas empresas por fazerem coisas -- como reinvestir o dinheiro que ganham em vez de distribuí-lo aos acionistas na forma de dividendos, fundir com outras empresas ou focar em pesquisas caras que levam a grandes projetos -- que as tornam ou mantêm grandes. A legislação trabalhista permite que empresas cooptem sindicatos enquanto reduz a pressão de táticas trabalhistas radicais. Financiando atividades de pesquisa e desenvolvimento, o estado torna possível o engrandecimento das empresas que dependem dessas atividades. E patentes, direitos autorais e subsídios de transporte também todos facilitam o funcionamento de grandes organizações hierárquicas.

\subsection*{Patentes e direitos autorais protegem empresas privilegiadas}

Patentes e direitos autorais concentram riquezas: empresas endinheiradas podem comprar direitos de patente e utilizá-los como base para os seus próprios processos e produtos quando outras empresas não podem. E a expectativa de retorno de patentes bem-sucedidas (e da proteção contra pessoas que não podem processar pedidos de patente e garantir suas reivindicações de patente no tribunal) cria incentivos para a criação de organizações ricas capazes de financiar o processo de longo prazo que leva à aquisição de patentes, algo que organizações com menos persistência obviamente acham mais difícil de se fazer. Patentes têm sido muito frequentemente utilizadas para impedir que pessoas entreguem determinados bens ou serviços e para desligar linhas de pesquisa e desenvolvimento que de outra forma seriam frutíferas e economicamente viáveis. E o licenciamento de patentes e as trocas de patentes também ajudaram empresas a criar cartéis com o potencial de esmagar outras empresas e explorar o público.

\subsection*{Tarifas protegem empresas privilegiadas}

O subsídio fornecido às empresas por meio de tarifas abusa dos clientes que são forçados a pagar preços mais altos. Mas também faz com que seja mais fácil que empresas se tornem e permaneçam grandes. Ao isolá-las contra pressões que de outra forma seriam exercidas por trabalhadores e empresas em outras comunidades, tarifas e barreiras semelhantes permitem que as empresas que se beneficiam delas se tornem grandes e preguiçosas, reduzindo assim a necessidade delas de adotar estruturas organizacionais mais eficientes.

\subsection*{Gastos rodoviários subsidiam o tamanho de empresas}

Quando o estado financia a construção e manutenção de rodovias com receitas públicas gerais, ele subsidia todo o trânsito rodoviário; mas mesmo quando ele usa fundos de impostos sobre combustíveis para apoiar projetos de rodovias, ele subsidia empresas que dependem de caminhões para transporte de longa distância se ele estruturar os impostos de forma que caminhões pesados paguem menos do que a sua parcela nos custos de construção e manutenção. De maneira semelhante, quando ele faz uso da desapropriação para adquirir terras para aeroportos e outros itens de infraestrutura de tráfego aéreo, ele os permite serem disponibilizados para empresas que dependem direta ou indiretamente deles a um custo mais baixo do que essas empresas teriam que de outra forma pagar para adquiri-los. Subsídios de transporte tornam a operação de grandes empresas com múltiplos locais mais econômica. Tais empresas podem enviar mercadorias em longas distâncias. Se (dada a existência do transporte subsidiado) economicamente faz sentido que elas façam isso em vez de produzir e distribuir localmente, então parecerá mais razoável para elas criar burocracias de grande escala para administrar suas operações em múltiplos locais.

\subsection*{Acabar com privilégios significa reduzir hierarquias}

Privilégios concedidos pelo estado, incluindo aqueles que efetivamente subsidiam modelos ineficientes de negócios, promovem hierarquias. Eliminar esses privilégios poderia, portanto, reduzir significativamente as hierarquias em locais de trabalho.

Organizações grandes são ineficientes. Elas também não são responsivas e têm maior probabilidade de serem desumanas quando comparadas a organizações menores. Não existe local de trabalho perfeito; mas locais de trabalho sem gerentes provavelmente são muito mais atrativos em alguns aspectos importantes do que locais de trabalho \emph{com} gerentes: ser tratado como um subordinado a sem autoridade suficiente para ajudar a moldar os objetivos organizacionais e a maneira como você realiza o seu próprio trabalho pode ser bem frustrante e humilhante. O argumento mais comum a favor da visão de que hierarquias dominadas por pessoas com MBAs são de alguma forma preferíveis a parcerias e cooperativas é que grandes organizações são complexas de maneiras que exigem a liderança supostamente especializada de gerentes treinados. Mas quanto menor a organização, menos razões pode haver para uma estrutura que distingue entre trabalhadores que produzem bens e serviços para os outros e gerentes que supervisionam os trabalhadores. Sem subsídios estatais, então, a justificativa para a dominação por indivíduos com MBAs começa a se desfazer.

Sem os subsídios diretos e indiretos fornecidos pelo estado às grandes empresas, seria significativamente mais difícil para elas serem grandes. E o argumento a favor da burocracia administrativa em oposição às estruturas de gestão participativa (ou aquelas que permitem que trabalhadores governem a si mesmos em cooperativas) é substancialmente enfraquecido à medida que o tamanho da empresa aumenta. A ação do estado facilita a persistência de burocracias organizacionais centralizadas que enfraquecem os trabalhadores.

Hierarquias não apenas são ineficientes -- elas são desagradáveis. A maioria das pessoas não gosta de trabalhar em ambientes organizacionais em que elas são tratadas como engrenagens impessoais de enormes máquinas, nas quais lhes são negadas oportunidades significativas de participar na tomada de decisões e de fazer uso eficaz da sua experiência prática. As pessoas preferem tomar as suas próprias decisões e a maioria prefere não lidar com insultos e humilhações mesquinhas que tornam muitos locais de trabalho infernais. Suspeito, portanto, que a maioria das pessoas optariam por trabalhar em parcerias ou cooperativas ou como contratantes independentes, em vez de em burocracias hierárquicas, se elas pudessem fazê-lo.

Então por que mais pessoas não trabalham por conta própria, ou em parcerias ou cooperativas? Há duas razões óbvias. Os custos associados a abrir uma empresa são muitas vezes altos. Além disso, trabalhar para outra pessoa pode significar evitar alguns riscos financeiros que as pessoas que trabalham para si mesmas têm de enfrentar.

Sem o estado, as barreiras para abrir uma pequena nova empresa, seja cooperativa ou independente, seriam menores. Requisitos de licenciamento não serviriam como barreiras de entrada em mercados específicos. Regras de zoneamento não impediriam as pessoas de trabalhar das suas próprias casas. Os vários tipos de subsídios para o tamanho organizacional que já discuti não estariam mais disponíveis, então seria mais difícil que grandes empresas expulsassem as pequenas. Grandes empresas existentes teriam que pagar o preço de seu tamanho em vez de repassar os custos para as pessoas comuns por meio do sistema tributário, de maneira que seria mais difícil para elas expulsar novos entrantes no mercado. Ao mesmo tempo, seria mais fácil obter capital para \emph{startup}. Por exemplo, na ausência de regras bancárias governamentais, as pessoas poderiam criar bancos mútuos e economizar dinheiro com custos administrativos e relacionados. Então abrir um novo negócio seria uma proposta menos assustadora do que atualmente é.

E sem a interferência do estado, como sugiro na \hyperref[sec:1]{próxima seção}, o custo de vida para as pessoas comuns seria menor, assim como os custos de abrir uma nova empresa para substituir uma falida, então os riscos associados ao desemprego também seriam menores. Sem códigos de construção e regulamentações de zoneamento, moradias seriam mais baratas e locais de trabalho fora-de-casa poderiam ser localizados mais perto das residências das pessoas. Sem tarifas e ``propriedade intelectual'', os bens de consumo seriam mais baratos. Sem regulamentações e subsídios corporativos, recursos seriam gastos com mais eficiência e os preços seriam mais baixos. Sem impostos, as pessoas teriam mais renda disponível.

Em suma, sem o estado, as pessoas achariam mais fácil abrir empresas. E com custos de vida mais baixos, seria mais fácil poupar para tempos difíceis e mais fácil juntar os cacos caso as coisas não corressem bem, então assumir os riscos associados a abrir uma empresa poderia ser menos estressante. 

E é difícil não pensar que isso colocaria uma pressão indireta sobre os gigantes hierárquicos para eles mudarem a maneira que operam. Sem a ajuda do estado para sustentar as suas abordagens antiquadas e autoritárias de produção e distribuição, seria mais difícil para eles manterem trabalhadores que estão plenamente cientes das possibilidades alternativas. Se os trabalhadores soubessem que poderiam criar locais de trabalho mais habitáveis por conta própria, eles seriam capazes de negociar por melhores salários e condições de trabalho: com mais opções, eles teriam um poder de barganha consideravelmente maior do que atualmente. Haveria boa razão para os gigantes empresariais pararem de mandar nos trabalhadores e começarem a criar ambientes de trabalho mais autônomos.

Obviamente, em vez de reduzir o tempo que gastam trabalhando para outros, algumas pessoas podem preferir deixar que outros carreguem os riscos até reduzidos que enfrentariam na ausência do estado e se concentrar em aumentar a sua renda e seus estoques de bens de consumo. Para muitas pessoas, porém, o objetivo de longo prazo sem dúvidas ainda seria o de deixar para trás as hierarquias do local de trabalho, e quanto mais pessoas fizessem isso, mais provável seria que essas hierarquias se tornassem parte de um mundo há muito abandonado. Sem o estado, seria muito mais fácil para as pessoas tornarem os seus ambientes de trabalho humanos e convidativos.

\section{O estado torna as pessoas pobres}
\label{sec:1}

Muitos fatores diferentes podem criar ou agravar a pobreza. Mas apesar da retórica que frequentemente se ouve de seus defensores, o estado desempenha um papel vital em tornar e manter as pessoas pobres.

\subsection*{Uma história do roubo}

O estado participa e sanciona a redistribuição forçada da riqueza das pessoas comuns para as elites privilegiadas. Por exemplo: em uma época em que a terra era a principal fonte de riqueza, o governo britânico impôs regras em que terras anteriormente compartilhadas pelos residentes de comunidades específicas não seriam mais acessíveis a todos eles, mas, em vez disso, seriam alocadas como propriedade pessoal de membros de grupos de elite locais. Portanto, ele privou as pessoas do acesso a recursos vitais e deixou muitas pessoas com poucas opções a não ser trabalhar em fábricas com ambientes perigosos e opressivos. E isso não foi coincidência: membros da elite britânica deixaram claro que era importante para eles que as pessoas comuns não tivessem a opção de ser ``preguiçosas'' -- elas tinham que ser forçadas a servir na economia industrial florescente. O governo também impôs regras que limitavam a quantidade de pessoas comuns que podiam viajar em busca de trabalho, assim as forçando a aceitar opções desfavoráveis mais próximas de suas casas, mesmo que pudessem obter um trabalho melhor mais longe. Naturalmente, esse arranjo tendia a manter as pessoas comuns pobres; enquanto isso, ao manter baixas as rendas dos trabalhadores, e portanto os custos trabalhistas dos empregadores, ele aumentava os lucros dos membros da elite que empregavam pessoas pobres -- que eram impedidas por lei de obter um trabalho melhor.

Esse dificilmente foi o único período em que funcionários do governo conspiraram contra as pessoas comuns de maneiras que tornaram a pobreza mais provável. Outro exemplo óbvio: quando a Grã-Bretanha colonizou a América do Norte, o governo britânico simplesmente reivindicou vastas extensões de terra, em virtude da suposta autoridade da Coroa, e as distribuiu para os politicamente bem relacionados. Terras que poderiam ter permitido que pessoas comuns -- tanto índios americanos quanto colonos europeus -- sobrevivessem e prosperassem, foram em vez disso concentradas, por meio do exercício descarado do poder, nas mãos de poucos. De maneira semelhante, o poder do governo criou vastas propriedades agrícolas na América Latina: terras que deveriam ter pertencido aos camponeses que nelas viviam e trabalhavam foram tratadas como propriedade de colonos politicamente conectados, enquanto os camponeses em si tornaram-se quase escravos. Governos ao redor do mundo deram a sua bênção à escravidão.

O roubo de terras em larga escala, a violenta criação de classes inferiores de camponeses e escravos e injustiças semelhantes são tão antigas quanto o próprio estado (e talvez mais antigas). Sancionadas e muitas vezes efetuadas pela força esmagadora à disposição do estado, essas injustiças contribuíram drasticamente para a distribuição da riqueza no mundo atual. Riqueza e pobreza são autoperpetuantes, e a privação efetuada no passado pelo estado das terras e do acesso aos recursos das pessoas comuns, e até das suas próprias vidas, tem ramificações contínuas: uma vez despojadas, as pessoas têm mais dificuldade para se reerguer e obter segurança e conforto reais. Ao mesmo tempo, com os privilégios conferidos a eles pelo estado, os membros da elite acham mais fácil manter as suas posições econômicas -- e influenciar agentes estatais de maneiras que os levarão a obter ainda mais privilégios.

\subsection*{Regras que empobrecem}

O estado patrocinou e endossou o despejo das pessoas de suas terras. Desapropriar as pessoas teve consequências terríveis, eliminando a independência delas e as forçando a empregos inseguros e modestos. A desapropriação e suas consequências definitivamente ajudaram a tornar e manter as pessoas pobres. As regras que o estado promulga e aplica muitas vezes têm o mesmo efeito. Elas são manipuladas a favor de grupos e indivíduos ricos. Elas são manipuladas a favor dos valores, normas e preferências das classes média e alta. E então elas tendem a excluir e empobrecer as pessoas que não estão à altura.

Leis de licenciamento ocupacional são um culpado óbvio. Tais leis podem impedir uma pessoa pobre de conduzir um serviço de táxi barato e não licenciado -- provavelmente porque ela não possui uma licença incrivelmente cara, parte de um sistema monopolista que busca aumentar a renda dos motoristas de táxi e empresas de táxi existentes. Podem puni-la se ela fornecer serviços de cuidado de cabelo sem obter uma licença cara -- mesmo que o custo inicial para obter tal licença, tanto em dinheiro quanto em tempo, possa excluí-la de trabalhar na ocupação escolhida, e mesmo que seus clientes possam estar perfeitamente satisfeitos em pagar pelos seus serviços. Uma variedade de regras a impedirá de vender medicamentos fora de uma farmácia licenciada. Toda uma gama de requisitos comerciais e outros requisitos de licenciamento podem impedi-la de abrir um \emph{food truck} na esquina de uma rua movimentada -- requisitos que, novamente, funcionam especialmente como proteção para pessoas fazendo o tipo de trabalho que desejam fazer e que desejam impedir novatos de fazer o mesmo tipo de trabalho que elas estão fazendo. Ao impedir que as pessoas usem as instalações de baixo custo que são as suas próprias casas para fins comerciais, os requisitos de licenciamento ajudam a encurralar as pessoas à condição de trabalhar para terceiros por remuneração e a limitar as suas capacidades de escapar da pobreza.

Regras de licenciamento podem simplesmente excluir algumas pessoas -- como aquelas que não querem pagar os custos exorbitantes dos licenciamentos de táxi de Nova York -- de realizar alguns tipos de trabalho legalmente: elas podem colocar limites absolutos no número de pessoas que podem oferecer certos tipos de bens ou serviços em determinadas regiões. Mas mesmo quando não o fazem, elas podem na prática ter o mesmo efeito. Quando, por exemplo, elas aumentam os custos de entrada e penalizam o trabalho em pequena escala ao exigir equipamentos, instalações ou treinamento caros, elas impedem que pessoas pobres que poderiam trabalhar para si mesmas o façam, as forçando a permanecer pobres -- e provavelmente dependentes das humilhações associadas à condição de trabalhar para terceiros por baixos salários.

Sem dúvidas, muitas das pessoas que colocam requisitos como esse são bem intencionadas. (As elites nem sempre são bem intencionadas, no entanto: por exemplo, durante a Revolução Industrial, alguns membros da elite inglesa disseram claramente que apoiavam regulamentações que na prática negariam aos pobres a opção de trabalhar por conta própria, para que eles fossem obrigados a aceitar o trabalho em fábricas que seria de outra forma indesejável.) Talvez burocratas, reguladores e legisladores queiram tornar a vida econômica mais ordenada. Talvez eles acreditem ingenuamente que estão ajudando os pobres. Ou talvez eles apenas queiram arrecadar dinheiro para um governo municipal. Mesmo quando os motivos dos reguladores são benignos, porém, o real efeito de suas escolhas muitas vezes é o de pressionar as pessoas a ingressar em ambientes de trabalho que carecem da liberdade e flexibilidade de trabalhar por conta própria.

Quaisquer que sejam os objetivos daqueles que os colocam em prática, esses mandatos estatais tornam mais difícil para os pobres -- pessoas que não têm muito capital de \emph{startup}, pessoas que não são bem relacionadas -- gerarem renda e adquirirem experiência de trabalho (especialmente ao trabalharem para si mesmos, o que, apesar de todos os riscos, pode ser muito mais livre e gratificante do que trabalhar para outra pessoa em um emprego vulnerável e de baixa remuneração). Isso tudo é bem típico do que o estado sistematicamente faz: a legislação e regulamentação garantem que manter o seu padrão de vida trabalhando para si mesmo custe mais do que custaria sem a interferência do estado. Por conta do que o estado faz, fica mais difícil evitar trabalhar para outras pessoas e se submeter às imposições hierárquicas e tiranias mesquinhas do local de trabalho.

O problema não são apenas as restrições que os requisitos de licenciamento ocupacional impõem aos pobres. Pense na enorme atenção que os governos costumam dar ao uso da terra. Ao controlar o que e onde algo pode ser construído, regras de zoneamento e códigos de construção aumentam os custos da construção comercial e residencial. Custos mais altos fazem com que seja mais difícil que as pessoas encontrem moradias a preços acessíveis. Tanto porque as pessoas têm menos renda para gastar se tiverem que usar muito de seu dinheiro para pagar por moradia, quanto porque o preço dos imóveis comerciais é mantido artificialmente alto, esses custos também fazem com que seja mais difícil que os pobres realizem trabalhos que exijam acesso a espaço comercial. É claro, leis de zoneamento muitas vezes os impedem de usar as suas casas como espaço comercial e, muitas vezes, de compartilhar casas com grandes grupos de amigos apenas porque não são parentes biológicos desses amigos.

Como sempre, é uma boa ideia perguntar, \emph{Quem se beneficia}? Acredito que esteja bem claro que os principais beneficiários de tais leis são as pessoas que constroem e vendem imóveis. Enquanto os preços em seus setores forem artificialmente elevados, seus bolsos serão confortavelmente acolchoados. É claro que algumas dessas regras também refletem os desejos das pessoas de classe média que possuem casas, ou alugam casas ou apartamentos para pessoas pobres, de ter uma renda de aluguel e revenda artificialmente elevada. Os incorporadores imobiliários e seus parasitas costumam ser as pessoas mais poderosas de uma cidade ou distrito, e muitos proprietários de imóveis são suscetíveis a se provarem aliados em discurso dos incorporadores. Então não deveria ser muito surpreendente que as suas visões tenham mais peso com as autoridades governamentais locais do que as dos pobres (ou, de fato, de outras pessoas em busca de moradia e espaço comercial baratos). Não é surpreendente que o padrão seja familiar: organizações estabelecidas com dinheiro e poder promovem governos, o público e talvez até elas mesmas com base na ideia de que medidas que injustamente encherão os seus próprios bolsos são na verdade benéficas a \emph{todos}. Se não houvesse aparato estatal para que esses plutocratas egoístas dominassem, esse tipo de traição dos interesses dos pobres nem sequer seria possível.

Os lucros que leis e regulamentações geram para incorporadores e proprietários residenciais existentes são as principais razões pelas quais essas restrições existem, suspeito eu. Mas também por trás disso, acredito eu, está um desejo reflexivo de impor a conformidade: \emph{Aquela quantidade de pessoas simplesmente} não deveriam \emph{viver juntas}! Eu \emph{não gostaria de colocar} os pés \emph{em um salão} desse tipo! A suposição de que legisladores e burocratas de classe média julgam melhor o que é bom para as pessoas do que elas próprias serve como uma desculpa conveniente para exigir a conformidade -- mesmo que, ao longo do caminho, isso também aumente a pobreza e a inconveniência.

Suponha que alguém queira consertar carros em casa ou reduzir custos de alimentação cultivando uma horta em seu quintal. Vizinhos bisbilhoteiros podem muitas vezes insistir que essa pessoa seja multada por fazer isso: afinal, potenciais compradores podem não estar dispostos a pagar tanto por suas casas se perceberem que não poderão evitar a vista de um carro quebrado ou de uma horta florida. Há algo de bizarro nesse tipo de raciocínio em qualquer contexto. Afinal, todo tipo de coisa pode influenciar o interesse de alguém de comprar uma determinada casa. A inauguração de uma nova fábrica ou de uma nova escola pode levar as pessoas a se mudarem de uma parte da cidade para outra, por exemplo, e tornar o trabalho em outras partes da cidade menos atrativo do que poderia ter sido anteriormente. Alguém realmente acredita que a nova fábrica ou escola deve ser obrigada a compensar as pessoas que não conseguem vender as suas casas pelo valor que poderiam ter vendido se a escola ou fábrica não fosse aberta? Mas o fascismo de valor de propriedade se torna muito mais do que simplesmente bobo quando é usado para limitar as opções das pessoas sem muito dinheiro. Regras que na prática exigem que pessoas não se envolvam em certos tipos de trabalho em casa, ou que obrigam pessoas a frequentar mercados caros em vez de produzir alimentos em casa, ajudam a manter os pobres, pobres e dependentes.

Claro, algumas pessoas não podem pagar por moradia e, então, podem se tornar sem-teto. Um sem-teto pode ser constantemente assediado pelo estado enquanto tenta simplesmente tirar uma soneca no carro ou no banco de um parque. Quando sem-tetos se apropriam de terras abandonadas, eles podem ser repetidamente expulsos por policiais, muitas vezes a mando de pessoas que simplesmente não gostam muito da sua aparência ou cheiro.

No sul de Jim Crow e na época do \emph{apartheid} na África do Sul, o estado desempenhou um papel fundamental para impedir que brancos pagassem pelo trabalho de negros e fornecessem serviços a eles nas mesmas condições que os brancos. Esse tipo de racismo imposto pelo estado obviamente contribuiu para a pobreza dos negros em ambos os ambientes. De maneira semelhante, na América moderna autoridades governamentais contribuem para a pobreza ao impedir que pessoas sem os documentos de imigração ``adequados'' trabalhem -- ou então permitindo que trabalhem, enquanto utilizam da ameaça da deportação para mantê-las cooperativas com os empregadores e com o próprio estado. Esse tipo de racismo aumenta a pobreza, tanto ao negar trabalho a pessoas que podem, como resultado, não ter renda nenhuma, quanto ao garantir que, quando as pessoas sem os documentos certos \emph{são} capazes de trabalhar, elas serão desencorajadas a bater de frente [\emph{making waves}] no emprego. Afinal, se elas fizerem isso, correrão o risco de chamar a atenção das autoridades, que provavelmente as deportarão e poderão até mesmo prendê-las. 

O estado também contribui indiretamente para a pobreza: ao limitar as opções até mesmo das pessoas que têm os documentos certos, ele na prática não deixa nenhuma alternativa razoável a não ser aceitar trabalhos em condições indesejáveis. Pessoas com a opção de trabalhar por conta própria, pessoas que não precisam gastar a maior parte de seu dinheiro em imóveis, podem negociar melhores condições no trabalho. Elas são muito mais livres para abandonar empregos opressivos, inseguros e de baixa remuneração. Pessoas com poucas ou nenhuma opção, em contrapartida, não terão escolha razoável a não ser aceitar quaisquer termos que lhes sejam oferecidos.

Já apontei a importância das extensas apreensões de terras para explicar a atual distribuição de riquezas e poder. As apropriações de terras por pessoas bem relacionadas não são somente má lembranças do passado: em muitos lugares ao redor do mundo nos dias de hoje, o estado cria pobreza e a agrava reivindicando arbitrariamente terras sem dono ou diretamente se apropriando de terras de camponeses e indígenas e as dando às elites políticas e seus comparsas, ou colocando selos legais de aprovação em confiscos violentos de terras executados diretamente pelas elites.

Impostos sobre produto e semelhantes impactam até mesmo -- e especialmente -- os mais pobres. E impostos de renda estaduais e federais certamente reduzem os recursos disponíveis para os trabalhadores americanos pobres e de colarinho azul. Tarifas também prejudicam os pobres -- ao aumentar significativamente os custos que eles precisam pagar por bens importados (incluindo, com grande frequência, alimentos necessários para uma boa saúde que, sem taxas de importação, seriam mais baratos do que as alternativas nacionais). Embora muitas vezes sejam elogiadas por sustentar a renda dos trabalhadores pobres, elas servem principalmente para aumentar os lucros dos produtores nacionais incompetentes às custas tanto dos consumidores domésticos (especialmente os pobres), quanto dos produtores estrangeiros.

Privilégios garantidos politicamente são responsáveis de diversas maneiras pelos lucros obtidos por muitas grandes empresas. Em um ambiente em que isso ocorre, a sindicalização pode ajudar a melhorar a posição econômica dos trabalhadores. Limitações legalmente impostas à atividade sindical podem tender a reduzir a influência dos sindicatos e, assim, a reduzir a renda de trabalhadores que poderiam receber mais se fossem livres para utilizar táticas de negociação mais radicais.

O estado limita o acesso ao trabalho. Ele limita o acesso à moradia. Ele tenta forçar as pessoas a aderirem ao molde genérico da classe média. Algumas vezes, ele diretamente (como no caso de terras) distribui ou redistribui riquezas para as elites. Ele interfere na capacidade das pessoas de se protegerem ao se organizarem. Ao fazer isso, ele cria e agrava a pobreza.

\section{Privilégios limitam o acesso à saúde}

O estado acentua a vulnerabilidade econômica daqueles que estão fora da esfera social e política, enquanto aumenta a segurança daqueles que estão dentro dela. Um exemplo particularmente bom: privilégios garantidos pelo estado dificultam o acesso das pessoas à saúde decente e acessível. O grau de acessibilidade dos serviços de saúde é determinado por fatores \emph{estruturais e políticos}. É muitas vezes resultado da maneira que aqueles politicamente bem relacionados fazem uso do seu acesso ao poder estatal.

\begin{itemize}
\item Empresas farmacêuticas e fabricantes de dispositivos médicos obtêm lucros de monopólio porque a lei lhes concede direitos de patente. Patentes não fazem sentido em termos econômicos, ao contrário do que seus apologistas corporativos talvez lhe digam. Mas elas aumentam os lucros de quem as tem, às custas das pessoas comuns.

\item Trabalhando em conjunto com grupos de profissionais de saúde, leis estaduais e nacionais impõem requisitos de licenciamento que limitam quem pode fornecer serviços de saúde. Ao restringir o número de pessoas que praticam diversas profissões de saúde e os tipos de serviços que determinados profissionais podem realizar, esses requisitos aumentam a renda dos profissionais de saúde e elevam artificialmente os preços de seus serviços.

\item Requisitos de licenciamento e credenciamento de hospitais limitam de maneira semelhante o número de hospitais em operação e, portanto, enchem os bolsos dos hospitais existentes enquanto elevam os preços que as pessoas têm de pagar por hospitalizações e outros serviços hospitalares.

\item Limitações legais sobre a publicidade na área da saúde também fazem com que seja mais fácil para que profissionais de saúde mantenham altas rendas.

\item Regras que permitem indenizações puramente punitivas em ações judiciais de prática médica aumentam a imprevisibilidade para prestadores de serviços de saúde e incentivam a prática da medicina defensiva, com resultados previsíveis para os custos de serviços de saúde.

\item O processo de aprovação da FDA\footnoteNT{N.T.: \emph{Food and Drug Administration}. Agência reguladora do governo dos EUA responsável pelo controle de alimentos e medicamentos em geral.} aumenta os custos dos medicamentos (e prolonga o tempo para que muitos medicamentos se tornem disponíveis) de maneiras que certamente nem sempre beneficiam os clientes da área de saúde. O mesmo acontece com outras restrições legais à produção e venda de medicamentos.

\item Regras que fornecem incentivos fiscais para que empresas adquiram seguros de saúde para os trabalhadores tendem a tornar mais fácil para as seguradoras cobrarem preços mais altos do que elas provavelmente seriam capazes de cobrar de clientes individuais.

\item Regras vigentes que impedem a compra de seguros entre estados também tornam mais fácil para as seguradoras cobrarem prêmios de seguro elevados e obterem grandes lucros.

\item Regras que limitam quem, para início de conversa, pode ser um segurador podem ter um efeito semelhante. Um médico que desejava oferecer atendimento aos pacientes com base em uma taxa fixa anual foi recentemente impedido de fazê-lo porque esse acordo parecia muito com um seguro e o médico não era um segurador licenciado. Quem se beneficiou? Não foram os pacientes, claramente -- mas a indústria de seguros.

\item Subsídios agrícolas também contribuem para custos de saúde ao incentivar a compra de vários alimentos de baixo teor nutricional. A compra desses itens simultaneamente redireciona recursos que poderiam ser utilizados para comprar alimentos benéficos à saúde das pessoas e incentiva a compra de itens que podem na verdade prejudicar a saúde e, assim, aumentar os custos de saúde.
\end{itemize}

Esses tipos de privilégios legais são vendidos para o público, é claro, como designados a ajudar as pessoas comuns de diversas maneiras. Mas seu efeito prático -- e, em muitos casos, intencional -- é o de tirar dinheiro dos clientes comuns da área de saúde e transferi-lo para pessoas e organizações com maiores privilégios políticos.

A ação do estado também limita o acesso à saúde ao \emph{reduzir} a renda das pessoas que podem querer ter acesso a cuidados de saúde ou outros bens -- mas não podem pagar por eles. O estímulo à pobreza pelo estado de todas as maneiras que descrevi anteriormente torna o impacto dos privilégios garantidos pelo estado sobre as pessoas economicamente vulneráveis ainda mais grave do que seria de outra forma.

Muitos participantes nas recentes discussões americanas sobre saúde compartilham a suposição de que agentes políticos privilegiados e bem relacionados podem e devem manter seus privilégios. As opções que receberam a maior atenção nos debates recentes sobre saúde são opções que, em grande parte, tratam os privilégios desfrutados por aqueles politicamente conectados, e os fardos legais sobre os economicamente vulneráveis, como coisas não problemáticas. Mas a melhor maneira de garantir o acesso à saúde é redistribuir a riqueza daqueles que se beneficiam de privilégios especiais para as pessoas comuns ao eliminar as regras injustas que permitem que os ricos e politicamente favorecidos obtenham lucros de monopólio e que tornam e mantêm os pobres, pobres.

\section{Pobreza, hierarquia ou anarquia?}

O estado não pode efetivamente fornecer uma gestão da economia em nível macro. E quando ele se envolve na operação de indústrias e empresas e no comportamento econômico de pessoas e famílias, ele previsivelmente reforça a riqueza e poder daqueles que já são ricos e poderosos. Ele cria monopólios que protegem as empresas pertencentes a membros da elite da pressão exercida por outras empresas. Ele força as pessoas comuns a subsidiá-los em todas oportunidades. Ele impõe fardos e ergue barreiras que tornam e mantêm as pessoas pobres. E, quando as pessoas podem trabalhar, elas muitas vezes se encontram trabalhando em organizações que podem se dar o luxo de manter estruturas hierárquicas centralizadas devido às posições monopolistas que elas desfrutam graças à ação do estado e dos subsídios que o estado lhes fornece.

O fato de que o estado serve aos interesses da elite enquanto frequentemente desconsidera ou prejudica o bem-estar dos trabalhadores e dos pobres não é mero acidente. Enquanto houver um estado, ele estará vulnerável ao \emph{lobbying} e à manipulação, e os ricos estarão mais bem equipados para realizar \emph{lobby} e manipular. Além disso, mesmo que os funcionários do estado pudessem de alguma forma se tornar invulneráveis ao \emph{lobby} e suborno, esses próprios funcionários do estado ainda poderiam, e sem dúvida frequentemente iriam, tirar o máximo proveito de seu poder para enriquecer às custas dos outros. Mesmo que a classe dominante existente fosse de alguma forma eliminada, os funcionários do estado poderiam e provavelmente iriam se transformar em uma nova classe dominante.

O problema, enfatizo, não são, \emph{per se}, pessoas específicas. O problema é o vasto poder que o estado exerce, seu poder de cartelizar, regular, subsidiar e exigir tributos e obrigar o cumprimento de leis através do medo. A capacidade de exercer esse tipo de poder é o que cria oportunidades de malícias e tentações para que as pessoas explorem e dominem os outros e beneficiem seus comparsas políticos. Não estou dizendo que todo mundo faz ou fará isso; mas, com todo esse poder à sua disposição, algumas pessoas quase certamente o farão, com resultados deploráveis. Pessoas organizando suas vidas econômicas de maneira livre e pacífica certamente podem cometer erros. Mas elas não serão prontamente capazes de transferir os custos de seus erros para outras pessoas. E sem um aparato estatal para ampliar drasticamente as consequências de suas escolhas ao aumentar o seu poder, pessoas que fazem escolhas erradas em uma sociedade sem estado não chegarão nem perto de serem capazes de causar tanto dano quanto agentes estatais. Há boas razões para pensar que uma sociedade sem estado seria mais livre, mais eficiente, menos hierárquica, menos pobre do que uma sociedade supervisionada por um estado. Isso é motivo suficiente para que eu seja anarquista.