% Sexto capítulo

\chapter{Para Onde Vamos?}
\label{chap:6}

\lettrine[lines=2]{\textcolor{LettrineColor}{\textbf{O}}}{ problema do} estado não é um ou outro político ruim. Não são apenas os Republicanos. Não são apenas os Democratas. (Ele não se limita a nenhum partido em nenhum país.) O problema é o estado. Ele cria oportunidades para pilhar e abusar que são incrivelmente atrativas a qualquer pessoa com a potencial capacidade de usá-las para explorar outras pessoas.

As pessoas no comando do estado previsivelmente tenderão a ser pessoas ruins. Políticos e funcionários nomeados de alto escalão não são membros representantes da população. Tornar-se um político bem-sucedido é muito mais fácil se você tiver um determinado tipo de caráter -- se você for ambicioso, lisonjeiro e disposto a comprometer os seus princípios (ou se você convenientemente se esqueceu de tê-los). O processo político -- qualquer processo político que dê às pessoas o acesso ao poder -- recompensa o mau comportamento.

Um compromisso com princípios tende a ser muito menos influente no comportamento de servidores públicos eleitos e nomeados do que a tendência de transigir, de buscar poder, de ser ambicioso a todo custo, de se tornar parte da elite (ou, se já for membro dela, de consolidar a sua posição), de recompensar os seus apoiadores da elite. Provavelmente não é de surpreender que Jim Kouri, vice-presidente da Associação Nacional de Chefes de Polícia, tenha sugerido que políticos profissionais exibem muitas das mesmas características de psicopatas reconhecidos.\footnoteA{\raggedright Jim Kouri, ``Serial Killers and Politicians Share Traits,'' \emph{Law Enforcement Examiner} (n.p., June 12, 2009) <\url{http://www.examiner.com/examiner/x-2684-Law-EnforcementExaminer~y2009m6d12-Serial-killers-and-politicians-share-traits}> (July 5, 2010).}

Embora certamente existam funcionários estatais idealistas, o estado serve principalmente às elites políticas e financeiras que fazem uso dos meios políticos para obter riquezas. O estado fornece à elite um insubstituível meio de acesso à riqueza e ao poder. Se os atuais servidores públicos forem substituídos por novos, o poder do estado se provará tentador demais para que muitos deles resistam.

Pessoas boas que se encontram em posições de poder serão profundamente tentadas a se tornar pessoas não tão boas. Quase todas as autoridades eleitas precisam do apoio dos ricos e poderosos para ganhar uma eleição. E quando alguém é eleito, esse alguém se torna alvo de um intenso \emph{lobbying} por grupos de interesse -- que sempre podem dedicar muito mais tempo, energia e dinheiro nos servidores públicos do que as pessoas comuns poderiam. Poderosos grupos de interesse podem oferecer não apenas contribuições de campanha (e subornos explícitos), mas também influência que os torna inestimáveis para os políticos.

Mesmo quando as autoridades estatais realmente \emph{desejam} tomar boas decisões, elas carecem -- qualquer autoridade central careceria -- das informações necessárias para tomar essas decisões, as informações distribuídas por toda a população. Funcionários estatais que se apegaram emocionalmente a suas posições e que não desejam confrontar a dissonância cognitiva que seria gerada se essas posições fossem consideradas inúteis farão o seu melhor para resistir às tentativas de deslegitimar o estado ou de defender que seu poder deveria ser reduzido. O impulso continuará a conduzir o estado em direção à aquisição de mais e mais poder. E a mídia a promover o respeito às agências militares, policiais e de inteligência e a valorizar o comportamento autoritário e o poder político.

No nível nacional, a presidência imperial cresce com o apoio de todo o espectro político tradicional. Em todos os níveis de governo, há uma crescente confiança em administradores \emph{experts} isentos de responsabilidade. E o poder estatal tende a se alimentar sozinho. Uma vez que uma entidade tem poder, não é muito provável que ela renuncie a esse poder. Os governos não abrem mão da autoridade; eles continuam acumulando-na. Oficiais e burocratas eleitos quase sempre acharão mais fácil reter o poder, mesmo o poder que eles argumentaram que não deveria ser exercido pelos ``caras maus'', do que cedê-lo. Por exemplo: não foi surpresa nenhuma, infelizmente, quando o governo Obama argumentou com entusiasmo a favor do privilégio dos segredos de estado pelo qual os Democratas haviam duramente criticado o governo Bush. O problema não é que esse ou aquele grupo de pessoas está controlando o estado. O problema é o estado.

O exercício desse tipo de poder não apenas corrompe aqueles que o usam, mas é também intrinsecamente assustador -- o estado afirma ter, e procura exercer, o poder sobre a vida e a morte das pessoas, o poder de reivindicar seus bens à vontade, de aprisioná-las, de escravizá-las. O estado não é acidentalmente mas essencialmente contrário à liberdade humana. E então as pessoas que buscam a libertação humana têm toda razão para idealizar e materializar alternativas reais ao estado. Agentes estatais podem ser facilmente corrompidos por grupos de interesse. O poder do estado torna fácil que pessoas sigam as suas próprias agendas, muitas vezes sob a aparência da lei, de maneiras que são muito difíceis que as pessoas comuns descubram e desafiem. Devido ao estado ter tanto poder, mesmo erros bem-intencionados podem ter consequências terríveis. E as perversidades coordenadas pelas elites que dirigem o curso do estado -- perversidades como a guerra -- podem ser devastadoras para sociedades inteiras. O estado é perigoso.

Tentativas de reformar e reorientar o estado não são muito promissoras. Eleger ou nomear (como?) pessoas mais decentes, honestas e justas e compassivas para cargos não eliminaria os problemas que descrevi aqui. A boa notícia, porém, é que as pessoas comuns podem elaborar e manter maneiras eficazes de conviver juntas e de resolver problemas humanos na ausência do estado.

Tenho convicção de que a vida sem o estado proporcionará muitas oportunidades para que diversas formas culturais e convicções ideológicas se expressem. E suspeito que as maneiras mais eficazes de ir além do estado possam incluir experimentos práticos de agir sem ele agora mesmo.

\section{A forma de viver sem o estado}

Anarquistas muitas vezes passam muito tempo imaginando como poderia ser a vida sem o estado. Confesso que eu não sei. Eu não tenho um plano e, se tivesse, não gostaria de impô-lo a todos os outros. A boa notícia, talvez a melhor notícia de todas, sobre a anarquia é que existem mais maneiras de organizar comunidades e resolver problemas do que você, eu ou qualquer outra pessoa poderíamos imaginar. É por isso que sou a favor do que às vezes é chamado de ``panarquia'' ou ``anarquia sem adjetivos''. Uma sociedade sem estado deve ser acolhedora a primitivistas e tecnófilos e transhumanistas, defensores de mercados e apoiadores entusiastas de economias do dom,\footnoteNT{N.T.: \emph{gift economies}. Também conhecida como economia da dádiva, é uma concepção da economia e organização social que remete ao trabalho do antropólogo francês Marcel Mauss.} ateístas e fundamentalistas, defensores tanto da propriedade individual quanto da comunal, localistas e cosmopolitas. Comunidades e redes voluntárias decentes (e, é claro, assim como é hoje, a decência não será algo universal) ajudarão as pessoas a resolver disputas, proteger pessoas -- especialmente as vulneráveis -- e animais contra a violência e a injustiça, assegurar [\emph{insure}] pessoas contra riscos e ajudar a protegê-las das consequências da insegurança econômica.

O estado tende a eliminar alternativas, tornando as pessoas dependentes por padrão dos serviços que ele fornece. Quanto mais o poder do estado cresce, mais uma próspera rede de provedores de serviços sociais fora do estado tende a atrofiar. No entanto, há uma forte tradição de ajuda mútua que fornece boas evidências de que as pessoas \emph{podem} cuidar umas das outras sem o envolvimento do estado.

É claro, o fato de que elas podem fazer isso não nos diz como os serviços serão prestados e apoiados, quais prioridades faz sentido adotar e como as tarefas serão alocadas entre as diferentes comunidades, redes e associações sobrepostas, às quais as pessoas estão afiliadas. O ``como'' sem dúvida será significativamente diferente de um lugar para outro. Diferentes abordagens refletirão o processo criativo e experimental de descoberta e invenção que a anarquia tornará possível.

\subsection*{Tipos de comunidades}

Provavelmente haverá todo tipo de comunidade em uma sociedade sem estado. Sem dúvidas algumas delas serão geográficas: será mais fácil para as pessoas administrarem alguns tipos de tarefas enquanto trabalham junto de pessoas que moram e trabalham perto de onde elas residem -- uma associação voluntária de pessoas que administram um hospital comunitário, por exemplo, provavelmente será composta, em geral, de pessoas geograficamente próximas. Afinal, deslocar pacientes e prestadores de serviços de saúde leva tempo e consome recursos, e as pessoas, presumivelmente, gostam de poder fazer uso de serviços hospitalares de forma rápida e eficiente. É claro, não haverá nenhum estado para insistir que alguém que more em uma determinada área geográfica participe de uma associação comunitária específica ou para forçar a exigência de tributos de tal associação. E não haverá guardas de fronteira e postos de passaporte para impedir as pessoas de se mudarem de uma região geográfica para outra e, então, de se afiliarem a diferentes comunidades estabelecidas geograficamente.

Ao mesmo tempo, é claro, muitas comunidades provavelmente não serão estabelecidas geograficamente. Por exemplo: algo como a antiga \emph{lex mercatoria} pode administrar certos tipos de transações econômicas. Pessoas interessadas em tirar proveito de um conjunto consistente de regras para tais transações podem pertencer a comunidades virtuais, conectadas pela Internet e construídas em redes de confiança interligadas, que as conectam umas às outras enquanto se movem e trabalham em diversas regiões geográficas. Essas comunidades virtuais voluntárias podem muito bem oferecer serviços de resolução de disputas fundamentados em princípios amplamente compartilhados entre seus membros. Ser membro de tais comunidades pode tornar mais fácil para as pessoas estabelecerem relacionamentos de confiança com os outros. E os membros que tiraram proveito da confiança dos outros membros logo se veriam sem o benefício de ser um membro da comunidade.

Pessoas de todos os tipos que estão dispostas a viver pacificamente, a permitir que os membros de suas próprias comunidades saiam ou se manifestem sobre ela, podem contribuir criativamente para o processo contínuo de experimentação e descoberta que permitirá que uma sociedade sem estado floresça. Nem todas as comunidades funcionarão bem -- algumas, de fato, podem ser disfuncionais e mesmo destrutivas. E muitas, sem dúvidas, assumirão visões sobre o que é uma boa vida que são bem diferentes da minha. Não tem problema. 

Quem irá decidir como será a vida na anarquia? \emph{Nós} iremos, todos nós, por meio das inúmeras decisões que tomamos em comunidades, locais de trabalho, associações e círculos espontâneos (em todos esses casos, geograficamente localizados e virtuais). Como será o plano, o sistema, as normas dominantes? Não haverá nenhum. Haverá apenas os diversos planos e sistemas e normas criadas e mantidas por pessoas genuinamente livres em uma fascinante variedade de ambientes e seguindo uma inacreditável riqueza de padrões.

\subsection*{A anarquia como um processo de descoberta}

Tenho fortíssimas convicções sobre como eu gostaria de ver as coisas funcionarem sem o estado. Algumas das minhas convicções são morais -- acredito que algumas coisas seriam injustas e exploradoras e subordinadoras e excluidoras. Algumas delas são práticas, empíricas -- acredito que as burocracias autoritárias não são muito competentes na gestão da produção e distribuição de bens e serviços. Eu não teria essas convicções se não acreditasse que elas fossem plausíveis. Mas reconheço que posso estar completamente errado sobre várias delas.

De fato, essa é uma das razões pela qual eu considero o anarquismo atraente. Sem um pouco de humildade intelectual, fica fácil assumir que tenho um modelo, um plano que é ideal para todos, que tudo que necessito é a rainha-filósofa benevolente correta para implementá-lo. Mas é claro que é esse tipo de idealismo ingênuo a respeito das capacidades dos estados e das motivações dos agentes estatais que nos colocou na confusão em que estamos agora, uma confusão em que o estado nos tiraniza -- supostamente para o nosso próprio bem.

Abraçar a humildade intelectual, reconhecendo que posso muito bem estar completamente errado, é uma razão crucial para não apoiar nenhum tipo de padrão genérico a ser imposto a todas comunidades em uma sociedade sem estado. A anarquia dará às pessoas a liberdade para experimentar, para descobrir o que funciona, para testar ideias e ideologias e descobrir o que acontece quando elas são realmente colocadas em prática. Algumas opções funcionarão bem -- as pessoas irão aprimorá-las e refiná-las. Outras provavelmente serão desastrosas -- as pessoas irão abandoná-las com alívio. E outras provavelmente se mostrarão estáveis o suficiente de forma que as pessoas que sejam apegadas a elas as preservem e continuem dando um jeito de persistir. A questão é que somente as experimentando que as pessoas realmente descobrirão quanto mérito elas realmente têm. (Uma característica vantajosa desse tipo de experimentação é que, se der completamente errado, os resultados não serão, nem podem ser, tão catastróficos quanto seriam se um enorme e poderoso aparato estatal cometesse um erro drástico. Um estado coercitivo, de grande escala, pode causar muito mais danos do que uma comunidade voluntária, virtual ou geográfica, de pequena escala.)

\subsection*{Variedades de méritos não estatais}

Isso não significa que todas as opções sejam igualmente boas, ou que a noção de que podemos formar juízos sólidos sobre o que é certo e errado, bom e mau, simplesmente vá por água abaixo. Ser um anarquista não o obriga a ser um relativista ou niilista. Mas existem diversas maneiras de ser prosperamente humano. Uma vida humana viável não exige que todos sigamos os mesmos padrões culturais, apoiemos as mesmas tradições, imponhamos os mesmos costumes populares à próxima geração. Algumas pessoas, por exemplo, prosperam em ambientes cosmopolitas, agitados, enquanto outras preferem a estabilidade e a familiaridade de comunidades relativamente independentes. Desde que ninguém em uma determinada comunidade seja coagido a obedecer, escravizado, impedido de sair, desde que todos sejam tratados com decência e respeito, não há boas razões para alguém se opor à existência ou funcionamento dessa comunidade. Indivíduos e redes em uma sociedade sem estado podem e devem ajudar outras pessoas que estão tentando se livrar da escravidão, fugir do abuso ou acabar com a tirania. Eles podem e devem desafiar os padrões culturais e as instituições comunitárias que oprimem e excluem. Mas isso não significa que anarquistas intrometidos gastarão seu tempo tentando moldar as comunidades dos outros para que elas se pareçam com as suas.

Na prática, as pessoas não terão tempo, energia e recursos para se envolver em campanhas humanitárias ingênuas [\emph{do-gooding}] sem pensar. E a tolerância mútua entre membros de diferentes comunidades e redes voluntárias (dentro de certos limites -- talvez nem sempre seja eficiente ou necessário que pessoas ativamente intervenham em situações injustas, mas a real injustiça nunca pode ser tratada como trivial) é obviamente crucial para evitar conflitos constantes. Moralmente, o respeito pela liberdade e dignidade dos outros criará uma presunção (mesmo que não seja uma presunção irrevogável) contra a tentativa de remodelar os seus modos de vida. E os benefícios extensivamente compartilhados de deixar que diferentes grupos de pessoas sejam livres para explorar diferentes estratégias voluntárias de viver bem dará a todos uma razão para deixar que o processo de descoberta nas várias comunidades continue.

\subsection*{O poder do exemplo}

Via de regra, as pessoas aprendem de maneira mais eficiente não assistindo aulas, mas vendo e experienciando por si mesmas. Se eu participar da vida de uma comunidade específica, certamente me beneficiarei com as descobertas contínuas dessa comunidade sobre o que funciona e o que não funciona. Mas também reconheço que o sucesso da minha comunidade tornará mais fácil que ideais importantes para mim sejam compartilhados com outras pessoas -- vendo esses ideais à mostra, elas tenderão a reconhecer o seu valor e a respeitar minha comunidade e a maneira como ela funciona.

Por sua vez, isso também significa que comunidades que -- pacificamente, voluntariamente -- exploram ideais drasticamente diferentes dos meus, mesmo diametralmente opostos a eles, fazem um favor a mim e a todos os outros. Se elas funcionarem, elas me desafiam a descobrir novas possibilidades humanas que eu estive inclinado a ignorar. E elas me ajudam a lembrar que não sou o centro do universo, que tudo nem sempre é como eu percebo, que as coisas nem sempre são como eu espero que sejam. Elas contribuem para o processo contínuo de libertação dos meus preconceitos, da minha relutância em ser surpreendido. Por outro lado, obviamente, se elas não funcionarem, isso também ajuda. Elas fornecem uma clara demonstração de que os tipos de ideais com os quais me importo fazem algum sentido, merecem alguma consideração cuidadosa. Elas estendem a discussão e aprofundam a compreensão humana, mesmo se fracassarem.

\section{Em vez de um manual de instruções}

A anarquia é nossa para que a criemos. Então, como começamos?

Diferentes tipos de anarquistas tenderão a traçar caminhos muito diferentes para a anarquia. Contrário ao que você possa ter concluído até o momento, não sou realmente um ideólogo dogmático. Acredito que muitas estratégias diferentes têm seu mérito. Acredito que muitas delas são complementares. E acredito que vale a pena explorar a maioria delas.

Uma das convicções básicas que fundamentam o anarquismo, percebi enquanto escrevia este livro, é que as pessoas são maravilhosamente, gloriosamente, surpreendentemente diversas. Não existe uma abordagem única para se dedicar à anarquia. Em primeiro lugar, não tenho a ilusão de que a maioria das pessoas que lêem este livro e que consideram convincente o que eu disse têm qualquer obrigação de se dedicar a buscar deliberadamente o surgimento de uma sociedade sem estado. As pessoas têm objetivos, compromissos, preocupações, interesses, responsabilidades e paixões diferentes. Não tem problema. E, em segundo lugar, mesmo para quem faz do ativismo político uma prioridade, não existe uma única estratégia correta a se seguir. Isso é verdade porque as livres escolhas das outras pessoas sempre tornam os resultados das nossas ações incertos. É verdade porque, mesmo se ninguém tivesse a capacidade de escolher livremente, não haveria como dizer quais seriam exatamente os resultados de seguir um determinado curso de ação individual. Também é verdade porque existem muitos objetivos importantes diferentes, e muitas maneiras diferentes e razoáveis de atingir esses objetivos; em geral, não faz muito sentido dizer que uma escolha moralmente boa gera ``mais bem'' no total do que outra.

Então, não tenho decreto para ninguém -- exceto: reconheça o valor do que as outras pessoas estão fazendo, mesmo que não seja o que você queira fazer você mesmo. Várias estratégias têm chance de ser bem-sucedidas em ambientes diferentes. Experimente, veja o que funciona e ouça o que as outras pessoas têm a dizer.

\subsection*{Comece libertando sua mente}

É difícil libertar outras pessoas quando você mesmo não é livre. É muito fácil acabar descarregando a sua própria bagagem emocional ou se tornar um cruzado hipócrita e sem senso de humor -- a imagem refletida da figura estatista de autoridade que você gostaria de se afastar. Leve o tempo que for preciso para descobrir como as estruturas da autoridade e da dominação são replicadas -- não somente sob a mira de armas, mas nas famílias, escolas e congregações religiosas, e, de fato, na sua própria mente e coração. Figuras de autoridade do passado -- professores, pais, pastores, chefes -- ainda mandam na sua vida de dentro da sua cabeça? Você ainda está envolvido em conflitos reprimidos, raivosos com pessoas que não estão de fato presentes? Quanta importância a culpa e medo irracional têm em sua vida?\footnoteA{Algumas culpas são racionais. Se eu fizer algo genuinamente errado -- talvez eu faça uso da violência agressiva, quebre a confiança de alguém ou responda com uma dura insensibilidade a um caso de necessidade genuína --, tenho boas razões para me sentir mal por fazer isso. Mas muitas culpas são irracionais: elas surgem não de uma real injustiça, mas de desconsiderar a suposta autoridade de pessoas e instituições que buscam dominar nossas vidas. Parte da libertação da sua mente consiste em abandonar impiedosamente \emph{esse} tipo de culpa.}

Trabalhe nessas questões consigo mesmo, com amigos de confiança, talvez com um terapeuta ou guia espiritual. Certifique-se de que você está pronto para oferecer a liberdade genuína às outras pessoas -- em vez de tentar impor a sua própria inflexibilidade ideológica e reafirmar suas próprias crenças convertendo os outros -- ao falar com elas sobre a atratividade de uma sociedade sem estado.

Dedique-se também a se libertar da raiva e do ressentimento. Você se esgotará, afastará outras pessoas e correrá o risco de tomar decisões imprudentes, até mesmo destrutivas, se o seu ativismo resultar de emoções negativas. As pessoas serão atraídas à sua causa, você encontrará energia e inspiração para continuar se dedicando a essa causa e você terá mais chances de conduzir essa causa a direções úteis se seu compromisso com uma sociedade livre refletir um amor genuíno pelas outras pessoas e por si mesmo.

\subsection*{Construa amizades libertadoras}

A maioria dos seus amigos não serão anarquistas. Muitos deles podem nunca levar o anarquismo a sério. Não tem problema: não é seu trabalho transformá-los em clones de você mesmo; e o sucesso de seus esforços para construir uma sociedade sem estado não depende que todos venham a compartilhar as suas crenças. Suas amizades certamente não precisam ser políticas -- a última coisa que precisamos é daquela revisão sombria das relações pessoais para correção ideológica que marcou muitos movimentos políticos em todo o espectro. Anarquismo é sobre viver uma boa vida e a amizade é um aspecto maravilhoso do bem-estar humano; é algo bom que seus amigos concordem com você sobre alguma coisa, e você perverte isso se transformar em uma oportunidade para proselitismo. E, é claro, nada o impede de se dedicar a campanhas específicas com amigos que podem discordar significativamente de você sobre muitas ou a maioria das questões.

No entanto, a maneira mais importante de deixar as pessoas empolgadas com a possibilidade da libertação é pessoalmente conectar-se a elas. Fazer isso é muito mais útil, muito mais efetivo do que postar um vídeo no YouTube de você mesmo fazendo uma apresentação no PowerPoint. Construir laços humanos genuínos com as pessoas permite que elas entendam suas preocupações e seus objetivos e vejam por que eles podem ser interessantes. Se você deseja relacionar-se politicamente com as pessoas, não palestre de uma posição de autoridade; converse com elas sobre problemas e frustrações, experiências de indignação moral com abusos do estado ou medos da autoridade do estado, descobertas de alternativas às maneiras de fazer as coisas baseadas no estado.

Em última análise, as pessoas são mais leais aos seus amigos do que a movimentos ou ideias. As pessoas fazem coisas para apoiar seus amigos que elas seriam menos propensas a fazer para apoiar ideologias abstratas. Compartilhe a possibilidade da liberdade com seus amigos, de forma não dogmática e generosa, e você tornará a causa do anarquismo incomensuravelmente mais forte.

\subsection*{Transforme instituições de pequena escala}

O autoritarismo começa em casa. Quando as crianças têm idade suficiente para entender argumentos, mas os pais ainda insistem que eles sejam obedecidos apenas porque são pais ou porque as crianças moram em suas casas, eles reforçam a mensagem de que o respeito à posição de autoridade é uma parte essencial e inevitável da vida. Pais que tratam seus filhos de forma humilhante ou que usam força física contra eles quando nunca fariam o mesmo contra adultos comunicam que pessoas com autoridade respondem a regras diferentes que os outros e que a violência agressiva ou punitiva é uma forma aceitável de resolver problemas. Se o tipo de personalidade que dá origem ao estado é formado em ambientes domésticos que são autoritários e brutais, então, se você se preocupa em tornar o mundo livre de estados, vale a pena dar uma olhada no que se passa na sua própria casa (e, em alguns casos extremos, na dos outros).

Os mesmos padrões que vemos nas casas também são percebidos nas escolas e congregações religiosas. Formatos rígidos de aprendizagem, respeito obrigatório à posição de autoridade dos professores e códigos de conduta impostos violentamente sufocam a individualidade e geram uma atitude marcada simultaneamente pela aceitação da autoridade e pelo desejo ressentido de escapar do seu domínio ao tornar-se capaz de exercê-lo sobre os outros. Pense em como as escolas na sua comunidade são organizadas e se alternativas são viáveis.

De maneira semelhante: congregações religiosas podem ser criadouros para o autoritarismo. Pastores que agem como se tivessem acesso especial à verdade revelada, que insinuam não possuir as falhas que criticam nos membros de suas congregações, que suscitam uma falsa culpa ao desconsiderarem sua autoridade ou ao se envolverem em violações de tabus que não prejudicam ninguém, todos contribuem para a criação de uma mentalidade de servidão ressentida. Além disso, muitos pastores proclamam a importância de reconhecer a legitimidade de outras figuras de autoridade, incluindo pais, parceiros masculinos das mulheres e funcionários do estado: basta observar as exposições patrióticas nos feriados proclamados pelo estado.

Se você faz parte de uma congregação religiosa, reflita sobre como sua congregação está organizada e como você pode contribuir para torná-la um lugar acolhedor e libertador em que as pessoas constroem amizades umas com as outras e cooperam para cuidar da sua comunidade. Dedique-se para garantir que os membros do clero entendam que trabalham em prol da congregação, em vez de enxergarem você como um subordinado. Aumente seu tom de voz para se opor ao autoritarismo -- especialmente quando ele tem o potencial de influenciar as atitudes das crianças -- e ao culto ao estado no altar religioso. Uma congregação saudável pode ser uma alternativa ao estado, uma comunidade que apresenta uma concepção muito diferente da inclusão e empoderamento mútuo. Ajude a sua a ser esse tipo de congregação, em vez de uma que aprova a autoridade do estado.

\subsection*{Ajude os outros a escaparem do controle do estado}

Forças militares e agências de aplicação da lei provavelmente fazem mais para sustentar a autoridade do estado do que quaisquer outras organizações. Ajude a deslegitimar essas organizações. Conscientize as pessoas das coisas destrutivas que o estado ordena que elas façam. E informe as pessoas sobre as coisas violentas que elas fazem mesmo quando \emph{não} estão agindo sob ordens. Você pode trabalhar com coalizões na sua comunidade que se opõem ao abuso policial, talvez principalmente em defesa de diversos grupos marginalizados -- membros de minorias étnicas, mulheres, imigrantes ilegais ou pessoas LGBT. E você pode encorajar pessoas que você sabe que querem fazer a diferença a não se juntarem ao exército permanente ou à força policial do estado e a explorarem maneiras alternativas de atingir os objetivos que pessoas benevolentes talvez queiram atingir ao se alistarem -- defender a paz e fornecer segurança aos outros.

\subsection*{Envolva-se em litígios}

Infelizmente, é improvável que um processo judicial derrube o estado. Mas processos judiciais individuais podem enfraquecer a autoridade arbitrária do estado. Um processo judicial pode tornar mais fácil para alguém evitar servir nas forças armadas do estado ou evitar que o estado impeça alguém de trabalhar apenas porque outra pessoa deseja ter um monopólio. Esforços direcionados de litígio podem ajudar a desgastar o poder autoritário -- assim como eles enfraqueceram a segregação escolar imposta pelo estado na metade do século XX. Considere participar do processo de um litígio no papel de advogado, demandante (se houver chance de você ser um demandante apelativo), pesquisador, financiador ou publicitário.

\subsection*{Busque estratégias eleitorais}

Existem razões tanto estratégicas quanto morais para sermos céticos em relação à política eleitoral. O envolvimento na política eleitoral pode consumir energia e recursos. E pode incentivar tanto eleitores quanto possíveis detentores de cargos a superestimar o real potencial do governo de mudar as coisas e a tratar a contínua existência do estado como aceitável. Além disso, pessoas que realmente são eleitas a cargos públicos podem ser cooptadas: elas podem se tornar parte da, em vez de se oporem à, expansão quase inevitável do poder estatal. A participação na política eleitoral pode acabar implicando ser um real cúmplice das injustiças do estado.

Em última análise, essas afirmações podem ser persuasivas, mas acredito que os anarquistas deveriam pelo menos considerar se o envolvimento na política eleitoral pode às vezes ser uma estratégia razoável para promover uma mudança social efetiva. Se esse envolvimento desperdiça tempo e energia em um determinado caso ou não, é um julgamento específico à situação que pode exigir mais conhecimento das consequências das ações do que muitas vezes (ou sempre) seria possível ter. Pode às vezes levar a crenças indesejáveis acerca do estado e da efetividade das estratégias de mudança baseadas no estado, mas me parece que, apesar disso tudo, pode ser razoável, dados seus possíveis resultados -- incluindo tanto um aumento da consciência pública quanto possíveis mudanças em leis e políticas. Não há nenhuma real maneira de responder à questão moral, acredito eu, sem responder a questões mais fundamentais sobre juízos morais que dificilmente cabem a um livro como esse. Então, irei simplesmente oferecer minha visão de forma conclusiva. Acredito que você é cúmplice da injustiça se apoiar propositalmente a injustiça ou agir de forma insensata para aumentar a chance dela acontecer. E acredito que o envolvimento em uma campanha política ou, de fato, em uma candidatura ou em um trabalho em um cargo político não requer que você se envolva em nenhuma dessas duas coisas.

Pode envolver, é claro. E a tentação pode ser especialmente difícil de resistir para quem vê seu objetivo político como o de melhorar a máquina do estado, fazer as coisas funcionarem de maneira mais eficiente, expulsar os bandidos ou restaurar o nível de atividade do estado ao seu nível ``ótimo'' (ou ``constitucionalmente permissível''). Certamente, enquanto estivermos sujeitos a um estado, temos toda razão para querer que ele seja menos custoso, menos destrutivo, mais justo. Mas se melhorar o estado, em vez de eliminá-lo, se tornar o objetivo de um político, ele achará extremamente difícil não se tornar parte do estado e fazer parte -- talvez não intencionalmente, talvez irrefletidamente -- da contínua expansão, quase inevitável, do seu poder. Alguém realmente comprometido com o anarquismo tem a maior probabilidade de fazer uma diferença politicamente positiva se esse alguém entender que o estado não pode ser curado, que ele deve ser eliminado.

Dado, no entanto, que \emph{alguns} tipos de envolvimento político podem às vezes fazer algum sentido estratégico e moral para os anarquistas, existem diversas formas de envolvimento que talvez sejam apropriadas.

\begin{itemize}
\item Um candidato poderia concorrer e ser apoiado por outros -- sob a bandeira de um partido que busca restringir, reverter ou eliminar o estado ou como um candidato a ser indicado por outro partido -- principalmente para que o candidato e seus apoiadores possam tirar vantagem da publicidade em torno da campanha para enfatizar uma posição anarquista nas questões relevantes.

\item Um candidato poderia concorrer e ser apoiado por outros -- sob a bandeira de um partido que busca restringir, reverter ou eliminar o estado -- com o objetivo de obter o cargo para o qual ele é candidato. Obviamente, esse tipo de abordagem faria sentido em um número limitado de casos, já que a maioria dos eleitores em todas as sociedades ocidentais são filiados a partidos que são, para dizer o mínimo, favoráveis ao estado, já que esses partidos controlam a máquina do governo, e já que esses partidos provavelmente farão com que seja muito difícil que um cético quanto ao estado tenha sucesso político. Esse tipo de abordagem teria uma maior chance de sucesso em uma comunidade relativamente pequena em que anarquistas e seus simpatizantes estivessem presentes em um número razoavelmente grande.

\item Anarquistas podem apoiar um candidato -- a um cargo ou nomeação -- afiliado a um partido tradicional se o candidato pessoalmente buscar restringir, reverter ou eliminar o estado, independente da agenda do próprio partido. Claro, esse tipo de abordagem é arriscada, visto que um candidato associado a um partido estatista pode realmente compartilhar de elementos significativos da agenda estatista do partido. Além disso, mesmo que não compartilhe, ele ainda pode ser suscetível à pressão de outros de dentro do partido para que não prejudique parte dessa agenda. No entanto, devido à influência substancial que os eleitos conseguem exercer, esse tipo de abordagem pode ocasionalmente fazer sentido.
\end{itemize}

Pode haver ocasiões em que seja razoável que um anarquista siga alguma dessas estratégias. Também pode fazer sentido que anarquistas apoiem referendos em comunidades que permitem que eleitores legislem. Uma iniciativa cuidadosamente planejada poderia desempenhar um papel muito efetivo na redução do poder do estado (considere, por exemplo, a recente aprovação pelos eleitores da Califórnia de uma iniciativa que abre espaço legal para a distribuição de maconha medicinal).

Eu não negaria nem por um segundo a razoabilidade das questões críticas sobre o envolvimento na política eleitoral que mencionei anteriormente. Mas acredito que campanhas eleitorais apropriadamente selecionadas podem criar oportunidades significativas para beneficiar uma agenda anarquista.

\subsection*{Lobby}

Quer seja razoável ou não que anarquistas participem da política eleitoral, não há como escapar do fato de que os oficiais eleitos impactam o ambiente em que todos nós vivemos. Se a injustiça e a violência infligidas pelo estado deveriam ser reduzidas ou eliminadas do nosso atual mundo político, incentivar os oficiais eleitos a apoiar políticas adequadas pode fazer bastante sentido. O processo de \emph{lobbying} é um processo potencialmente sujo, e os anarquistas que participam dele podem correr o risco de perder sua integridade e seu compromisso com a visão anarquista. O que é particularmente preocupante é a possibilidade que anarquistas fazendo \emph{lobby} enquanto fundamentados em princípios sejam cooptados por parceiros mais experientes e cínicos que estejam inclinados a tirar proveito dos compromissos dos anarquistas com seus princípios para uso de suas próprias agendas.

Também potencialmente problemático: é crucial lembrar que atualmente existe uma teia fortemente entrelaçada de políticas estatais interconectadas. O processo de desmantelamento do estado não pode ser realizado sem uma consciência das ligações entre essas políticas. Suponha que a política A seja posta em vigor para recompensar algum grupo privilegiado. Não muito surpreendente, essa política tem um efeito adverso sobre o grupo menos relacionado, mais vulnerável econômica e politicamente. À maneira caracteristicamente \emph{ad hoc} adotada pelos políticos, a política B é então posta em vigor para garantir que as coisas não sejam tão ruins para esse grupo desfavorecido.

Uma estratégia de \emph{lobbying} obstinadamente antiestatal, focada em reverter as ações do estado, pode levar um anarquista a promover a eliminação da política B. Talvez haja um argumento superficial a favor de eliminar essa política. Mas fazê-lo deixaria em vigor o privilégio criado pela política A e, portanto, a deficiência criada pelo estado resultante dessa política. Já que a política B foi designada para corrigir, ao menos em parte, os problemas criados pela política A, deixar A em vigor simplesmente submete o grupo desfavorecido aos danos criados pela concessão de um privilégio especial ao grupo beneficiado pela política A. Uma estratégia sensata de \emph{lobbying} anarquista, portanto, precisaria focar ou em eliminar as políticas A e B juntas \emph{ou} em eliminar A antes de eliminar B, em vez de primeiro eliminar B (principalmente porque, dado que A foi designada para beneficiar um grupo politicamente privilegiado, há grandes chances de que a eliminação de A possa vir a ser mais difícil do que a eliminação de B). Se a política B reduz o privilégio relativo de um grupo, cortesia da implementação da política A pelo estado, certamente não faria sentido que os anarquistas fizessem \emph{lobby} para eliminar a política B sem eliminar a política A: deixar a política A em vigor enquanto elimina a política B resultaria na realidade em um \emph{aumento} dos privilégios injustos conferidos pelo estado.

O \emph{lobbying} representa um real risco pessoalmente para os anarquistas e para as pessoas afetadas pelas políticas em relação às quais eles fazem \emph{lobby}. Ainda assim, se os oficiais eleitos puderem ser convencidos de que políticas antiestatais específicas fazem sentido, o \emph{lobbying} pode trazer resultados genuinamente positivos.

O \emph{lobbying} pode assumir a forma relativamente convencional de comunicação individual com oficiais públicos específicos e suas equipes. Ele também pode envolver uma comunicação mais generalizada com oficiais públicos e formadores de opinião. Uma estratégia óbvia seria usar \emph{think tanks} para produzir relatórios, pesquisas patrocinadas (desde que sua integridade possa ser garantida), livros, artigos acadêmicos, textos opinativos e artigos em jornais destacando a ilegitimidade do estado e a disponibilidade de alternativas confiáveis. (Um exemplo anarquista óbvio: o trabalho do \emph{Center for a Stateless Society}.)

\subsection*{Forme coalizões}

O avanço de uma agenda anarquista certamente não precisa ser realizado apenas por meio de instituições ou organizações explicitamente anarquistas ou anti-estado. Anarquistas práticos trabalharão lado a lado com parceiros propícios. Obviamente, alguns parceiros não compartilharão muitos dos objetivos anarquistas; isso não significa que anarquistas não possam trabalhar com eles para conquistar a liberdade e a justiça. Parceiros naturais podem incluir grupos preocupados com os direitos dos imigrantes, abuso policial, liberdades civis, nacionalismo, \emph{bailouts} corporativos, manipulação estatal da oferta monetária, militarismo e paz.

Embora a formação de coalizões focadas em questões específicas faça muitas vezes bastante sentido para os anarquistas, obviamente é importante aqui, como em outros contextos, evitar ser cooptado ou manipulado. Muitas organizações apaixonadas e guiadas pela justiça serão parceiros maravilhosos para os anarquistas. Outras, no entanto, podem usar a retórica anarquista para promover agendas racistas, corporativistas ou desagradáveis de outra forma. E mesmo parceiros bem intencionados e com princípios, mas de ideologias não anarquistas, podem adotar posições que anarquistas não podem endossar. Isso não significa que parceiros bem intencionados que às vezes assumem posições pró-estado devam ser evitados. É claro que não. Significa, entretanto, que o envolvimento em campanhas focadas em questões específicas com parceiros não anarquistas, principalmente aqueles que podem não ser familiares, requer uma atenção cuidadosa e uma persistente vontade de não permitir que o núcleo dos princípios anarquistas seja comprometido.

\subsection*{Apoie movimentos de secessão}

Estaríamos melhor sem estado nenhum. Mas enquanto houver estados, certamente estaremos todos em melhor situação se eles forem menores do que maiores. Estados menores são mais responsivos às pessoas comuns e menos capazes de serem perigosos para outras comunidades. Então há boas razões, em geral, para apoiar a liberdade das pessoas de se separarem dos estados existentes. Isso também é verdade porque fazê-lo ressalta o fato de que as pessoas têm a liberdade de optar por sair do estado, de dizer \emph{não}. E uma vez que esse princípio tenha sido reconhecido, é difícil não seguir a cadeia de raciocínios que leva à sua conclusão lógica -- que não apenas unidades políticas menores, mas pessoas individuais, deveriam ser capazes de optar por sair.

Movimentos de secessão se apresentam em diferentes formas e tamanhos. Nos Estados Unidos, talvez os mais visíveis sejam os movimentos em Vermont -- associados à esquerda política -- e em Nova Hampshire -- associados em parte à ala populista da direita política (digo em parte porque o \emph{Free State Project} de Nova Hampshire claramente inclui muitas pessoas com uma atitude socialmente tolerante as quais seria difícil rotular de ``direita''). Os anarquistas terão boas razões para endossar ambos os movimentos (com a ressalva, é claro, de que nenhum deles vá longe demais). Mas quem pensa sobre secessão no contexto americano obviamente se lembrará que alguns -- não todos -- pedidos de secessão às vezes andavam lado a lado com o apoio à continuação da escravidão em meados do século XIX.

A experiência americana pode ser vista como trazendo à tona um problema com os movimentos de secessão: um grupo de pessoas que buscam fugir do que experienciam como opressão nas mãos de uma unidade política maior é perfeitamente capaz dele mesmo ser opressor. De fato, um movimento de secessão organizado em torno do desejo de independência de um grupo étnico ou religioso pode levar à criação de um estado para o qual a solidariedade a esse grupo é o principal. Grupos que são minorias dentro do recém-criado estado podem agora eles próprios desejar independência porque têm sido alvo de perseguição, discriminação ou até mesmo genocídio. E o próprio compromisso com a solidariedade que é o coração do novo estado (do ponto de vista da maioria) pode fazer com que seja difícil que os governantes tolerem outra secessão: o que era aceitável para eles pode não ser, do ponto de vista deles, aceitável para os outros.

Mas esse tipo de preocupação não deve ofuscar o fato de que a secessão pode servir à causa da liberdade. É importante enfatizar, por exemplo, que alguns abolicionistas na era pré-Guerra Civil argumentaram que o \emph{Norte} deveria se separar para que os governos estaduais do norte não tivessem que cooperar com estados escravagistas ao, por exemplo, aplicar leis de escravos fugitivos.

De qualquer forma, o apoio à independência de, digamos, Vermont e Nova Hampshire (ou, aliás, do Havaí, Alasca, Califórnia ou qualquer outro estado) deve ser facilmente aceito pelos anarquistas. Mas está claro que os movimentos de secessão se apresentam em todas as formas e tamanhos. Enquanto anarquistas nunca devem apoiar o uso da força para impedir uma secessão, eles também devem estar dispostos (moral, e talvez financeira e fisicamente) a apoiar medidas voluntárias e não estatais (as medidas baseadas no estado são muito propensas a concordar com o imperialismo e o aumento do poder do estado e a exigir a cobrança contínua de tributos) para conter e, se necessário, destituir separatistas (e outros) que promovem a escravidão, genocídio ou outros tipos de tirania.

\subsection*{Crie instituições alternativas}

Não faz sentido esperar que o estado perceba o fato de que ele é injusto e irrelevante. Existem muitos interesses pessoais que têm toda motivação para querer mantê-lo. Se um grupo de anarquistas bem-intencionados assumisse controle do aparato estatal, seria de se esperar uma autodisciplina quase sobre-humana deles se mesmo eles não ficassem tentados a se aproveitar do vasto poder do estado ``apenas para consertar alguns problemas''. Reivindicar o poder do estado não é um caminho muito realista para a anarquia. Uma coisa é se envolver na política eleitoral de fora, por assim dizer -- para restringir e, se possível, redirecionar o estado. Mas outra é sustentar a ilusão de que manter o estado sob o controle de pessoas boas é compatível com a autêntica libertação. O estado é completamente podre por dentro. O problema não é que o estado é administrado por pessoas que são inerentemente más; o problema é que a posse do poder estatal cria incentivos para que \emph{qualquer um} que o exerça abuse dele.

Então faz mais sentido, independentemente dos projetos paralelos que possamos ter no mundo do estado, que foquemos na criação de instituições alternativas que simplesmente tornem o estado irrelevante. Algumas dessas instituições são vitais para privar o estado do poder. Outras são importantes como formas de limpar a bagunça que o estado fez. Outras são ainda cruciais devido à sua capacidade de demonstrar que o estado não é necessário. Os \emph{Local Exchange Trading Systems} (LETSs), por exemplo, que envolvem troca de bens e serviços sem moeda, podem iniciar um movimento fora do sistema bancário controlado pelo estado, um movimento que é vital tanto porque pode fornecer segurança financeira quanto porque pode ajudar a garantir a privacidade. A criação de sistemas anônimos de transação online pode desempenhar o mesmo papel, assim como o desenvolvimento de novas moedas não estatais lastreadas em \emph{commodities}. Grupos como o \emph{Food Not Bombs} podem fornecer serviços básicos de baixo custo e alto valor para pessoas economicamente vulneráveis nas nossas comunidades, demonstrando que os serviços estatais administrativos, ineficientes e manipuladores, não são a única opção.

Quanto mais as pessoas podem confiar umas nas outras, mais podem ignorar o estado e as empresas que ele sustenta. O desenvolvimento de instituições não estatais que garantirão a estabilidade das nossas comunidades em uma sociedade sem estado é uma maneira de -- para usar a frase que ficou famosa pelos \emph{Industrial Workers of the World} -- ``construir a nova sociedade das ruínas da velha''.

\subsection*{Participe da contraeconomia?}

Alguns anarquistas optam por se envolver na contraeconomia -- a economia (do dom e de troca) não violenta e clandestina que funciona violando as regras do estado. Eles vendem versões falsificadas de medicamentos patenteados; possibilitam que aqueles sujeitos a governos autoritários acessem sites da Internet que os governos desejam impedí-los de visitar; facilitam a importação de bens sem o pagamento de tarifas de importação; ou ajudam imigrantes a atravessar postos de fronteira. Anarquistas que participam desse tipo de atividade obviamente correm riscos significativos; mas muitos deles consideram que o que estão fazendo vale a pena não apenas devido às recompensas financeiras que eles às vezes podem receber, mas também devido à oportunidade que seu trabalho oferece de ajudar pessoas necessitadas e de enfraquecer estruturas autoritárias, injustas.

\subsection*{Participe de protestos pacíficos}

O protesto é uma clássica maneira de contestar um governo injusto. As consequências pessoais de participar dos protestos de Tiananmen foram devastadoras para muitas pessoas, mas a escolha delas de confrontar o estado a qualquer custo tornou a sua coragem e integridade, e a brutalidade do regime que se opôs a elas, tanto inegáveis quanto inesquecíveis. A desobediência civil também pode ser usada para chamar fortemente a atenção para a injustiça das ações do estado: pense, por exemplo, na obra heróica dos \emph{Freedom Riders} que se sujeitaram ao grande risco de violência ao se oporem às proibições de Jim Crow ao serviço de ônibus [racialmente] integrados. 

Se as pessoas puderem ser mobilizadas para se opor a abusos estatais como guerras imperiais, sua visível oposição combinada, expressa por meio de atos públicos de protesto, pode energizar outras pessoas que não estão elas mesmas protestando, mas que se opõem ao que o estado está fazendo, e desafiar os funcionários do estado a confrontar a realidade dos abusos do estado e a seriedade da oposição pública a esses abusos. Como anarquista, você terá maior impacto, obviamente, se for capaz de protestar contra a injustiça em coalizão com uma variedade de pessoas de todo o espectro ideológico, pessoas cuja presença aumentará o número de manifestantes que você busca reunir e deixará claro para os oficiais estatais que as ações deles não estão sendo contestadas por apenas um pequeno grupo de radicais.

Não há nada de errado em usar a força para se defender ou defender outra pessoa contra um ataque injusto. Mas \emph{nada} justifica atacar o corpo ou a mente de outra pessoa com o propósito de prejudicá-la ou, instrumentalmente, para atingir algum outro objetivo, e nunca é razoável causar dano aos bens de outra pessoa quando você não estaria disposto a aceitar danos aos seus próprios se os papéis fossem invertidos. É sempre crucial se certificar de que você não provoque ataques; é principalmente importante fazê-lo quando você estiver protestando. Os analistas da mídia tradicional e outros oponentes do anarquismo caracteristicamente consideram os anarquistas como criminosos caóticos e violentos. É de sua responsabilidade mostrar a eles que isso não é verdade, que você não está no mesmo plano moral dos provocadores que buscam instigá-lo a brigas e que você é autodisciplinado o suficiente para dizer \emph{não} à raiva e à violência retaliatória. Lembre-se de que você está protestando não por medo ou raiva, mas por amor e paixão pela liberdade.

\subsection*{Apresente soluções alternativas}

Uma das melhores maneiras de atrair pessoas para a causa anarquista é demonstrar que sociedades sem estado podem funcionar na prática. Projetos de demonstração podem simultaneamente desempenhar a função prática e imediata de fornecer às pessoas alternativas à vida sob o domínio do estado \emph{e} deixar claro aos estados e indivíduos que o domínio do estado não é a única alternativa disponível.

Estados reivindicam toda, ou quase toda, terra firme em nosso planeta. É provável que seja difícil ou impossível reivindicar, de um estado existente, o território necessário para criar uma sociedade sem estado. Mas existem diversas alternativas criativas a fazer uso das terras reivindicadas pelo estado como o local de tal sociedade. Uma das opções mais interessantes envolve a criação de comunidades sem estado no oceano -- ``\emph{seasteading}''. Uma coisa é falar sobre a vida sem o estado -- mas outra bem diferente é mostrar às pessoas como ela pode ser.

É provável que isso seja ameaçador para alguns estados, da mesma forma que o estabelecimento de uma comunidade agrícola interracial por Clarence Jordan em uma época de segregação na Geórgia provocou respostas violentas de brancos que não estavam sendo forçados por Jordan a interagir com negros. O simples fato de que Jordan estava demonstrando que outra forma de conviver era \emph{possível} já foi o suficiente para enfurecer as pessoas: afinal, os sistemas autoritários (como o estado) tiram muito da sua força da ilusão de que eles são inevitáveis. Projetos de demonstração deixam claro que o estado é tudo menos inevitável.

Então as pessoas envolvidas na demonstração da possibilidade da anarquia precisam estar cientes de que elas podem provocar respostas violentas. Mas elas também precisam estar cientes de que essa pode ser uma das maneiras mais práticas e interessantes de fundamentar a anarquia.

\section{Em direção da surpresa}

Juntos, podemos ajudar a criar um mundo em que pessoas livres possam viver juntas em comunidades livres, vibrantes e criativas. Esse é o mundo que os anarquistas desejam: não um mundo em que a violência caótica tome a liberdade e a segurança das pessoas (o mundo que alguns comentaristas maliciosos parecem acreditar que os anarquistas buscam), mas um mundo em que a ausência do domínio do estado crie uma impressionante variedade de maneiras empoderadoras, libertadoras de se ser humano; não um mundo governado por estados e as elites que os controlam e os usam para dominar, excluir e empobrecer, mas um mundo em que as pessoas comuns sejam livres para florescer.

Não podemos dizer exatamente como esse mundo será. Mas isso é porque ele ainda não existe. É um mundo que moldaremos, um mundo que está nos esperando para que o tragamos à existência. Esse mundo está além da violência do estado, além de seu apoio à hierarquia e ao empobrecimento, além de sua repressão das diferenças e da sua sufocante eliminação dos ofensivos e dos dissidentes. Não é um mundo perfeito -- ele ainda será habitado por seres humanos, por você e eu. Mas é um mundo melhor, um mundo mais livre, mais pacífico, mais humano do que aquele em que vivemos hoje.

Vejo você lá.