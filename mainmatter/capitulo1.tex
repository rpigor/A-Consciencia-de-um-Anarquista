% Primeiro capítulo

\chapter{A Dissidência dos Governados}
\label{chap:1}

\lettrine[lines=2]{\textcolor{LettrineColor}{\textbf{E}}}{u sou anarquista porque} a reivindicação do estado à autoridade justificada é implausível. Ao contrário do que seus defensores afirmam, essa reivindicação não pode ser defendida com um apelo ao suposto consentimento daqueles que o estado tem a intenção de governar.

Os Estados Unidos têm uma teoria política oficial. É uma teoria incorporada nas familiares palavras do segundo parágrafo da Declaração de Independência:\footnoteA{Estou focando neste e em outros exemplos americanos por diversas razões. Sou americano e conheço melhor o cenário americano. Acredito que a teoria política ``oficial'' dos Estados Unidos, conforme refletida na Declaração, e a atitude instintiva de muitos americanos em relação à autoridade do estado são especialmente receptivas ao anarquismo. E, ao mesmo tempo, ironicamente, abusos do poder americano dentro do país e no exterior destacam o quão perigosos os estados podem ser.}

\begin{quote}
Consideramos estas verdades como evidentes por si mesmas, que todos os homens são criados iguais, dotados pelo Criador de certos direitos inalienáveis, que entre estes estão a vida, a liberdade e a procura da felicidade. Que a fim de assegurar esses direitos, governos são instituídos entre os homens, derivando seus justos poderes do consentimento dos governados; que, sempre que qualquer forma de governo se torne destrutiva de tais fins, cabe ao povo o direito de alterá-la ou aboli-la e instituir novo governo, baseando-o em tais princípios e organizando-lhe os poderes pela forma que lhe pareça mais conveniente para realizar-lhe a segurança e a felicidade.
\end{quote}

Observe a frase central: os governos adquirem ``seus justos poderes do consentimento dos governados''. Acredito que isso aponta para a visão do poder estatal que muitos americanos automaticamente aceitariam. E há algo muito plausível nisso.

Afinal, as pessoas que formulam e implementam as decisões do governo são apenas isso -- pessoas. Se, como a Declaração também insiste, todas as pessoas são iguais em valor moral e direitos morais, então ninguém -- nenhum imperador ou rei ou príncipe, nenhum papa ou lama ou imame, mas também nenhum presidente ou senador ou governador -- tem um direito natural de governar. Como ninguém tem o direito natural de governar, qualquer reivindicação de governo é inerentemente suspeita. Então a igualdade moral básica das pessoas que a Declaração testifica cria uma presunção de anarquia. Se as pessoas são moralmente iguais, cabe à pessoa que afirma que alguém tem autoridade sobre os outros mostrar o porquê. De onde essa autoridade deveria vir? O que deveria fundamentá-la?

\section{Consentimento e autoridade}

A Declaração tem uma resposta para essa questão: uma autoridade governamental pode ter o direito de me governar \emph{se eu der a ela esse direito}. Ninguém é naturalmente um governante; mas, sugere a Declaração, alguém poderia adquirir a autoridade de um governante se o povo que esse alguém deveria governar \emph{consentir}.

Mas seria difícil apontar para qualquer estado existente cuja autoridade se firma no consentimento real dos governados. Você consentiu com a autoridade do estado em cujo território vive -- e comunicou seu consentimento às autoridades? Seus amigos o fizeram? Você conhece alguém que o fez? Não é surpreendente, acredito eu, que, em uma pesquisa de julho de 2010, 62\% dos americanos disseram que o governo dos EUA não tinha o consentimento dos governados, enquanto outros 15\% disseram que não tinham certeza. (Sete a cada dez dos pesquisados também relataram acreditar que o estado ``e as Grandes Empresas \ldots\ [estavam] no mesmo time'' -- aliados contra as pessoas comuns.)\footnoteA{``23\% Say U.S. Government Has the Consent of the Governed,'' \emph{Rasmussen Reports} (Rasmussen Reports, July 16, 2010) <\url{http://www.rasmussenreports.com/public_content/politics/general_politics/july_2010/23_say_u_s_government_has_the_consent_of_the_governed}> (July 16, 2010).}

\section{Votar como consentimento}

Algumas pessoas argumentarão, é claro, que você o fez exatamente ao votar. Mas você realmente colocou seu selo de aprovação no estado só porque optou por votar em suas eleições? Não é óbvio que você o tenha feito.

Suponha que você more em uma pequena cidade invadida por um grupo de bandidos. Os bandidos, podemos supor, são um bando ativo. Nem todos vão morar na \emph{sua} cidade; em vez disso, eles querem coletar tributo de várias comunidades vizinhas. Mas eles pretendem ocupar a sua vila para manter todos na linha. Para deixar clara sua benevolência -- e para ajudar a seduzir você e seus conterrâneos a apoiar o governo deles -- eles anunciam que você poderá escolher entre dois subordinados do chefe dos bandidos, Jean e Chris. Um deles governará a sua vila, mas você escolherá quem. Chris tem ataques de fúria, enquanto Jean tende a ser mais calmo e agradável. Então você e a maioria de seus companheiros da vila expressam seu apoio a Jean. Há alguma razão para pensar que, ao escolher Jean, você endossou a ocupação dos bandidos em sua vila? Dada uma escolha forçada, você selecionou a opção com maior probabilidade de beneficiar a vila, mas certamente fazer isso não é o mesmo que apoiar a presença dos bandidos.

Certamente o mesmo é verdade quando você decide se vai votar em uma eleição nacional, estadual ou local. Sua escolha de votar fornece uma boa evidência (se não uma evidência irrefutável) de que você prefere o candidato em quem votou em relação aos outros. Mas não fornece nenhuma razão particularmente boa para pensar que você deseja ser governado por um dos candidatos -- ou, na verdade, que deseja ser governado por alguém sequer.

\section{Imobilidade como consentimento}

De acordo com outro argumento estatista, simplesmente permanecer dentro do território de um estado de alguma forma constitui consentimento à sua autoridade. Mas não há nenhuma razão óbvia para isso. Certamente, permanecer no território reivindicado por um determinado estado não é interpretado da maneira mais natural como um sinal de apoio à autoridade do estado. Talvez eu permaneça lá porque as oportunidades de trabalho são abundantes, ou porque meus amigos estão lá, ou porque gosto do estilo da arquitetura. E talvez eu não me mude de lá porque gangues de bandidos parecem estar no comando de todos os outros lugares. Não é nada óbvio que ficar parado comunicaria a um observador razoável a mensagem de que provavelmente concordei com a autoridade do estado. O que exatamente na minha permanência deveria comunicar a mensagem de que aceito a autoridade do estado?

Bem, talvez o estado fixe sinalizações em todo o seu território com a seguinte frase: ``Quem permanecer por mais de vinte e quatro horas dentro do território mostrado nesta sinalização e marcado por várias placas de sinalização semelhantes, assim sinaliza consentimento à autoridade do Soberano Reino da Bozarkia.'' Se ele o fizer, e daí? Existem muitas razões, como sugeri, pelas quais as pessoas podem ficar exatamente onde estão que não seja por seu consentimento à autoridade de Bozarkia. Ficar parado não \emph{sinaliza} consentimento. Bozarkia precisa alegar que isso \emph{constitui} consentimento.

Os governantes da Bozarkia poderiam razoavelmente alegar que isso constituiria consentimento à sua autoridade somente se eles já \emph{tivessem} autoridade legítima sobre o território em que as sinalizações foram fixadas. Se esse fosse o caso, então, pelo menos em algumas circunstâncias, talvez eles pudessem legitimamente insistir que as pessoas deixassem o território. Nesse caso, as pessoas que permanecem podem estar agindo de má fé se escolherem permanecer em condições diferentes das estabelecidas pelos governantes de Bozarkia. (Mesmo então, se eles deixassem claro que não consentiram, e as autoridades bozarkianas os deixassem permanecer de qualquer maneira, o argumento das autoridades perderia muito da sua força.) Mas se a suposição é de que um estado com autoridade sobre um determinado pedaço de território pode insistir que as pessoas ou façam o que ele diz ou deixem o território, há um problema bastante óbvio nesse caso: o que Bozarkia deveria estar tentando fazer nesse exemplo é \emph{estabelecer} a sua autoridade. Um procedimento para estabelecer a autoridade do estado que assume que o estado já \emph{possui} autoridade não demonstra muita coisa.

Os aspirantes a governantes de Bozarkia podem ter o direito de exigir obediência das pessoas já obrigadas a aceitar a sua autoridade; mas se eles possuem alguma autoridade ou não é exatamente o ponto em questão. É fácil imaginar que minha fixação de sinalizações afirme a minha autoridade e insista que as pessoas que permanecem em suas casas aceitam minha autoridade ao fazê-lo, em todo o bairro em que moro. Se as pessoas da vizinhança não conseguissem derrubá-las e não se mexessem, estariam elas aceitando minha autoridade como governante local? Certamente não, e elas claramente estariam em seus direitos ao simplesmente me ignorar. Isso é porque eu não tenho autoridade para insistir que elas aceitem meu governo ou deixem suas casas \emph{para início de conversa}.

A ideia de que permanecer no território de um estado equivale a consentir à sua autoridade não funciona. Permanecer não \emph{sinaliza} consentimento; fazê-lo comunica muitos significados possíveis. E permanecer equivale a \emph{aceitar} a autoridade do estado apenas se já tivermos estabelecido que o estado realmente \emph{possui} autoridade para início de conversa.

Há boas razões para pensar que muitas pessoas não consentiram com a autoridade do estado. Mas \emph{há} pessoas que apoiam a autoridade do estado e querem que o resto de nós se alinhe. Elas insistem que devemos obediência ao estado. Teriam elas algum argumento restante uma vez que mostramos que não consentimos com a autoridade do estado (ou que nunca consentimos para início de conversa, ou que retiramos nosso consentimento depois de perceber o desastre que o estado realmente é)?

\section{Consentimento como exigido para a justiça}

Estatistas podem ser engenhosos. Existem outros argumentos que eles podem oferecer para mostrar que o estado \emph{é} realmente legítimo. Por exemplo: ``Talvez você tenha retirado seu consentimento'', eles podem dizer. ``Talvez você nunca tenha consentido para início de conversa. Mas é injusto da sua parte \emph{não} consentir.''

Obviamente, há uma grande diferença entre ser obrigado a um acordo que eu realmente fiz e ser obrigado a um acordo que alguém pensa que eu \emph{deveria} ter feito. Mas ignore isso por enquanto. Por que seria injusto da minha parte desconsiderar as ordens do estado?

O estatista pode tentar explicar desta forma: ``A maioria decidiu que uma determinada demanda deve ser provida. Você é obrigado a colaborar, mesmo que discorde. Caso contrário, você estaria alegando que a sua posição minoritária deve governar, e não a opinião da maioria.''

Como outros argumentos estatistas, esse não é muito bom. É claro, o anarquista \emph{não está} dizendo tal coisa: o anarquista não acredita que \emph{alguém} deva governar. E o argumento assume, novamente, exatamente o que deve provar. Se o estado realmente fosse um empreendimento cooperativo que todos nós tivéssemos escolhido participar, e se tivéssemos consentido com um conjunto de regras básicas, incluindo a regra da maioria, então seria injusto recusar essas regras apenas porque elas levaram a resultados que não gostamos. Mas a questão é precisamente se \emph{concordamos} com as regras básicas. Muitos de nós não o fizemos.

\section{Consentimento como exigido por aceitar benefícios}

Outro argumento estatista sugere que, se aceitamos benefícios do estado, nada mais justo que obedecê-lo. Mas esse argumento tenta provar demais.\footnoteNT{Nota do Tradutor [N.T.]: No original, \emph{prove entirely too much}. No contexto da filosofia, diz-se que um argumento ``\emph{proves too much}'' (ou, nesta tradução, ``prova demais'') quando ele também pode ser usado para provar uma proposição que o proponente toma como falsa.}

Ele não fornece nenhum apoio para a obediência-ao-estado-em-geral, mas apenas para qualquer obediência que possa ser necessária para garantir a provisão de quaisquer benefícios que alguém venha a receber do estado. Suponha que eu gratuitamente aceite os serviços de saúde providos pelo estado, por exemplo. Talvez isso pudesse tornar injusto que eu evitasse contribuir para o sistema de saúde do estado. Mas não fornece nenhuma razão em específico para que eu coopere com o estado quando ele tenta impedir que outras pessoas leiam ou digam coisas que as autoridades não gostam ou quando ele me alista nas forças armadas.

O tipo de obediência envolvida aqui é claramente apenas o tipo de obediência envolvida em fornecer apoio financeiro para este ou aquele esquema patrocinado pelo estado. Mas a realidade é que não tenho nenhuma escolha significativa para fazer isso de tal forma. Se eu não obedecer quando o estado exige dinheiro de mim, tenho boas razões para temer que ele irá confiscar meus bens ou me prender. Uma vez que já estou sendo tributado de forma a apoiar uma variedade de serviços providos pelo estado -- tanto os que eu talvez queira que sejam providos, quanto os que eu definitivamente \emph{não} quero que sejam providos --, não há nenhuma razão para não tirar proveito dos benefícios relevantes. Mas isso dificilmente significa que estou consentindo com a operação do sistema onde minhas contribuições são tomadas de mim e dos outros.

Aceitar os benefícios providos pelo estado não significa que você está endossando o estado como um todo. Já que você está sendo forçado a pagar pelos programas que provêm esses benefícios, muitas vezes você nem irá se deparar com a questão de se seria \emph{justo} você não pagá-los, a não ser que você esteja disposto a enfrentar a prisão e a apropriação de seus bens pelo estado. E se você \emph{conseguir} evitar pagar o dinheiro que o estado deseja que você pague, não há do que você se sentir culpado desde que recuse os benefícios que ele oferece. Muitos anarquistas insistiriam que, enquanto uma organização (incluindo o estado) reivindicar tributos sob a mira de uma arma, ela não merece nada, mesmo pelos serviços que ela fornece. Mas mesmo que você deva algo em troca de serviços específicos que recebe voluntariamente, é difícil argumentar que você deve ao estado o pagamento por qualquer benefício que você \emph{não} aceite voluntariamente. E mesmo que você deva ao estado algum tipo de compensação por um benefício específico, aceitar esse benefício não o compromete a aceitar qualquer dever generalizado de \emph{obediência} ao estado.

Para deixar a questão ainda mais turva: mesmo que você \emph{aceite} benefícios providos pelo estado, e mesmo que deva algo em troca como resultado, ainda não está claro que você deve algo ao próprio \emph{estado}. Isso porque não foi \emph{realmente} o estado quem financiou os benefícios que você recebeu; e sim os pagadores de imposto comuns. Já que outros podem ser tributados com taxas mais altas porque você não paga pelos serviços que escolhe receber, você pode estar tratando \emph{eles} injustamente -- mas isso não significa que você tenha qualquer obrigação particular para com o próprio estado. Isso não torna legítima as exigências do estado por tributos -- especialmente a reivindicação do estado da sua cooperação ou não interferência quando ele tenta fazer qualquer coisa \emph{além} de financiar quaisquer benefícios que você possa ter aceito de bom grado.

Suponha que os bandidos que mencionei anteriormente exijam uma certa quantia de tributo da sua cidade. Se for virtualmente garantido que mais será roubado dos outros porque você consegue evitar que os bandidos roubem de \emph{você}, pode haver circunstâncias em que os \emph{outros} teriam boas razões para vê-lo como tendo os sobrecarregado com fardos injustos. Mas você ainda não deveria nada \emph{aos bandidos em si}.

\section{Limites do consentimento}

É provável que não exista maneira de sinalizar consentimento à autoridade do estado. Afinal, pode-se esperar que o estado o puna -- tome suas coisas, o ataque fisicamente ou o prenda -- se você não obedecer as suas imposições. Então, só porque você coopera com oficiais do estado, até mesmo se você declara publicamente o seu apoio ao estado, ninguém está realmente justificado a acreditar que você realmente consentiu livremente com a autoridade do estado. Ninguém pode legitimamente exigir de você uma promessa feita sob a mira de uma arma. Sempre corremos o risco de sermos colocados sob a mira de uma arma pelo estado. Então ninguém pode razoavelmente exigir de qualquer um de nós alguma promessa que deveríamos ter feito ao estado, já que qualquer promessa que poderíamos ter feito teria sido feita sob coação.

Além disso, se não houver uma maneira real de optar por \emph{sair} [\emph{opt out}], se o estado não fornecer uma maneira de permitir que as pessoas \emph{não consintam com sua autoridade} enquanto permanecem dentro do território que ele reivindica, então não há realmente nenhuma forma de optar por \emph{aceitar} [\emph{opt in}] também. O estado nos trata como se tivéssemos consentido com a sua autoridade em tudo que fazemos, então nem sequer realmente nos é dada a escolha de consentir. E é difícil levar a sério a ideia de que seu consentimento \emph{significa} alguma coisa, que esse consentimento deve obrigá-lo de alguma forma, se você não tem a opção de não consentir.

\section{Apenas diga ``Não''}

Muitas pessoas, talvez a maioria delas, não consentiram deliberadamente com a autoridade do estado. E não há nenhuma razão geral para supor que a participação delas em várias atividades relacionadas ao estado ou a aceitação delas de vários benefícios relacionados ao estado as comprometa a consentir com isso. Na verdade, é bem possível que seja \emph{impossível} sinalizar consentimento à autoridade do estado de maneira confiável, dada a contínua ameaça de violência do estado contra as pessoas que não cooperam com as autoridades. Se a autoridade legítima depende do consentimento, parece que o estado provavelmente não é legítimo. E isso significa que você provavelmente não tem nenhum dever geral de obedecê-lo.