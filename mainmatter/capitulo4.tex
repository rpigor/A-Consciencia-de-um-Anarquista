% Quarto capítulo

\chapter{Estado, Guerras e Império}
\label{chap:4}

\lettrine[lines=2]{\textcolor{LettrineColor}{\textbf{E}}}{u sou anarquista porque} os estados matam e conquistam. Seus exércitos militares causam destruições inacreditáveis. E por meio de uma combinação de força militar, uso de truques sujos e aplicação de pressão econômica, eles dominam sociedades menos poderosas.

Tomemos um exemplo óbvio bem perto de nós: as guerras declaradas e não declaradas do governo dos EUA são muitas vezes exercícios de expansão imperial injusta. A construção de um império assume formas militares, políticas e econômicas: as guerras do governo são frequentemente \emph{imperiais} porque servem para estender o seu poder militar ao redor do mundo, criando novas alianças, novas oportunidades de localizar bases e tropas a todo momento, deixando claro quem manda para qualquer um que passe dos limites, e porque muitas vezes parecem designadas a estender a influência das grandes empresas americanas.

De maneira semelhante, as guerras do governo dos EUA são inúteis porque não tornam os americanos mais seguros. As intervenções militares na Coreia, Vietnã, Líbano, Granada, Iraque, Bálcãs, Somália e Afeganistão não serviram para proteger os americanos contra ataques estrangeiros.\footnoteA{A invasão do Afeganistão foi realizada apesar do regime do Talibã ter tentado conter Osama bin Laden e impedi-lo de realizar ataques fora do Afeganistão. O regime parece ter feito várias ofertas para deportá-lo ou entregá-lo para julgamento. Veja, por exemplo, Gareth Porter, ``Taliban Regime Pressed bin Laden on Anti-US Terror,'' \emph{AntiWar.Com} (Randolph Bourne Institute, Feb. 12, 2010) <\url{http://original.antiwar.com/porter/2010/02/11/taliban-regime-pressed-bin-ladenon-anti-us-terror}> (July 5, 2010); Alexander Cockburn e Jeffrey St. Clair, ``How Bush Was Offered Bin Laden and Blew It,'' \emph{Counterpunch} (n.p., Nov. 1, 2004) <\url{http://www.counterpunch.org/cockburn11012004.html}> (July 5, 2010).}

Elas geram cada vez mais hostilidade contra o governo dos EUA e, muitas vezes, violência contra os americanos. Elas não espalham os ideais americanos -- elas os mancham: o vasto ``dano colateral'' contra não combatentes não ajuda; e as intervenções militares do governo dos EUA normalmente servem não para criar ou fortalecer sociedades livres, mas para fortalecer os regimes autoritários que são mais populares entre os tomadores de decisões em Washington devido às suas atitudes cooperativas em vez de algum compromisso genuíno com a liberdade e prosperidade do povo de suas sociedades.

\section{Guerras e igualdade moral}

A ameaça de violência é a principal fonte do poder interno do estado. Mas estados repetidamente se envolvem em atos de violência fora das suas supostas fronteiras.

A mais básica convicção moral anarquista, creio eu, é que ninguém tem um passe livre no que diz respeito à moralidade. Se é injusto que você faça algo em um dado conjunto de circunstâncias, então é injusto que eu faça a mesma coisa em circunstâncias relevantemente semelhantes. Em contrapartida, o estado parece operar sob a premissa que, uma vez em posse do tipo de mandato correto, pessoas podem moralmente fazer todo tipo de coisas que não poderiam fazer em outro caso. Para os anarquistas, porém, o fato de que o estado ordenou que alguém fizesse algo não muda, \emph{per se}, as circunstâncias morais em que ela age. Assim, anarquistas tenderão a concordar com a ácida observação de Mark Twain de que ``[t]oda guerra'' (e aqui ele claramente se refere aos tipos de guerras agressivas que os estados mais frequentemente se envolvem) ``há de ser apenas a matança de estranhos contra os quais você não tem inimizade pessoal alguma; estranhos que, em outras circunstâncias, você ajudaria se encontrasse em necessidade, e que o ajudariam se você precisasse.'' Condutas que seriam de outra forma imoral não se tornam aceitáveis quando são ordenada pelo estado.

É por isso que anarquistas dizem um decisivo \emph{não} às guerras. Alguns anarquistas são pacifistas; mas a maioria acredita que o uso da força para defender a si mesmo ou aos outros contra a violência é perfeitamente razoável. Os estados, no entanto, não reconhecem tal limitação: eles travam guerras para dominar, ameaçar, retaliar e conquistar. E esse tipo de guerra não faz sentido de uma perspectiva anarquista. Se agentes estatais não obtêm nenhum tipo de isenção das exigências da justiça, o que os dá o direito de atacar e destruir quando pessoas comuns não podem fazer o mesmo? Reagiríamos horrorizados se pessoas comuns começassem a atacar seus vizinhos, ocupando suas casas, fugindo com seus pertences, os confinando e exigindo tributo deles -- mas os estados fazem esses tipos de coisas o tempo todo. Uma vez que não somos iludidos pela auto-apresentação enganosa do estado, uma vez que percebemos que os agentes estatais são apenas, por assim dizer, seus vizinhos armados, pessoas comuns sem maior direito à autoridade do que você, podemos ver que não há razão para aceitar as guerras agressivas do estado ou considerá-las como algo além de exercícios de brutalidade.

\section{Dissipando as névoas da guerra}

Existem boas razões para se opor à máquina estatal de guerra. Mais fundamentalmente, as guerras matam. Os estados escravizam potenciais soldados e cobram tributo das pessoas a fim de financiar suas máquinas de guerra. As guerras dão aos estados desculpas para expandir o seu poder. Guerras levam a mais guerras. E elas fragmentam famílias e comunidades.

\subsection*{Guerras matam}

Para reafirmar o óbvio, a violência estatal mata. Antes do século XX, a violência estatal (que inclui mais do que guerras declaradas, mas que terá sido resultado ou de militares ou de outros agindo sob as ordens do estado) provavelmente custou entre 89.000.000 e 260.000.000 de vidas.\footnoteA{Rudolph Joseph Rummel, ``Statistics of Pre-20th Century Democide: Estimates, Calculations, and Sources,'' \emph{Statistics of Democide} (Freedom, Democide, War, 1997) <\url{http://www.mega.nu/ampp/rummel/sod.chap2.htm}> (July 1, 2010). O que está disponível online é o ``manuscrito editado pré-publicado'' do \emph{Statistics of Democide: Genocide and Mass Murder Since 1900} (Charlottesville, VA: Center for National Security Law; Piscataway, NJ: Transaction 1997).} Estimativas razoáveis colocam o número total de mortes relacionadas a guerras apenas no século XX em mais de -- possivelmente bem mais de -- cem milhões de pessoas. Talvez quinze milhões de pessoas morreram na Primeira Guerra Mundial ou como resultado dela; a Segunda Guerra Mundial pode ter ceifado cerca de cinquenta e cinco milhões de vidas.\footnoteA{[Matthew White,] ``Source List and Detailed Death Tolls for the Twentieth Century Hemoclysm,'' \emph{Historical Atlasof the Twentieth Century} (n.p., Feb. 12, 2005) <\url{http://users.erols.com/mwhite28/warstat1.htm}> (July 1, 2010).}

Para focar em eventos mais recentes: as mortes documentadas de não combatentes no Iraque desde a invasão de 2003 totalizam em torno de 100.000. Mais de 31.000 militares do governo dos EUA foram feridos em combate; mais de 4.200 militares, e mais de 1.100 mercenários, foram mortos.

\subsection*{Guerras levam à extração de tributos}

Guerras geram custos crescentes e alimentam uma burocracia estatal inchada. Não é fácil estimar os custos de todas as guerras ao longo da história. Mas a Primeira Guerra Mundial custou cerca de \$2,6 trilhões de dólares de hoje aos seus participantes,\footnoteA{[John Simkin,] ``Financial Cost of the First World War,'' \emph{Spartacus Educational} (Spartacus Educational n.d.) <\url{http://www.spartacus.schoolnet.co.uk/FWWcosts.htm}> (Jan. 27, 2011). Utilizei dados do CPI [\emph{Consumer Price Index}] para ajustar o total relevante à inflação.} enquanto a conta (com a inflação ajustada) da Segunda Guerra Mundial parece ter girado em torno de \$3,3 trilhões.\footnoteA{Ajustei os números (já ajustados à inflação) de ``Three World War Statistics Compared,'' \emph{Three World Wars} (Three World Wars,Jan. 26, 2009) <\url{http://www.threeworldwars.com/overview.htm}> (Jan. 27, 2011) usando dados do CPI (ao usar essas estatísticas, não quero endossar a análise específica do site das questões históricas e políticas relevantes).} Um analista do \emph{Congressional Research Service} (CRS) estima que o governo dos Estados Unidos gastou \$341 bilhões de dólares de hoje na Guerra da Coreia e \$738 bilhões de dólares de hoje na Guerra do Vietnã (e isso ignora vários custos indiretos e não monetários, assim como custos financeiros que não foram bancados pelos americanos).

Não podemos ter certeza de exatamente quanto os americanos acabarão pagando pela guerra do Iraque e do Afeganistão e pelas melhorias de segurança em bases militares estrangeiras iniciadas desde os ataques de 11 de setembro de 2001. Mas uma estimativa recente do CRS sugere que as despesas aprovadas do ano fiscal de 2001 até meados do ano fiscal de 2010 totalizaram \$1.121 trilhões. Isso é aproximadamente \$400 por americano por ano, ou \$1.600 para uma família de quatro pessoas -- e lembre que os empreendimentos de guerra do governo dos EUA estão sendo pagos com dinheiro emprestado: a conta sairá ainda mais cara quando ela finalmente vencer.

Leve uma variedade de custos indiretos em consideração e o impacto no bolso dos americanos parece pior ainda. De fato, dois economistas, Linda Bilmes e o ganhador do Prêmio Nobel Joseph Stiglitz, argumentaram que os custos totais da guerra do Iraque e do Afeganistão podem chegar a pelo menos \$3 trilhões -- e provavelmente serão maiores do que isso.\footnoteA{Para as estimativas do CRS, veja Stephen Daggett, \emph{Costs of Major U.S. Wars}, CRS Report for Congress RS22926 (Washington, DC: Congressional Research Service, June 29, 2010) <\url{http://www.fas.org/sgp/crs/natsec/RS22926.pdf}> (Jan. 31, 2011); Amy Belasco, \emph{Statement of Amy Belasco, Specialist in U.S. Defense Policy and Budget, Congressional Research Service, before the House Budget Committee Hearing on the Growing Cost of the Iraq War, October 24, 2007} (Washington, DC: Congressional Research Service, Oct. 24, 2007) <\url{http://budget.house.gov/hearings/2007/10.24Belasco_testimony.pdf}> (July 1, 2010); Amy Belasco, \emph{The Cost of Iraq, Afghanistan, and Other Global War on Terror Operations Since 9/11}, CRS Report for Congress RL33110 (Washington, DC: Congressional Research Service, Sep. 2, 2010) <\url{http://www.fas.org/sgp/crs/natsec/RL33110.pdf}> (Jan. 27, 2011). A estimativa de três trilhões de dólares é de Joseph E. Stiglitz e Linda J. Bilmes, \emph{The Three Trillion Dollar War: The True Cost of the Iraq Conflict} (New York: Norton 2008). Em um artigo recente, Stiglitz e Bilmes sugerem que sua estimativa anterior pode ter sido muito baixa; veja ``The True Cost of the Iraq War: \$3 Trillion and Beyond,'' \emph{Washington Post} (The Washington Post Co., Sep. 5, 2010) <\url{http://www.washingtonpost.com/wpdyn/content/article/2010/09/03/AR2010090302200.html}> (Jan. 31, 2011).}

Barack Obama anunciou ``o fim das operações de combate'' no Iraque. Mas cerca de 50.000 soldados do governo dos EUA permanecem lá, e o presidente e os líderes do Congresso continuam comprometidos com uma guerra prolongada no Afeganistão. Não há fim à vista e as contas continuam aumentando.

George Bush e Dick Cheney talvez estejam afastados do cargo, mas o Partido da Guerra -- formado por pessoas, sejam quais forem suas filiações partidárias, que são a favor do uso da guerra para atingir os objetivos imperiais do estado -- ainda está no poder e ainda está gastando o que aspirantes a grandes sábios gostam de chamar de ``sangue e tesouro''.\footnoteNT{N.T.: \emph{blood and treasure}. Expressão idiomática que se refere aos grandes custos de vidas e riquezas.}

\subsection*{Guerras criam desculpas para o abuso de poder}

Guerras criam novas oportunidades para o exercício abusivo do poder. Os governos da época da Guerra da Independência Americana criaram programas de recrutamento e executaram desertores sem julgamento. Aspirantes a fomentadores de guerras aprovaram os \emph{Alien and Sedition Acts} no fim do século XVIII para suprimir críticas às políticas com o potencial de levar à guerra. Ambos os lados da Guerra Civil implementaram programas de recrutamento e impuseram penalidades criminais a dissidentes verbais. Durante a Primeira Guerra Mundial, Woodrow Wilson prometeu que ``uma mão firme de repressão severa'' seria usada contra oponentes supostamente ``desleais''; o governo dos Estados Unidos prendeu radicais domésticos em grande número, processando pessoas apenas por expressarem oposição à guerra. A Segunda Guerra Mundial forneceu aos líderes militares e políticos nos Estados Unidos desculpas para internar pessoas simplesmente em virtude das suas origens japonesas, sem qualquer apuração de riscos individuais, e a legislação anti-insubordinação forneceu cobertura para repressões à liberdade de expressão. Os processos e perseguições de esquerdistas sem nenhum real envolvimento em espionagem ou nos planos soviéticos de expansão imperial na época da Guerra Fria são dolorosamente bastante conhecidos. Grupos opositores ao aventureirismo do estado no sudeste asiático na época do Vietnã eram alvos de agências de aplicação da lei por infiltração e manipulação, e as pessoas eram processadas simplesmente pelo ato simbólico de queimar certificados de alistamento militar.

A guerra ainda fornece desculpas para abusos dos direitos humanos. Antes de George Bush anunciar uma ``guerra global ao terror'', você já tinha ouvido falar de afogamento simulado? Ou ``extradição forçada''? Já lhe passou pela cabeça que agentes do governo dos Estados Unidos forçariam prisioneiros indefesos a se sentirem como se estivessem se afogando? Você já pensou que agentes do governo dos Estados Unidos capturariam pessoas e as entregariam a agentes de governos estrangeiros -- que as espancariam e torturariam? Você já imaginou que o dinheiro que você pagou em impostos seria usado para criar ``prisões secretas'' ao redor do mundo, supostamente fora do alcance da Constituição, onde os prisioneiros poderiam ser detidos sem julgamento?

Desde que os Estados Unidos se envolveu em uma guerra perpétua e agressiva -- usando os militares para resolver o problema da aplicação da lei -- todos nós passamos a conhecer melhor os eufemismos ardilosos que o governo pode usar para encobrir a brutalidade.

Alguns dos piores abusos dos anos Bush podem ter chegado ao fim. Mas o discurso de posse de Barack Obama nos disse que a ``guerra ao terror'' de Bush ainda não chegou ao fim. O governo Obama aparentemente não tem a intenção de processar os torturadores da era Bush: a possibilidade teórica de julgamentos pode ainda existir, mas não há nenhuma chance realista que eles de fato ocorram. Existem algumas razões para pensar que suas recém-anunciadas regras impeçam a tortura por funcionários do governo dos EUA -- mas não necessariamente por estrangeiros em terras do governo dos EUA. O governo parece estar usando uma retórica cuidadosamente elaborada para se distanciar do péssimo histórico de direitos humanos de seu antecessor, enquanto ainda mantém suas próprias opções em aberto. 

E, é claro, o problema não termina dentro das fronteiras nacionais. Existe algo mais divertido do que esperar na fila de segurança do aeroporto para uma revista humilhante? Você gosta de saber que guardas estão impedindo seus amigos e familiares de atravessar os pontos de segurança com você até o portão de embarque? Você não está feliz em saber que os agentes do governo têm espionado as ligações dos celulares dos americanos? Ou emitido ``Cartas de Segurança Nacional'' exigindo informações privadas das pessoas -- enquanto as proíbe de revelar o fato de que as obteve?

A liberdade foi a primeira vítima da guerra não declarada do estado americano.

Como senador, Barack Obama apoiou a reautorização do \emph{USA PATRIOT Act}, apesar de anteriormente ter notado problemas de liberdades civis com a lei. E ele assinou o \emph{FISA Amendment Act}, que pretendia tornar legais as escutas telefônicas sem mandado (apenas fique feliz que o Congresso não pode mudar a Constituição) e deu às empresas de telecomunicações um passe livre para ajudar no programa de vigilância doméstica do governo Bush.

Guerras dão ao estado a desculpa de ocultar informações do público. Há não muito tempo, o Departamento de Justiça argumentou que, como um processo judicial envolvia segredos de estado, todo o processo deveria ser arquivado. Ele disse que o privilégio reivindicado pelo governo de manter segredos de estado poderia justificar que o próprio governo fosse impedido de ser processado.

Esse é o Departamento de Justiça do governo Obama.

Durante sua campanha presidencial, Obama criticou o governo Bush por sua interpretação abrangente do privilégio dos segredos de estado. Agora, seu Departamento de Justiça está tomando a mesma posição do seu predecessor da era Bush.

O governo afirma que precisa do privilégio dos segredos de estado e outras regras de sigilo para nos manter seguros de nossos adversários na ``guerra ao terror''. Mas o principal efeito das regras de sigilo é nos impedir de responsabilizar o estado. Elas facilitam que fraudes, violações das liberdades civis e tortura não sejam descobertas e reparadas.

Randolph Bourne observou notoriamente que ``a guerra é a saúde do estado''. O Partido da Guerra se utiliza da infindável ``guerra ao terror'' -- que pouco faz para manter os americanos seguros, mas que promove um intenso sentimento antiamericano ao colocar o governo dos EUA como um valentão global -- como desculpa para justificar o abuso dos direitos humanos, a deterioração da liberdade, o desperdício do nosso dinheiro e a expansão do poder executivo, ocultando abusos da vista ao apelar para o valor do sigilo de estado.

Em tempo de guerra, o estado busca silenciar ou marginalizar dissidentes, descartando aqueles que se opõem à história oficial como anti-patriotas ou mesmo traidores. Dissidentes são rotulados como perigosos, colocados em listas de observação, ridicularizados e perseguidos. Talvez eles não sejam presos como alguns foram durante a Primeira Guerra Mundial. Mas os comparsas e apologistas da grande mídia os tratam como idiotas, ingênuos e tão obviamente errados que eles na prática são silenciados -- suas vozes são inaudíveis para a maioria das pessoas comuns.

\subsection*{Violência leva a mais violência}

Guerras levam a ainda mais violência. A intervenção militar e manipulação política no Oriente Médio -- mais recentemente a guerra em duas frentes no Iraque e Afeganistão -- suscitaram uma forte antipatia pelo governo dos EUA em toda a região e todo mundo islâmico. As contínuas campanhas terroristas não refletem uma aversão mítica à decadência americana. Elas estão intensamente focadas no objetivo de expulsar da região os soldados e ``investimentos'' militares do governo dos EUA.

Declarações públicas identificadas como originando da rede terrorista Al-Qaeda repetidamente justificam os ataques terroristas como respostas a ações militares: respostas à contínua intervenção do governo dos EUA no Iraque, por exemplo, e à presença de suas forças armadas na Arábia Saudita.\footnoteA{Cp. Michael Scheuer, ``What If Osama Calls Obama's Bluff?,'' \emph{AntiWar.com} (Randolph Bourne Institute, June 9, 2009) <\url{http://original.antiwar.com/scheuer/2009/06/08/what-if-osama}> (July 5, 2010).} (Você se lembra quando a Secretária de Estado Madeleine Albright notoriamente disse que manter um embargo que levou à morte de crianças iraquianas ``valeu a pena'' porque reprimia Saddam Hussein? Você acha que isso estimulou o sentimento pró-americano no Iraque?)

Um dos réus no caso do atentado ao \emph{World Trade Center} em 1993 disse a mesma coisa. O aspirante a terrorista da \emph{Times Square} em 2010 novamente fez a mesma alegação.

Todo ataque intencional (ou indiscriminado) a não combatentes é errado. Não há exceções. Mas a violência terrorista dirigida a americanos é explicável. Não é fruto de um ódio profundo pelas liberdades americanas, nem faz parte de algum plano sinistro para dominar o mundo. É uma resposta direta a injustiças sentidas e à violência da guerra.

\subsection*{Guerras fragmentam famílias e comunidades}

Guerras fragmentam famílias de militares. A mobilidade que é exigida para fazer um exército permanente funcionar exerce uma enorme pressão sobre a vida familiar. Também torna as comunidades instáveis, pois os militares, com muito a contribuir para suas comunidades, são incapazes de criar as raízes necessárias para investir emocional e financeiramente em suas comunidades. Ironicamente, muitos daqueles mesmos ``conservadores'' que ostentam suas credenciais pró-família parecem não se preocupar com o fato de que seu apoio a uma instituição militar inchada tem um alto custo para muitas famílias no mundo real.\footnoteA{Veja Bill Kauffman, \emph{Ain't My America: The Long, Noble History of Antiwar Conservatism and Middle-American Anti-Imperialism} (New York: Metropolitan 2008).}

\section{Guerra e império}

Não há como escapar: quer os americanos gostem ou não, os Estados Unidos se tonaram um enorme império global. Obviamente, o imperialismo do governo dos EUA não se parece com, digamos, o imperialismo do governo britânico. De maneira geral, o governo dos EUA não tenta controlar diretamente milhões de quilômetros quadrados de território fora das fronteiras do território que ele reivindica como seu. Em vez disso, seja por meio de ação militar, manipulação política, apoio a conspiradores golpistas ou pressão econômica, ele incentiva a instalação e manutenção de regimes amigáveis aos seus interesses. Às vezes, esses regimes parecem relativamente livres, às vezes não. O que importa para as pessoas que controlam as rédeas do império é criar e manter uma rede de aliados que possam oferecer acesso a recursos e mercados, locais para bases militares e apoio a outros objetivos estratégicos.

Guerra e império andam lado a lado. A guerra é um instrumento de expansão imperial tão comum quanto é um meio de defesa. A história do crescimento do poder imperial é uma história não de alianças pacíficas, mas de guerra e conquista. Alexandre construiu o império macedônio não com a compreensão de filosofia que ele adquiriu de Aristóteles, mas com suas habilidades como um impiedoso general. Roma dominou o mundo mediterrâneo porque suas tropas suprimiram fontes alternativas de poder e permaneceram para manter a ordem e ameaçar com a violência. Espanha e Portugal lutaram, queimaram, roubaram e escravizaram em toda a América Latina. Exércitos instauraram e mantiveram a presença imperial da França na África Ocidental e no sudeste da Ásia. O sol nunca se pôs no império britânico porque as tropas britânicas conquistaram a Índia, mantiveram os povos nativos sob controle na América do Norte e África e patrulharam o mundo nos navios da inigualável Marinha Real.\footnoteNT{N.T.: A expressão ``o império no qual o sol nunca se põe'' retrata a ampla extensão territorial de determinados impérios, comumente em referência ao império britânico como aqui é o caso.}

Considere as últimas cinco décadas. E foque apenas nos Estados Unidos. O governo dos EUA é dificilmente a única potência militar do mundo. É dificilmente o único governo a invadir outros países além daquele que ele afirma governar. Mas tem sido o governo mais poderoso do mundo pelo menos desde o fim da Segunda Guerra Mundial, e isso significa que o que ele faz tem um impacto particularmente drástico em outras sociedades -- e deixa especialmente claro o que estados podem fazer, e provavelmente farão, com recursos militares.

O governo dos Estados Unidos gastou bilhões de dólares na guerra do Vietnã, onde morreram entre três e quatro milhões de vietnamitas e cerca de 60.000 americanos. Muitos americanos suspiraram de alívio quando a guerra do Vietnã terminou. Mas era óbvio que o militarismo e o imperialismo não tinham acabado. As consequências não intencionais do alcance imperial no sudeste da Ásia incluíram derramamento de sangue e opressão totalitária no Camboja. Poucos anos depois o governo Reagan estava de volta, apoiando bandidos que torturavam camponeses e estupravam e assassinavam freiras. Então, é claro, houve a primeira Guerra do Golfo e a intervenção imperialista nos Bálcãs disfarçadas de o que Noam Chomsky rotulou de ``o novo humanismo militar''. O governo Bush lançou uma invasão desastrosa e injusta, fundamentada em mentiras. E usou a sua declaração de guerra indeterminada contra o terrorismo para justificar a vigilância doméstica, a prisão por tempo indeterminado sem julgamento e a tortura.

Agora, o governo Obama está aumentando a presença militar do governo dos EUA no Afeganistão. Fala-se continuamente em atividade militar no Paquistão. Falcões em Washington ainda estão procurando desculpas para atacar o Irã. E os EUA ainda estão presos à problemática do Iraque.

\section{O principal partido da América: O Partido da Guerra}

Comentaristas têm tagarelado sobre uma suposta mudança do poder ``duro'' [\emph{``hard''}] para o poder ``suave'' [\emph{``soft''}]. Mas o poder do estado ainda é, em última instância, o poder de mutilar e matar, e as elites que querem usar o poder para estender sua influência econômica e política continuam a determinar a agenda. Elas se sentem em casa em ambos os principais partidos políticos americanos: a guerra não é apenas uma preocupação dos republicanos (Nixon, Reagan, Bush I, Bush II) ou dos democratas (Johnson, Clinton e agora Obama). Então, em vez de tomar o Partido Republicano ou o Partido Democrata como o culpado, devemos focar no verdadeiro vilão da peça, o Partido da Guerra.

Os estados defendem a guerra com argumentos fracos e propaganda duvidosa. Você se lembra de quando Kanan Makiya anunciou que as tropas do governo dos EUA seriam ``saudadas com doces e flores'' ao invadir Bagdá? Quando Colin Powell disse ao Conselho de Segurança da ONU que Saddam Hussein possuía armas de destruição em massa? Quando Tony Blair disse que Saddam Hussein poderia lançar um ataque nuclear em 45 minutos? Quando o governo Bush anunciou: ``Missão Cumprida''? Os especialistas da mídia tradicional deram a entender que não havia outra escolha a não ser invadir o Iraque, que os eventos após a invasão ocorreriam sem problemas, que a democracia emergiria em todo o Oriente Médio, que a benevolência dos invasores da coalizão seria universalmente aplaudida. Os fatos têm sido, para dizer o mínimo, diferentes. E a contínua dominação do Partido da Guerra significa que podemos esperar mais guerras injustas e sem sentido e mais ataques à nossa liberdade e ao nosso bolso.

\section{Sobre não enviar os fuzileiros navais}

Enquanto eu escrevia a primeira versão deste capítulo, estava ouvindo um sensato comentarista anarquista objetando à tentativa do governo dos EUA de interferir na ``política interna de outra nação''. Que tipo de interferência? A aprovação de uma resolução sem força vinculante expressando apoio a manifestantes antigovernamentais no Irã.

Acredito que haja boas razões para se opor a esse tipo de resolução. Mas certamente não é devido a qualquer preocupação com os ``assuntos internos'' de qualquer estado. Afinal, para os anarquistas, todos estados são definitivamente ilegítimos: suas fronteiras são criações arbitrárias, seus governos bandidos disfarçados (ou não). O fato de que algo ocorre do outro lado da fronteira de um estado não é motivo algum para não criticá-lo. As pessoas têm todo direito de criticar injustiças em qualquer lugar. E, de fato, temos boas razões para ativamente nos opor à ação estatal injusta em qualquer lugar.

Mas veja que não estamos realmente falando da oposição por indivíduos ou grupos íntegros: estamos falando da oposição por um governo, o governo dos Estados Unidos. É o governo mais poderoso do planeta, com um enorme exército -- e uma história de participação ativa no autoritarismo ao redor do mundo. 

Então quando o governo dos EUA adota uma posição acerca dos manifestantes nas ruas do Irã, é difícil não ver isso como ligado aos esforços contínuos para intervir na política iraniana e implantar um governo mais amigável. É por isso que, como anarquista, eu me oponho à posição oficial do governo dos EUA sobre o Irã: não porque eu quero que o povo iraniano seja governado pelos mulás, mas porque não quero que seja governado por ninguém, incluindo fantoches do governo dos EUA.

\section{Estados, guerras e exércitos permanentes}

``Eu abomino e detesto a ideia de um governo onde há um exército permanente'', disse certa vez o antigo político americano George Mason. Para Luther Martin, um dos contemporâneos de Mason, era óbvio que ``quando um governo deseja privar seus cidadãos da liberdade e reduzi-los à escravidão, ele geralmente faz uso de um exército permanente''. Elbridge Gerry descreveu um exército permanente como ``a ruína da liberdade''. Thomas Jefferson incluiu a proibição dos exércitos permanentes entre ``as restrições contra o mal que nenhum governo honesto deveria recusar''. Acredito que eles tinham alguma razão.

Exércitos permanentes -- estou usando ``exércitos'' [\emph{``armies''}] como um termo conveniente para me referir a todos os tipos de forças militares -- são exércitos não simplesmente convocados para fins defensivos em tempo de guerra, mas mantidos em tempo integral pelo estado. Em princípio, é claro, tais exércitos poderiam existir sem estados, e um estado não precisa ter um exército permanente. Então um argumento contra um exército permanente não é um argumento decisivo a favor do anarquismo. Mas há uma conexão natural entre exércitos permanentes e estados. Estados podem se dar o luxo de manter grandes exércitos permanentes porque podem apoiá-los usando dinheiro de impostos. Exércitos permanentes têm maior probabilidade de atrair membros quando mantidos por estados porque os estados podem gastar enormes quantias para promover o alistamento. Além disso, os estados podem estimular as pessoas a se juntarem a exércitos permanentes fazendo uso de propaganda que manipula a tendência natural das pessoas a favor da lealdade, focando essa lealdade no próprio estado em vez de comunidades locais genuínas. Ao convencer as pessoas de que ele merece sua lealdade e que essa lealdade é mais bem expressa por meio do serviço militar, o estado é capaz de manipular as pessoas a ingressarem ao exército de uma forma que, digamos, uma associação comunitária ou empresa que fornecesse serviços de segurança não seria capaz.

\subsection*{A existência de exércitos permanentes faz com que seja mais fácil que estados travem guerras}

Os estados precisam de exércitos permanentes para conduzir suas máquinas de guerra. Se uma sociedade conta com uma milícia voluntária para se defender contra invasões ou conflitos civis violentos, a milícia fará o seu trabalho e depois se dissolverá. Membros da milícia apenas participarão das suas atividades se acreditarem que estão fazendo algo importante, se for para defender seus interesses ou os de seus vizinhos ou amigos. Um exército permanente, em contrapartida, está em prontidão para ser mobilizado. Um estado não precisa esperar por um ataque ou pela ameaça de um ataque para fazer uso de seu exército permanente. Se ele desejar invadir outro país, as ordens para a invasão simplesmente têm de ser repassadas ao longo da cadeia de comando militar e soldados e marinheiros em tempo integral iniciarão a execução das tarefas que eles foram treinados a executar.

\subsection*{A existência de exércitos permanentes facilita a manipulação da lealdade dos soldados}

Os soldados de um exército permanente são aculturados a pensar que desempenham tarefas inestimáveis para sua comunidade, a confiar e obedecer a seus superiores, a ver o que fazem como uma expressão de lealdade. Quando o estado estalar os dedos, eles provavelmente obedecerão. Em contrapartida, os membros voluntários de uma milícia não são principalmente soldados ou marinheiros -- eles são professores, eletricistas, contadores, metalúrgicos, advogados, jornalistas e encanadores. Eles têm as suas próprias vidas para viver. Isso significa que eles provavelmente resistirão às tentativas de persuadi-los a abandonar suas vidas normais para participar da guerra \emph{e} que eles terão tido consideráveis oportunidades de desenvolver perspectivas independentes sobre o que está acontecendo no mundo. Eles não terão sido submetidos à constante propaganda que os lembra de seu papel como guardiões da liberdade, da democracia ou da Pátria.

\subsection*{Estados efetivamente fazem propaganda a favor da guerra}

Um estado tem os recursos necessários para submeter continuamente não apenas soldados, mas todas pessoas, à constante propaganda em apoio à guerra. Uma vez que eles optam pela guerra, os líderes do estado podem usar a enorme quantia de recursos originados de impostos à sua disposição para convencer o público de que o perigo é iminente, que os pretensos inimigos da guerra que eles pretendem empreender são maus, que a justiça está do seu lado e que a vitória é certa. De fato, grupos menores não estatais também podem fazer propaganda a favor da guerra em uma sociedade sem estado. Mas tais grupos simplesmente não teriam os recursos, a influência ou a estatura do governo de um estado.

Os funcionários do estado desfrutam de um prestígio estabelecido irracionalmente que aumenta sua credibilidade quando fazem propaganda a favor da guerra. Além disso, eles afirmam repetidamente que estão em posse de informações que justificam a ação militar que as pessoas comuns não têm e que, portanto, eles devem simplesmente ser confiados. Ao mesmo tempo, eles podem afirmar que não estão dispostos a divulgar essas informações de forma mais ampla porque fazê-lo iria, supostamente, comprometer a segurança de todos. ``É para o seu próprio bem'', eles podem anunciar. ``É por isso que não contaremos para você. Mas acredite na nossa palavra.''

\subsection*{Estados mobilizam exércitos de escravos}

Os estados têm, é claro, outro método para aumentar o tamanho dos exércitos permanentes: eles podem escravizar pessoas. Autocracias mais antigas literalmente classificavam as pessoas como escravos, como propriedade do estado, e às vezes as forçavam a guerrear (é claro, armar escravos é sempre arriscado, já que eles podem se revoltar). Os estados modernos raramente chamam alguém de escravo. Mas eles estão muito dispostos a exigir trabalho involuntário de pessoas no exército. O estado não precisa se preocupar se as pessoas considerarão uma guerra como justa ou necessária, se verão as medidas tomadas para garantir que eles possam desempenhar suas funções militares de maneira eficaz como adequadas, ou se eles têm outros compromissos que conflituam com o serviço militar. (Bem, é verdade que Dick Cheney tinha ``outras prioridades'' durante a guerra do Vietnã. Mas nesse caso era diferente.) Ele é simplesmente livre para anunciar às pessoas -- como é de seu costume fazer -- que elas são \emph{obrigadas} a ingressar no exército, a trabalhar por baixo salários enquanto aceitam riscos enormes.

\subsection*{Exércitos permanentes originam servidão}

Os exércitos permanentes criados pelo estado também promovem o respeito e confiança nas autoridades. Pessoas comuns que defendem sua comunidade enquanto lutam em uma milícia retornam ao seus trabalhos normais e continuam tomando suas próprias decisões. Pessoas que são treinadas especificamente para serem soldados aprendem a obedecer, a fazer o que lhes é mandado sem questionar. Essa é uma das razões pelas quais muitos dos fundadores dos Estados Unidos não confiavam em exércitos permanentes: o tipo de mentalidade necessária para ser um bom soldado não é a mesma para ser livre e autônomo.

\subsection*{Estados podem fazer uso de exércitos permanentes para reprimir a dissidência}

Os estados também podem fazer uso dos exércitos permanentes para a repressão. Embora a afirmação típica seja de que as forças militares são mantidas para operações defensivas, elas podem facilmente ser usadas para impedir a dissidência e manter pessoas comuns sob controle. Novamente, o ponto não é que grandes grupos de indivíduos armados também não poderiam fazer isso em uma sociedade sem estado, mas que o estado tem os recursos necessários para investir na manutenção de uma máquina militar poderosa que ele pode passar a efetivamente usar, ao seu critério, para espancar, prender, torturar e matar dissidentes.

\subsection*{Serviço militar e a cultura da violência policial}

Outro problema: muitas pessoas que deixam o serviço militar tornam-se policiais. Não acredito que haja qualquer ligação imediata entre estar em serviço militar ativo ou ser um reservista e se tornar um policial violento e abusivo. Mas é muito fácil que policiais tratem pessoas comuns como inimigas, e alguns tipos de experiências militares podem reforçar essa tendência. Organizações militares e ambientes de combate de alta pressão podem estimular a desumanização de aparentes inimigos. E as pessoas podem levar isso para a vida civil. Isso não significa, de forma alguma, que o serviço militar transforma todos em bandidos violentos. Meu pai serviu na Segunda Guerra Mundial e não constituiu, até onde eu sei, ameaça alguma aos seus vizinhos quando voltou; eu não temo pela minha vida perto dos meus amigos com histórico de serviço militar. Eu apenas me preocupo com as maneiras que algumas pessoas possam estar propensas, em virtude de alguns tipos de experiências militares, a se comportar de maneira cruel e violenta tanto enquanto veste a farda militar quanto na vida civil.

Convocar um grande número de pessoas para o serviço militar é muitas vezes aclamado como uma maneira de construir conexões entre pessoas para além das barreiras criadas pela classe, geografia, cultura e etnia: pense no vasto número de filmes de guerra que focam no tema do serviço militar como um enfraquecimento dessas barreiras. Mas a experiência compartilhada do serviço militar também serve como uma forma de encorajar um grande número de pessoas a aceitar a violência estatal e a se identificar, tal como seus servos, com os objetivos imperiais e militares do estado.

O passado de serviço militar de policiais dificilmente é a única ligação entre abusos do poder militar e abusos do poder policial. Agências policiais estão sendo cada vez mais incentivadas a trabalhar com o exército e as pessoas estão sendo levadas a pensar no policiamento e no uso da força militar como dois lados da mesma moeda. Enquanto eu escrevia este capítulo, surgiram evidências de que o governo Bush havia seriamente considerado mobilizar unidades militares para prender suspeitos de terrorismo em solo americano, em uma degradação ainda maior da tradicional -- e muito sábia -- proibição do uso de militares para a aplicação doméstica da lei. Acredito que temos muitas razões para ficarmos preocupados quando o estado usa o poder das suas forças militares contra o seu próprio povo. Mas não deveríamos nos surpreender: a ameaça da força que é tão claramente representada pelo uso doméstico de um exército permanente é, afinal, o que faz do estado o estado.

Thoreau estava sem dúvidas correto: ``objeções que foram feitas contra um \emph{exército} permanente \ldots\ também podem no fim das contas serem feitas contra um \emph{governo} permanente. O exército permanente é apenas um braço do governo permanente. O próprio governo \ldots\ está igualmente passível de ser abusado e pervertido\ldots\,''\footnoteA{Ahimsa Dhamapada et al., ``1.3 What is a Standing Army,'' \emph{Lawful Arrest/Search/ Seizure FAQ} (n.p., Feb. 21, 2009) <\url{http://stason.org/TULARC/society/lawful-arrest/1-3-What-is-a-Standing-Army.html}> (July 1, 2010). Essa é também a fonte das citações a respeito de exércitos permanentes no início dessa seção.}

O problema real, subjacente, não é o exército permanente -- por mais perigoso que ele seja -- mas o estado que o mantém. Estados podem manter exércitos permanentes quando outros não poderiam porque podem extrair enormes quantias de dinheiro das pessoas através do sistema tributário, da criação inflacionária da moeda e de outros meios que dependem do seu vasto poder sobre as vidas e bens das pessoas. Os exércitos do estado ajudam a manter e estender o seu poder, envolvendo-se em terríveis abusos ao longo disso -- abusos pelos quais eles raramente (contanto que não sejam derrotados) são responsabilizados. E servir nesses exércitos prepara as pessoas para participar, como policiais, após cumprir o serviço militar, na supressão da oposição ao estado.

\section{A face amigável do império}

O poder imperial muitas vezes se expande sob a mira de uma arma. E mesmo quando esse não é o caso, a consciência de que uma força militar sempre está disponível para respaldar as demandas de uma potência imperial ajuda a garantir que as pessoas se sujeitarão a essas demandas. Os Estados Unidos, por exemplo, mantêm quase mil bases militares ao redor do mundo, prontas para agir. Os estados também costumam usar meios violentos, mas menos diretos ou explícitos, de impor a sua vontade aos outros. Um estado pode fazer propaganda, financiar ou ativamente planejar um \emph{coup d'etat} em outro estado.

O fato dos estados possuírem recursos financeiros substanciais e poderem empregar muitas pessoas competentes no uso da violência e de artimanhas faz com que seja mais fácil que eles manipulem eventos fora das suas fronteiras. E o fato das sociedades que eles parecem manipular serem quase todas governadas por estados significa que eles podem concentrar os seus esforços em um número relativamente pequeno de pessoas: quando um estado governa uma sociedade, tudo o que se precisa fazer para alterar a sociedade é mudar o pequeno número de pessoas que tomam decisões relevantes para o estado -- ou mudar suas atitudes ou motivações. Muitas vezes, é claro, não é necessária muita mudança. As elites que dirigem diferentes estados muitas vezes têm mais em comum umas com as outras do que com as pessoas das quais elas fingem agir a favor.

Governantes imperiais e seus parceiros corporativos rapidamente persuadem as elites locais de estados clientes a investir o dinheiro tomado de pessoas comuns através da força ou fraude em projetos de elefante branco que talvez inflem o ego dos políticos mas que principalmente beneficiam não as economias locais, mas multinacionais politicamente conectadas. Essas mesmas multinacionais se beneficiam quando as elites das sociedades emergentes expulsam os camponeses de suas terras, deixando-lhes pouca escolha a não ser aceitar empregos arriscados e mal remunerados e pouca possibilidade de negociar por melhores condições. 

Muitas vezes é bem mais barato -- financeira e politicamente -- para um estado poderoso manter um império à distância. Em vez de realmente ocupar outros estados, posicionando tropas lá e assumindo o custo de administrá-las, um estado poderoso pode simplesmente delegar essa responsabilidade às elites locais. Ele pode consolidar laços com seus estados clientes por meio de tratados, relações econômicas e fornecimento de assistência militar, técnica e de outros tipos. Então, ele pode proclamar com entusiasmo que, diferente de outros estados -- seja qual deles que ele deseje se distinguir --, \emph{ele} não é uma potência imperial, mas sim um generoso parceiro.

Existem óbvias vantagens de Relações Públicas em proceder dessa maneira. É fácil convencer os ingênuos de que o império na verdade não é um império. A abordagem do império moderno, atualizado, de ganhar poder de maneira aparentemente consensual contrasta fortemente com a necessidade do uso da força nua e crua do império tradicional. Mas o contraste é mais aparente do que real.

O império tem cada vez mais uma face amigável -- tanto porque isso impede que o seu próprio povo e aqueles que ele procura controlar em outros lugares se indisponham demais sob seu domínio, quanto porque alguns de seus líderes, sem dúvidas, desejam acreditar que seu exercício de poder sobre os outros realmente é para o próprio bem deles. E um império de face amigável é sem dúvidas menos terrível, em certo sentido, do que um império que sujeita os outros sob a mira de uma arma. Mas quer sejam eles mantidos através da guerra, da ameaça de guerra, da manipulação por debaixo dos panos ou apenas das cômodas relações com as elites locais, os impérios injustamente redistribuem recursos para as elites privilegiadas e estimulam atos contínuos de violência e expropriação.

\section{Escolhendo o estado ou escolhendo a paz}

Estados matam, mutilam e destroem. A capacidade humana para a violência é amplificada, glorificada e protegida de críticas pela máquina militar do estado. E as guerras do estado, por sua vez, fornecem novas oportunidades para o estado crescer -- não apenas pela expansão de suas fronteiras, mas pela ampliação de seu aparato administrativo (burocracias nunca encolhem) e do escopo de seu controle sobre as pessoas que habitam o território que ele reivindica. As forças militares permanentes que somente um estado poderia se dar o luxo de manter instiga medo em território nacional e no exterior e treina as pessoas a obedecer mestres arbitrários. E o poder diplomático e econômico -- ainda respaldado pela força e exercido junto das elites econômicas que são comparsas do estado -- permite que o estado imperial exerça influência para além das suas fronteiras de maneira pouco custosa e ao mesmo tempo mantendo a ilusão da benevolência.

Você pode ter o estado, com seu anseio imperial, sua capacidade de construir impérios e sua máquina de guerra destrutiva. Ou você pode ter a anarquia. As pessoas de uma sociedade sem estado provavelmente não seriam mais virtuosas do que as pessoas do atual mundo de estados violentos. Mas elas pelo menos não teriam oportunidades para a guerra, expansão imperial e os incentivos estruturais para engajar em ambas, criados pela existência de estados. Sem o estado, sem dúvidas ainda haverá atos de violência. No entanto, sem exércitos permanentes financiados por impostos e todos os outros recursos que os estados podem fazer uso, o potencial destrutivo da violência humana será significativamente reduzido. E sem mecanismos estatais de controle para se dominar através da guerra, o incentivo para a violência em grande escala será significativamente diminuído. A anarquia não oferece uma utopia. Mas oferece mais paz e segurança do que o estado.