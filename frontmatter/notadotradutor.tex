% Nota do Tradutor

\chapter{Nota do Tradutor}

\lettrine[lines=2]{\textcolor{LettrineColor}{\textbf{T}}}{endo minhas convicções} sobre propriedade intelectual em mente, reservei um tempo para, sozinho, traduzir e editar o livro \emph{The Conscience of an Anarchist}, de Gary Chartier. O que quero dizer com isso é que essa é uma tradução livre, não oficial, de iniciativa própria, e que portanto não está diretamente associada nem a Gary Chartier, nem à \emph{Cobden Press}, nem ao \emph{Center for a Stateless Society}. Apesar disso, eu, como tradutor, mas acima de tudo como alguém apaixonado pelo anarquismo, gostaria de agradecer aos citados por toda energia dedicada à causa da libertação humana. Pois são esses esforços que, na forma de esperança, nos guiam em direção a um mundo mais significativo.

Como foi um empreendimento de uma pessoa só, é muito provável que essa tradução esteja repleta de erros. Ainda assim, dei o meu melhor para tornar esse projeto o mais profissional possível. Para isso, revisei essa versão algumas vezes e adicionei notas de tradução (indicadas por ``N.T.'' e enumeradas em algarismo romano, ou então sinalizadas entre colchetes) para explicar expressões idiomáticas ou o que fosse necessário. Também dei muita importância para a edição do livro, que inclui uma capa feita pelo meu colega Eliãn -- agradeço a sua ajuda --, para o contentamento estético do leitor. Nesse quesito, acredito que o resultado tenha sido bem satisfatório.

Por fim, essa tradução está sendo publicada em nome do \emph{Rota Libertária}, um blog de escrita e tradução de artigos anarquistas e libertários (apesar de não se limitar a isso). Caso o assunto seja do interesse, convido o leitor a navegar pelos materiais ali publicados. O código fonte desse livro, escrito em \LaTeX, será disponibilizado, através do \emph{Rota Libertária}, em um repositório público no GitHub.

\begin{flushright}
\large\emph{Igor R.}
\end{flushright}
