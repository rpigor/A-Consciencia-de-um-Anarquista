% Introdução

\chapter[Introdução]{Introdução\\\vspace{-1.75ex}\hrule\vspace{0.25ex}\normalfont\LARGE Abra sua mente para o anarquismo}

\lettrine[lines=2]{\textcolor{LettrineColor}{\textbf{C}}}{omo uma ideia}, o anarquismo é a convicção de que as pessoas podem e devem cooperar pacífica e voluntariamente. Como um programa político, é o projeto de agir sem o estado.

Como os governos são fundamentados no uso da força, os anarquistas defendem que nenhum governo real é legítimo e que, de qualquer forma, estaríamos melhor sem o estado. Os anarquistas rejeitam qualquer tipo de autoridade adquirida ou mantida por meio de violência agressiva ou da fraude. De maneira mais ampla, muitos anarquistas -- incluindo eu -- defendem que os mesmos ideais que motivam sua oposição à violência agressiva os levam a contestar as instituições sociais e os padrões culturais que subordinam, excluem ou empobrecem as pessoas, complicam suas vidas, ou as forçam a uma conformidade que atordoa a alma.

As pessoas podem e devem organizar suas interações das suas próprias maneiras. Podemos nos defender contra agressões; não precisamos do estado para nos forçar a não matar uns aos outros. E não precisamos da ajuda do estado para coordenar as nossas interações. Trabalhando juntos, podemos construir vidas significativas e comunidades habitáveis.

\section{Anarquismo como visão positiva}

Às vezes, as pessoas usam o rótulo anarquista, ou hasteiam bandeiras pretas anarquistas, quando seu principal objetivo é apenas o de espalhar caos. Mesmo pessoas mais sensatas às vezes podem agir como se ``anarquia'' fosse apenas outra palavra para desordem. Mas o anarquismo como eu o entendo é sobre o melhor tipo de ordem que poderíamos imaginar: o tipo de ordem que surge voluntariamente, espontaneamente, à medida que as pessoas trabalham juntas de forma criativa para moldar as suas vidas e planejar os seus futuros. Anarquia é o que surge quando a ordem social se origina não dos canos das armas do estado, mas da cooperação pacífica e voluntária.

Grosso modo, um estado é uma organização que afirma ter autoridade legítima sobre quem usa a força em um determinado território e que faz um trabalho ao menos moderadamente eficaz de manter a violência não aprovada sob controle. (Mais sobre isso depois.) O estado no sentido moderno esteve conosco por mais de trezentos anos, e estados dos mais variados tipos são muito mais antigos do que isso. Então é fácil tratar a existência dos estados como inevitável. Mas, para os anarquistas, não há nada de necessário no estado. Os estados persistem por causa do interesse próprio de pessoas poderosas que os administram ou os manipulam e porque as pessoas comuns não perceberam o seu próprio poder de imaginar e implementar alternativas.

Neste livro, quero ajudar a afrouxar o controle que o estado ainda exerce sobre a imaginação das pessoas. Quero salientar que, como no famoso conto de Hans Christian Andersen, o imperador realmente não tem nada de especial. Quero encorajá-lo a mudar o seu ponto de vista -- a passar a ver o estado como um grupo de pessoas nem um pouco diferentes dos seus vizinhos, sem mais autoridade inerente, sem maior direito de dizê-lo o que fazer. (Claro, seus vizinhos provavelmente não irão ameaçá-lo com armas se você não fizer o que eles dizem. Mas essa diferença dificilmente é favorável ao estado.) Quero enfraquecer o mito de que o estado \emph{nos} representa em qualquer sentido significativo, de que, quando políticos e generais agem, eles estão agindo em \emph{nosso} nome. Quero enfatizar o fato de que as pessoas que tomam e executam as decisões estatais estão seguindo as suas próprias agendas, frequentemente em conflito com as nossas -- assim como as pessoas poderosas em grandes empresas e outras instituições semelhantes -- e que não temos razões para tratá-las com reverência, para vê-las como outra coisa senão \emph{pessoas comuns} com direitos \emph{exatamente como os nossos}.

Isso não é uma cartilha, um trabalho estritamente acadêmico sobre filosofia ou economia ou ciência política ou história, embora seja informado pelos resultados da investigação em todas essas disciplinas. É um \emph{manifesto}, um chamado para a ação: não para mais violência, que é apenas o reflexo da própria destrutividade do estado, mas para o empreendimento criativo de imaginar um novo tipo de sociedade e começar a construí-la aqui e agora, bem debaixo do nariz das pessoas no poder.

\section{Por que eu sou anarquista}

Eu sou anarquista por diversas razões.

Eu sou anarquista porque acredito que \emph{não existe direito natural de governar}. Acredito que as pessoas são iguais em dignidade e valor essencial, o que significa, por sua vez, que elas têm igual \emph{status} moral. Isso faz com que seja difícil justificar dar a algumas pessoas -- aqueles que governam o estado e aqueles que aplicam as decisões dos governantes -- direitos que outros não têm. E eu sou anarquista porque acredito que \emph{o estado não tem legitimidade}. Algumas pessoas argumentam que os governantes merecem ter mais direitos do que aqueles que eles governam porque seus sujeitos consentiram e continuam a consentir com a sua autoridade. Mas acredito que eles não o fazem. Falarei mais sobre essas razões de ser anarquista no \hyperref[chap:1]{Capítulo 1}.

Eu sou anarquista porque acredito que \emph{o estado é desnecessário}. Tento explicar o porquê no \hyperref[chap:2]{Capítulo 2}. Os estatistas costumam afirmar que ter um estado é a única maneira de ter uma sociedade pacífica. Eu discordo, tanto em bases teóricas quanto empíricas. Acredito que instituições não governamentais podem fornecer os serviços que o estado fornece -- mas de maneira mais eficiente e flexível; e há evidências sólidas de que elas são capazes de fazer isso. Além disso, tenho convicção de que se o estado tem o poder para fazer coisas boas, mesmo coisas muito boas, muito úteis, muito importantes, ele quase inevitavelmente usará esse poder de maneiras autoritárias: ele usará o poder que possui para regular a vida das pessoas -- e para adquirir mais poder.

Como enfatizarei no \hyperref[chap:3]{Capítulo 3}, eu sou anarquista porque \emph{o estado vira a balança a favor das elites privilegiadas e contra as pessoas comuns.} (Ao contrário do que pessoas do tipo ``governos são bons'' dirão, isso é exatamente o que ele é designado a fazer.) O estado tende a promover ineficiências através de subsídios, monopólios, patentes, tarifas e outros mecanismos que permitem que as elites se livrem de pagar os custos reais do que elas fazem. Ele força as pessoas comuns a arcarem com os custos das decisões da elite e a ajustarem as suas preferências e comportamentos para se adequar às maiorias conformistas. Acredito que uma sociedade sem estado teria mais chance do que a nossa de promover a eficiência e produtividade e evitar variedades de hierarquia e exclusão que os estados tendem a promover e proteger. Qualquer um que se preocupe com o poder dos ricos e grandes empresas, a prosperidade das pessoas comuns e o bem-estar dos pobres e vulneráveis deve dizer um intenso ``não'' ao estado.

Eu sou anarquista porque \emph{o estado tende a ser destrutivo}. Ele se envolve em guerras e saques, e parece estar sistematicamente envolvido no aumento do nível de violência e injustiça para além das suas fronteiras -- que são, é claro, elas mesmas criações do estado (há mais sobre isso no \hyperref[chap:4]{Capítulo 4}). Acredito que uma sociedade sem estado apresentaria muito menos violência em larga escala do que a nossa.

Eu sou anarquista porque \emph{o estado restringe a liberdade individual} -- como uma forma de manter a ordem, beneficiar os privilegiados, preservar seu próprio poder ou subsidiar as preferências moralizantes de algumas pessoas. E há uma conexão natural entre o poder do estado e a imposição de limites à liberdade. Ofereço alguns exemplos no \hyperref[chap:5]{Capítulo 5}.

Eu sou anarquista porque desejo uma sociedade marcada pela diversidade, exploração e experimentação, porque acredito que os estados impõem a conformidade e resistem à criatividade e porque acredito que \emph{uma sociedade sem estado proporcionaria oportunidades para que as pessoas explorem diversas maneiras de viver vidas plenas e prósperas} e apresentem os resultados da sua exploração. Apresento esse ponto com mais detalhes no \hyperref[chap:6]{Capítulo 6}: falarei sobre a forma de vida sem o estado e delinearei alguns dos passos concretos que podemos dar para acabar com a opressão e violência e começar a criar um novo mundo.